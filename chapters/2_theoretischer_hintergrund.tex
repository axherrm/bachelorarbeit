\chapter{Theoretischer Hintergrund}
\label{chap:theoretischer-hintergrund}

Es liegt bereits einige Forschung zum Architekturprinzip der Microservices vor.
Folgend wird sowohl auf grundlegende Literatur zu Microservices verwiesen werden, als auch verwandte Arbeiten erwähnt.
Dabei wird in Literatur zur Migration von monolithischen Systemen zu Microservices-basierten und in Literatur zum Refactoring von Microservices-basierten System unterschieden.

\section{Vergleich von monolithischen und Microservices-Systemen}

\section{Migration zu Microservices}

In dieser Thesis wurde sehr vordergründig das Microservices Migration Framework des ESE  untersucht.
Dieses ist aufgrund vorheriger Arbeiten wie \Citet{10.1007/978-3-030-06019-0_10} entstanden.

In 2019 kategorisieren \Citet{10.1007/978-3-030-06019-0_10} zehn verschiedene Migrations-Methoden.
Dabei wird ein Mängel an praktisch anwendbaren Methoden hervorgehoben, die gute Werkzeugunterstützung und Metriken zur Verifikation der Ergebnisse bieten.

Als Folge darauf stellen \Citet{fritzsch2022architecturecentric} 2022 dann die Planung ein solchen Frameworks vor. Der Migrationsprozess mit Framework und dessen Arbeitsweise in drei Phasen wird bereits beschrieben. Auch wird die Entwicklung eines Web-basierten Werkzeugs erwähnt, das das Framework benutzt und Entwickler bei der Migration unterstützt.

\Citet{10.1145/3242163.3242164} beschreiben geläufige Gefahren bei der Migration zu Microservices-Architekturen.
Dabei  werden fünf architektonische Fehlermuster und vier Fehlermuster bei der Migration aufgeführt.
Außerdem werden Lösungsmöglichkeiten für jedes Muster beschrieben.

\section{Refactoring}

Obwohl in dieser Thesis ein Framework zur Migration zu Microservices im Mittelpunkt steht, sieht man schnell, dass die Migration eng verwandt mit dem Refactoring ist.
Deswegen werden folgend ein paar Quellen vorgestellt, die das Refactoring von Microservices-Architekturen behandeln.

\Citet{https://doi.org/10.1002/spe.2974} beschreiben in einem Artikel zur $\mu$TOSCA toolchain, wie sie mit dem  $\mu$TOSCA Modell Architekturen von Microservices-basierten Systemen mit dem OASIS Standard TOSCA darstellen können.
Außerdem beschreiben sie die Möglichkeit, das $\mu$TOSCA Modell eines Systems durch das zugehörige Kubernetes Deployment automatisch zu ermitteln.
Das $\mu$TOSCA Modell einer Applikation soll dann dazu genutzt werden, potentielle architektonische Probleme zu finden.
Für das Ermitteln des $\mu$TOSCA Modells und das Identifizieren von Architektur Smells stellen sie zwei Werkzeug Prototypen ($\mu$Miner und $\mu$Freshener) vor.

\section{Ähnliche Forschung}

\section{Microservices Migration Framework}

\section{jadice flow}

\emph{jadice flow} ist ein Produkt der Firma \emph{levigo solutions}\footnote{\url{https://solutions.levigo.de/}}.
\emph{levigo solutions} beschäftigt über 30 Mitarbeiter, von denen ungefähr sechs seit vier Jahren an \emph{jadice flow} arbeiten.
\emph{jadice flow} basiert bereits auf einer Microservices-Architektur, ist jedoch erst die erste Generation nach der Überführung des vorherigen Monoliths \emph{jadice server}\footnote{\url{https://jadice.com/produkte/server/}}.
Beide Produkte bieten Workflow-basierte Dokumentenverarbeitung an, bei der mit komplexen Datenströmen umgegangen werden kann.
Durch den Umstieg auf eine Microservices-Architektur können in der neuen Generation einzelne Verarbeitungsschritte skaliert werden.
Der Großteil der Services ist in Java geschrieben, doch durch den Betrieb in Containern ist jadice flow prinzipiell unabhängig von der Programmiersprache.

Einen beispielhaften Workflow stellt der \emph{E-Mail converter} dar.
Dabei werden E-Mails in Einzelteile aufgeteilt (Körper, Anhänge) und diese dann einzeln in das Zielformat konvertiert.
Die einzelnen Arbeitsschritte dabei können parallelisiert werden und separat skaliert werden.
Am Ende werden die Ergebnisse kombiniert.
Im Gegensatz zu einfacher PDF-Darstellung von E-Mails in geläufigen Programmen bietet \emph{jadice flow} zusätzliche Features, wie die tiefe Analyse und Darstellung von Archiven in Anhängen.

Da \emph{jadice flow} die erste Generation in Microservices-Architektur ist, wird noch einiges Verbesserungspotential vermutet.
Durch die Analyse von \emph{jadice flow 1.0} mithilfe des Frameworks und weiterer Forschung im Rahmen dieser Bachelorarbeit soll ein Refactoring-Prozess für \emph{jadice flow 2.0} angestoßen werden.

Da das Framework für die Anwendung auf Monolithen gedacht ist, soll in dieser Arbeit die Anwendung des Frameworks auf ein bereits migriertes System angewendet werden, um dessen Eigenschaften weiter zu verbessern.
Dadurch wird erforscht, inwiefern sich Framework und Werkzeug auf eine bereits existierende Microservice-Architektur anwenden lassen.
\todo{Ändern je nachdem, ob jadice server verwendet wird}Falls es deshalb für diesen Anwendungsfall zielführend ist, wird zusätzlich die ursprüngliche monolithische Anwendung von \emph{jadice flow} (\emph{jadice server}) hinzugezogen.
Die Ergebnisse dessen könnten dann mit \emph{jadice flow} verglichen werden.



