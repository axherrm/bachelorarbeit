\chapter{Motivation}

In Zeiten, in denen viele Softwarelösungen auf Cloud-Computing basieren, ist auch die Microservices-Architektur immer geläufiger geworden.
Viele Produkte wurden innerhalb des letzten Jahrzehnts schon in diese Architektur übergeführt, hauptsächlich mit dem Ziel, von besserer Wartbarkeit und Skalierbarkeit zu profitieren~\cite{Fritzsch_2019}.
Abhängig von dem Ausmaß und der Komplexität monolithischer Systeme, kann es jedoch sehr zeit-, kostenaufwendig und kompliziert sein, diese in eine Microservices-Architektur zu migrieren.
Es liegen aktuell eine Vielzahl akademischer Forschungsarbeiten zum Thema Microservices vor, doch diese wird nur sehr selten in der Industrie benutzt~\cite{fritzsch2022architecturecentric}.

Um diese Migration zu vereinfachen und die wissenschaftliche Expertise der Industrie leichter zugänglich zu machen, hat das ESE ein Microservices Migration Framework entwickelt (nachfolgend \emph{ESE-MMF} genannt).
%, mit dem es möglich ist, den Migrationsprozess zu unterstützen.
Mit der Entwicklung des ESE-MMF, das Erkenntnisse über die Migration von monolithischen Applikationen in Richtung Microservices in die Industrie bringen soll, hat das ESE erste Schritte unternommen, die erwähnte Lücke zwischen Industrie und Forschung zu schließen.
% Mit der Entwicklung dieses Frameworks hat das ESE erste Schritte unternommen, die erwähnte Lücke zwischen Industrie und Wissenschaft zu schließen

Offen ist bisher jedoch die Anwendung des auf dem Framework basierenden Werkzeugs an einem realen Produkt, durch die das Framework und Werkzeug weiterentwickelt und verbessert werden kann.
Im Rahmen dieser Bachelorarbeit soll das ESE-MMF deshalb erstmals an einem realen Produkt evaluiert werden.
Im Zuge dessen sollen Schwachstellen der bestehenden Architektur gefunden und Optimierungspotential analysiert werden.
