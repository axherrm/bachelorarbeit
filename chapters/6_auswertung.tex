\chapter{Auswertung}
\label{chap:auswertung}

Die Auswertung dieser Fallstudie erfolgt mithilfe drei verschiedener Methoden.
Zu Beginn wird eine qualitative Bewertung des Gesamtprozesses vom Autor vorgenommen. 
Diese kann subjektiv sein und soll daher nur als erste Orientierung dienen und durch Experteninterviews unterstützt werden.
Dabei wurden drei andere an \jf beteiligte Personen über die Verwendung des \gls{arh} mit dem Produkt befragt.
Außerdem werden abschließend die Feldnotizen ausgewertet.

\section{Qualitative Bewertung des Refactoringprozesses}

In diesem Abschnitt soll der gesamte Refactoringprozess, den das \gls{mmf} leitet, bewertet werden.
Dabei werden die einzelnen Funktionen des \gls{arh} qualitativ evaluiert.
Der Prozess besteht aus den folgenden wesentlichen Funktionen:
\begin{itemize}
	\item Erfassen der \glspl{qa} in Phase 1
	\item Einstellen von Suchfiltern in Phase 2
	\item Suche nach Migrationsverfahren in Phase 2
	\item Suche nach Patterns und Best Practices in Phase 3
\end{itemize}

Diese Qualität dieser Funktionen kann über die Ergebnisse der Suchen ausgewertet werden, da die ersten beiden Punkte in die Ergebnisse der letzten beiden Punkte einfließen.
Eine hypothetisch optimale Filterfunktion würde erreichen, dass Nutzer die Suchergebnisse nach Einstellen seiner Filter hinsichtlich der Nützlichkeit für sein Ziel genau gleich ordnen würde wie der \gls{arh} es tut.
Natürlich ist eine so genaue Filterfunktion unrealistisch und nicht erreichbar.
Im Folgenden wird versucht, von den Problemen mit Ergebnissen der Suchen Rückschlüsse auf Probleme in Filterfunktion durch Filter und \glspl{qa} zu ziehen.
Wenn beispielsweise hoch platzierte Suchergebnisse vom Nutzer als schlecht bewertet werden, könnte ein Zusammenhang mit fehlenden Filtermöglichkeiten oder zu stark bewerteten Filtern bestehen.

\subsection{Phase 2}

Bei Betrachtung der Ergebnisse in Phase 2 ist auffällig, dass das am höchsten bewertete Ergebnis von \Citet{arh-result-no-filter-1} eine der am wenigsten passenden Methoden beinhaltet.
Das Verfahren wurde aufgrund des Fehlens eines konkreten \gls{sia} ausgeschlossen.
Eine Filterfunktion für das Ziel der Verfahren wäre sinnvoll, um solche Ergebnisse gar nicht erst vorgeschlagen zu bekommen.

 - Fehlende quantitative Einordnung, wie stark die Methode den jeweiligen Filter erfüllt (Ergebnis 1 so oberflächlich, dass es viele SPs anreist, aber keins wirklich spezifisch lösen kann)

\subsection{Phase 3}


\section{Experteninterviews}

In diesem Abschnitt werden die Ergebnisse der Experteninterviews beschrieben.
Die Experteninterviews wurden mit dem Ziel durchgeführt, die Anwendung des \gls{mmf}/\gls{arh} auf \jf von weiteren Personen, die nicht an der Durchführung beteiligt waren, evaluieren zu lassen.
Die genaue Methodik der Interviews ist in \cref{sec:methodik-interviews} zu finden.
Im Groben sind die Fragen der Interviews auf zwei Abschnitte unterteilt, deren Ergebnisse im Folgenden getrennt beschrieben werden.
In \cref{sec:evaluation-mmf-allgemein} wird die Auswertung der Fragen, die sich auf die Evaluation des \gls{mmf}/\gls{arh} allgemein beziehen, vollzogen.
Der hauptsächliche Teil der Auswertung findet dann in \cref{sec:evaluation-mmf-anwendung} statt, wo die spezifische Anwendung des Frameworks in dieser Arbeit evaluiert wird.

\subsection{Evaluation des \gls{mmf}/\gls{arh} allgemein}
\label{sec:evaluation-mmf-allgemein}
\subsection{Evaluation der Anwendung des \gls{mmf}/\gls{arh} auf \jf}
\label{sec:evaluation-mmf-anwendung}

\subsubsection{Evaluation der ausgewählten Migrationsmethoden für \jf}
\label{sec:evaluation-mmf-anwendung-methoden}
\subsubsection{Evaluation der \bpp für \jf}
\label{sec:evaluation-mmf-anwendung-bp-patterns}
\subsubsection{Evaluation der Anwendung des \gls{mmf}/\gls{arh} auf \jf insgesamt}
\label{sec:evaluation-mmf-anwendung-insgesamt}

\section{Feldnotizen}
Während der Anwendung des Frameworks auf \jf in dieser Arbeit wurden in den Phasen 2 und 3 strukturierte Feldnotizen erstellt, die wichtige Schritte dieser Phasen festhalten.
Die Methodik für die Erstellung der Feldnotizen ist in \cref{sec:structured-field-notes} beschrieben.
In diesem Abschnitt wird die Auswertung dieser Feldnotizen mit der in \cref{sec:methodik-auswertung-feldnotizen} erläuterten Codierung  durchgeführt.

Insgesamt wurden 18 Feldnotizen erstellt, wovon 14 in die zweite Phase und vier in die dritte Phase des \gls{mmf} fallen.
In Phase 1 wurden keine Feldnotizen formuliert.
Aufgrund der großen Anzahl von Feldnotizen in der zweiten Phase wurde die Phase für die Auswertung weiter in die übergeordneten Schritte \emph{Filterwahl} mit den \cref{feldnotiz:1,feldnotiz:2,feldnotiz:3} und \emph{Ergebnisbetrachtung} mit den \cref{feldnotiz:4,feldnotiz:5,feldnotiz:6,feldnotiz:7,feldnotiz:8,feldnotiz:9,feldnotiz:10,feldnotiz:11,feldnotiz:12,feldnotiz:13,feldnotiz:14} unterteilt.
Im Folgenden wird die Auswertung der zugehörigen Feldnotizen für jeden dieser Schritte beschrieben.

\subsection{Phase 2: Filterwahl}

In diesem Schritt wurden die Filter für die Suche nach Migrationsverfahren ausgewählt.
Die Entscheidungen dafür wurden ausschließlich in Besprechungen zwischen dem Autor und dem \gls{po} von \jf oder dem universitären Betreuer getroffen.
Dabei wurden drei Feldnotizen erstellt, deren Auswertung durch die festgelegte Kodierung in \cref{tab:auswertung-feldnotizen-1} zu sehen ist.

\begin{table}[!ht]
  \centering
  \begin{tabular}{m{2.8cm} | c c c | c}
    \toprule
    \multirow{2}{*}[0cm]{\textbf{Code}} & \multicolumn{3}{c|}{\textbf{Feldnotiz}} & \multirow{2}{*}[0cm]{\textbf{Summe} (3)} \\
     & \textbf{\fn{1}} & \textbf{\fn{2}} & \textbf{\fn{3}} & \\ \midrule
    Gut gelaufen                        & \checkmark & \checkmark & \checkmark & 3 (100\%)  \\ \hline
    Schlecht gelaufen                   & \checkmark &                       & & 1 (33\%)  \\ \hline
    Mehr gut gelaufen                   & \checkmark & \checkmark & \checkmark & 3 (100\%)  \\ \hline
    Mehr schlecht \:\:\:\:\:\: gelaufen & &                                  & & 0 (0\%)    \\ \hline
    Gutes Gefühl                        & & \checkmark & \checkmark            & 2 (67\%)   \\ \hline
    Schlechtes Gefühl                   & \checkmark & &                       & 1 (33\%)   \\
    Unsicherheit                        & \checkmark & &                       & 1 (33\%)   \\
    Frustration                         & & &                                  & 0 (0\%)    \\ \hline
    Verbesserungs\-vorschlag            & \checkmark & &                       & 1 (33\%)   \\ \hline
    Einzelarbeit                        & & &                                  & 0 (0\%)    \\ \hline
    Besprechung                         & \checkmark & \checkmark & \checkmark & 3 (100\%) \\
    \bottomrule
  \end{tabular}
  \caption[Auswertung Kodierung Feldnotizen Filterwahl]{
    Auswertung der Kodierung der Feldnotizen in Phase 2 beim übergeordneten Schritt \emph{Filterwahl}.
    Prozentuale Angaben hinter der Summe beziehen sich auf den Anteil des Auftreten des Codes zu der Anzahl der Feldnotizen.
  }
  \label{tab:auswertung-feldnotizen-1}
\end{table}


Wie in der Tabelle erkennbar ist, verlief dieser Schritt sehr positiv.
In jeder Feldnotiz wurde eine Sache vermerkt, die gut gelaufen ist und nur in einem Fall ein Problem.
Das ist auch an der Häufigkeit von 100 Prozent der Kodierung \glqq Mehr gut gelaufen\grqq{} zu sehen.

In den Einträgen der Empfindungen ist zu erkennen, dass trotz des positiven Verlaufs dieses Schritts ein Mal auch das schlechte Gefühl \glqq Unsicherheit\grqq{} festgehalten wurde.
Das hängt in diesem Fall jedoch nicht mit dem \gls{arh} oder der Migration zusammen, sondern wird auf die Unerfahrenheit des Autors mit der Anwendung der Methodik von Feldnotizen zurückgeführt.
Ansonsten wurde in 67 Prozent der Feldnotizen ein positives Gefühl vermerkt.
Zudem wurde ein Verbesserungsvorschlag kodiert, der direkt mit der Kodierung \glqq Schlecht gelaufen\grqq{} der \cref{feldnotiz:1} zusammenhängt.
Es wird die Beschreibung der Filteroptionen bemängelt.
Da der \gls{arh} noch in der Entwicklung ist, sind Mängel in der Dokumentation zu erwarten.

\subsection{Phase 2: Ergebnisbetrachtung}

In diesem Schritt wurde mithilfe der im vorherigen Schritt gesammelten Filter eine Suche nach Migrationsverfahren durchgeführt.
Für jedes der acht in Einzelarbeit betrachteten Verfahren wurde je eine Feldnotiz erstellt. 
Außerdem wurden drei Feldnotizen bei Besprechungen über die Bewertung der Verfahren für \jf erstellt.
Die Kodierung dieser ist in \cref{tab:auswertung-feldnotizen-2} zu sehen.

\begin{table}[!ht]
  \centering
  \begin{tabular}{m{2.8cm} | c c c c c c c c c c c | c}
    \toprule
    \multirow{2}{*}[0cm]{\textbf{Code}} & \multicolumn{11}{c|}{\textbf{Feldnotiz}} & \multirow{2}{*}[0cm]{\textbf{Summe} (11)} \\
     & \textbf{\fn{4}} & \textbf{\fn{5}} & \textbf{\fn{6}} & \textbf{\fn{7}} & \textbf{\fn{8}} & \textbf{\fn{9}} & \textbf{\fn{10}} & \textbf{\fn{11}} & \textbf{\fn{12}} & \textbf{\fn{13}} & \textbf{\fn{14}} & \\ \midrule
    Gut gelaufen                        & & \checkmark & \checkmark & \checkmark & & \checkmark & & \checkmark & \checkmark & \checkmark & \checkmark & 8 (73\%) \\ \hline
    Schlecht gelaufen                   & \checkmark & & & & \checkmark & \checkmark & \checkmark & & \checkmark & \checkmark & \checkmark            & 7 (64\%) \\ \hline
    Mehr gut gelaufen                   & & \checkmark & \checkmark & \checkmark & & & & \checkmark & & &                                             & 4 (36\%) \\ \hline
    Mehr schlecht \:\:\:\:\:\: gelaufen & \checkmark & & & & \checkmark & & & & & &                                                                   & 2 (18\%) \\ \hline
    Gutes Gefühl                        & & \checkmark & \checkmark & & & & & & & &                                                                   & 2 (18\%) \\ \hline
    Schlechtes Gefühl                   & \checkmark & & & & & & & & & & \checkmark                                                                   & 2 (18\%) \\
    Unsicherheit                        & & & & & & & & & & & \checkmark                                                                              & 1 (09\%) \\
    Frustration                         & \checkmark & & & & & & & & & &                                                                              & 1 (09\%) \\ \hline
    Verbesserungs\-vorschlag            & \checkmark & & & & & & & & & &                                                                              & 1 (09\%) \\ \hline
    Einzelarbeit                        & \checkmark & \checkmark & \checkmark & & \checkmark & \checkmark & \checkmark & & \checkmark & \checkmark & & 8 (73\%) \\ \hline
    Besprechung                         & & & & \checkmark & & & & \checkmark & & & \checkmark                                                        & 3 (27\%) \\
    \bottomrule
  \end{tabular}
  \caption[Auswertung Kodierung Feldnotizen Ergebnisbetrachtung]{
    Auswertung der Kodierung der Feldnotizen in Phase 2 beim übergeordneten Schritt \emph{Ergebnisbetrachtung}.
    Prozentuale Angaben hinter der Summe beziehen sich auf den Anteil des Auftretens des Codes zu der Anzahl der Feldnotizen.
  }
  \label{tab:auswertung-feldnotizen-2}
\end{table}


Die Ergebnisse dieses Schritts fallen gemischt aus.
In 73 Prozent der Feldnotizen wurden positive Aspekte notiert, jedoch wurden in fast genauso vielen Feldnotizen (64 Prozent) auch negative Aspekte festgehalten.
Durch \glqq Gut gelaufen\grqq{} und \glqq Schlecht gelaufen\grqq{} wurde in diesem Schritt bei der Ergebnisbetrachtung der Eindruck von dem jeweiligen Verfahren kodiert. 
Bei nur zwei der Feldnotizen der Ergebnisbetrachtung in Einzelarbeit wurde \glqq Gut gelaufen\grqq{} kodiert, ohne dass \glqq Schlecht gelaufen\grqq{} kodiert wurde.
Diese beiden Verfahren wurden für die Phase 3 ausgewählt.
Letztendlich wurde jedoch doppelt so oft \glqq Mehr gut gelaufen\grqq{} kodiert wie \glqq Mehr schlecht gelaufen\grqq{}.
Der Schritt kann daher trotz gewisser Schwierigkeiten als größtenteils positiv verlaufen interpretiert werden.
In den Besprechungen wurde früh festgestellt, dass die Ergebnisse Optionen enthalten, deren Anwendung potentiell möglich ist.
Das spiegelt sich auch darin wider, dass in 67 Prozent der Besprechungen mehr gut gelaufen ist und nur in einem Fall gleich viel gut wie schlecht gelaufen ist.
Die große Anzahl der Kodierungen von \glqq Schlecht gelaufen\grqq{} hängt größtenteils mit Verfahren zusammen, die sich nicht so gut für die Anwendung an \jf eignen.
Jedoch ist auch von anderen Suchmaschinen bekannt, dass nicht alle Suchergebnisse zur Vorstellung des Suchenden passen.
In der Praxis wird sich jedoch vermutlich zeigen, dass es ausreicht, ein bis zwei passende Verfahren mit \gls{arh} zu finden zu und diese zu verwenden.
Eine größere Auswahl an passenden Verfahren wäre dennoch wünschenswert.
Allerdings ist es aufgrund des frühen Stadiums des \gls{arh} wahrscheinlich, dass die Auswahl an Verfahren in Zukunft noch erweitert wird.
Außerdem bildet die durchgeführte Suche nur die Kriterien ab, die für \jf als sinnvoll befunden wurden.
Die Anzahl der sinnvollen Ergebnisse kann für andere Anwendungsfälle variieren.

Die in diesem Schritt notierten Gefühle sind relativ neutral.
In der ersten Feldnotiz (\cref{feldnotiz:4}) wurde Frustration vermerkt, da das erste Suchergebnis, für welches der \gls{arh} mit Abstand die größte Übereinstimmung angezeigt hat, als ungeeignet befunden wurde.
In der letzten Feldnotiz (\cref{feldnotiz:14}) wurde vermerkt, dass es Unsicherheiten bezüglich der zeitlichen Umsetzbarkeit der beiden ausgewählten Verfahren gibt.
Diese Gefühle überwiegen aber nicht den verzeichneten positiven Gefühlen bei der Betrachtung der zwei favorisierten Verfahren in \cref{feldnotiz:5,feldnotiz:6}.

Der in \cref{feldnotiz:4} notierte Verbesserungsvorschlag bezieht sich auf die Frustration in dieser Feldnotiz. 
Es wird vorgeschlagen, eine Filteroption zu implementieren, um unpassende Ergebnisse auszuschließen.

\subsection{Phase 3: Methodenanwendung}

In der dritten Phase des \gls{mmf} wurden lediglich vier Feldnotizen erstellt, da nur ein kleiner Teil der Phase durchgeführt wurde.
Die Umsetzung der Phase 3b wurde frühzeitig ausgeschlossen. 
Der Hauptteil der Phase 3a, welcher die Anwendung eines Migrationsverfahrens beinhaltet, konnte aufgrund von Zeitmangel und Problemen mit den Verfahren nur teilweise durchgeführt werden.
Da in dieser Thesis nicht bis zur Planung einer neuen Architektur fortgeschritten wurde, wurden auch keine \bpp ausgewählt.

Die vier vorhandenen Feldnotizen beschreiben den Versuch, die beiden als Favoriten identifizierten Migrationsverfahren anzuwenden. 
Wie in \cref{tab:auswertung-feldnotizen-3} zu sehen ist, war dieser Versuch nicht erfolgreich.
\begin{table}[!ht]
  \centering
  \begin{tabular}{m{2.8cm} | c c c c | c}
    \toprule
    \multirow{2}{*}[0cm]{\textbf{Code}} & \multicolumn{4}{c|}{\textbf{Feldnotiz}} & \multirow{2}{*}[0cm]{\textbf{Summe} (4)} \\
     & \textbf{\fn{15}} & \textbf{\fn{16}} & \textbf{\fn{17}} & \textbf{\fn{18}} & \\ \midrule
    Gut gelaufen                        &  & \checkmark &  &                                & 1 (25\%)  \\ \hline
    Schlecht gelaufen                   & \checkmark & \checkmark & \checkmark & \checkmark & 4 (100\%) \\ \hline
    Mehr gut gelaufen                   &  &  &  &                                          & 0 (0\%)   \\ \hline
    Mehr schlecht \:\:\:\:\:\: gelaufen & \checkmark & \checkmark & \checkmark & \checkmark & 3 (75\%)  \\ \hline
    Gutes Gefühl                        &  &  &  &                                          & 0 (0\%)   \\ \hline
    Schlechtes Gefühl                   & \checkmark & \checkmark & \checkmark & \checkmark & 4 (100\%) \\
    Unsicherheit                        &  & \checkmark &  &                                & 1 (25\%)  \\
    Frustration                         & \checkmark &  & \checkmark & \checkmark           & 3 (75\%)  \\ \hline
    Verbesserungs\-vorschlag            &  &  &  &                                          & 0 (0\%)   \\ \hline
    Einzelarbeit                        & \checkmark & \checkmark & \checkmark &            & 3 (75\%)  \\ \hline
    Besprechung                         &  &  &  & \checkmark                               & 1 (25\%)  \\
    \bottomrule
  \end{tabular}
  \caption[Auswertung Kodierung Feldnotizen Methodenanwendung]{
    Auswertung der Kodierung der Feldnotizen in Phase 3 beim übergeordneten Schritt \emph{Methodenanwendung}.
    Prozentuale Angaben hinter der Summe beziehen sich auf den Anteil des Auftretens des Codes zu der Anzahl der Feldnotizen.
  }
  \label{tab:auswertung-feldnotizen-3}
\end{table}

In nur einer Feldnotiz wurde etwas notiert, das gut gelaufen ist.
Dagegen ist in allen Feldnotizen etwas schlecht gelaufen.
75 Prozent der Notizen wurden mit \glqq Mehr schlecht gelaufen\grqq{} bewertet.
Jede Notiz enthielt ein negatives Gefühl, dreimal Frustration und einmal Unsicherheit. 
Ein positives Gefühl wurde nie vermerkt.

Die schlechten Ergebnisse in dieser Phase sind hauptsächlich auf Zeitgründe und die Schwierigkeit bei der Umsetzung der ausgewählten Migrationsverfahren zurückzuführen.
Eine detailliertere Beschreibung der Probleme ist in \cref{sec:anwendung-verfahren} gegeben.
Für die Probleme werden hauptsächlich zwei Quellen vermutet.
Zum einen wurden die Verfahren nur von einer Person im Detail betrachtet und versucht umzusetzen.
Diese Person ist relativ unerfahren, was das Verständnis von wissenschaftlichen Artikeln mit komplexen Algorithmen und Formeln einschränken könnte.
Auf der anderen Seite sind bestimmte Details, zumindestens im zweiten Verfahren, wenig ausführlich oder gar nicht beschrieben.
Für eine erfolgreiche Umsetzung dieses Verfahrens wird ein hoher Eigenentwicklungsaufwand eingeschätzt, um den Algorithmus und enthaltene Formeln zu verstehen und umzusetzen.

\subsection{Insgesamt}

Nachdem die Ergebnisse der Auswertung der Feldnotizen pro Phase beziehungsweise Schritt in einzelnen Abschnitten beschrieben wurden, wird in diesem Abschnitt aus einer übergeordneten Perspektive auf die Ergebnisse der Feldnotizen geblickt.
Für jeden übergeordneten Schritt wurde die Summe der Kodierung aller enthaltenen Feldnotizen zusammen in der \cref{tab:auswertung-feldnotizen} visualisiert.
\begin{table}[!ht]
  \centering
  \begin{tabular}{m{2.8cm} | M{1cm} m{0.3cm} M{1.8cm} m{0.3cm} M{1.9cm} m{0.3cm} p{0cm} | M{1.8cm}}
    \toprule
    \multirow{2}{*}[-0.3cm]{\textbf{Code}} & \multicolumn{6}{c}{\textbf{Übergeordneter Schritt}} && \multirow{2}{*}[-0.3cm]{\textbf{Summe} (18)} \\
    & \textbf{Filter\-wahl} & \hspace*{-0.2cm}(3) & \textbf{Ergebnis\-betrachtung} & \hspace*{-0.2cm}(11) & \textbf{Methoden\-anwendung} & (4) && \\ \midrule
    Gut gelaufen                        & \mcTwo{3 (100\%)} & \mcTwo{8 (73\%)} & \mcTwo{1 (25\%) } && 12 (67\%) \\ \hline
    Schlecht gelaufen                   & \mcTwo{1 (33\%) } & \mcTwo{7 (64\%)} & \mcTwo{4 (100\%)} && 12 (67\%) \\ \hline
    Mehr gut gelaufen                   & \mcTwo{3 (100\%)} & \mcTwo{4 (36\%)} & \mcTwo{0 (0\%)  } && 7 (39\%) \\ \hline
    Mehr schlecht \:\:\:\:\:\: gelaufen & \mcTwo{0 (0\%)  } & \mcTwo{2 (18\%)} & \mcTwo{3 (75\%) } && 5 (28\%) \\ \hline
    Gutes Gefühl                        & \mcTwo{2 (67\%) } & \mcTwo{2 (18\%)} & \mcTwo{0 (0\%)  } && 4 (22\%) \\ \hline
    Schlechtes Gefühl                   & \mcTwo{1 (33\%) } & \mcTwo{2 (18\%)} & \mcTwo{4 (100\%)} && 7 (39\%) \\
    Unsicherheit                        & \mcTwo{1 (33\%) } & \mcTwo{1 (09\%)} & \mcTwo{1 (25\%) } && 3 (17\%) \\
    Frustration                         & \mcTwo{0 (0\%)  } & \mcTwo{1 (09\%)} & \mcTwo{3 (75\%) } && 4 (22\%) \\ \hline
    Verbesserungs\-vorschlag            & \mcTwo{1 (33\%) } & \mcTwo{1 (09\%)} & \mcTwo{0 (0\%)  } && 2 (11\%) \\ \hline
    Einzelarbeit                        & \mcTwo{0 (0\%)  } & \mcTwo{8 (73\%)} & \mcTwo{3 (75\%) } && 11 (61\%) \\ \hline
    Besprechung                         & \mcTwo{3 (100\%)} & \mcTwo{3 (27\%)} & \mcTwo{1 (25\%) } && 7 (39\%) \\
    \bottomrule
  \end{tabular}
  \caption[Auswertung Kodierung Feldnotizen]{
    Auswertung der Kodierung der Feldnotizen.
    Prozentuale Angaben in Klammern beziehen sich auf den Anteil des Auftreten des Codes zu der Anzahl der Feldnotizen in diesem Schritt beziehungsweise insgesamt.
  }
  \label{tab:auswertung-feldnotizen}
\end{table}


Bei Betrachtung der unterschiedlichen Ergebnisse in den drei Schritten fällt vor allem auf, dass sich die Kodierungen \glqq Mehr schlecht gelaufen\grqq{}/\glqq Mehr gut gelaufen\grqq{} im Verlauf sehr negativ entwickeln.
Im ersten Schritt wurden 100 Prozent mit \glqq Mehr gut gelaufen\grqq{} kodiert, das kehrt sich im letzten Schritt um.
Dort sind 75 Prozent mit \glqq Mehr schlecht gelaufen\grqq{} kodiert.
Dabei lässt sich insbesondere folgendes beobachten: Mit abnehmendem Erfolg der Aktivitäten ist der \gls{arh} immer weniger involviert an den Aktivitäten.
Der erste Schritt befasst sich noch komplett mit dem \gls{arh} und verläuft gut.
Ab dem zweiten Schritt wird die Wertung schlechter und hier ist der \gls{arh} schon weniger involviert.
Obwohl der \gls{arh} die Ergebnisse vorgibt, kann der er nur entfernter mit der Betrachtung der wissenschaftlichen Artikel assoziiert werden.
In der dritten Phase ist die Bewertung dann am schlechtesten, doch die Kritik in der dritten Phase kann fast nur auf die Verfahren bezogen werden und nur in sehr geringem Maße auf den \gls{arh}.

Daraus kann geschlossen werden, dass der \gls{arh} einen Migrationsprozess dort, wo es die Grenzen seiner Funktionsweise es erlauben, gut begleitet.
Es gibt immer Optimierungspotenzial.
Aus der Perspektive dieser Arbeit waren die Verfahren sowie deren etwaige Qualität am kritischsten. 
Daher scheint es am wichtigsten, die Sammlung von Verfahren zu optimieren.