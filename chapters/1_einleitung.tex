\chapter{Einleitung}
\label{chap:einleitung}

In einer Zeit, in der viele Softwarelösungen auf Cloud Computing basieren, ist auch die Mi\-cro\-services-Architektur immer gebräuchlicher geworden.
Viele Produkte wurden innerhalb des letzten Jahrzehnts schon in diese Architektur übergeführt.
Das Hauptziel ist, von Vorteilen wie besserer Skalierbarkeit, Wartbarkeit und Flexibilität zu profitieren~\cite{Fritzsch_2019,taibi2017processmotivations}.
Abhängig von dem Ausmaß und der Komplexität monolithischer Systeme kann die Migration in eine \gls{msa} jedoch sehr zeitaufwendig, kostenintensiv und kompliziert sein.
Es liegt aktuell eine Vielzahl akademischer Forschungsarbeiten zum Thema Microservices vor, deren Erkenntnisse jedoch nur selten in der Industrie eingesetzt werden \cite{fritzsch2022architecturecentric}.

Um diese Migration zu vereinfachen und die wissenschaftliche Expertise der Industrie leichter zugänglich zu machen, hat die Abteilung \gls{ese} des \gls{iste} das \gls{mmf} entwickelt.
Mit der Entwicklung des \gls{mmf}, das Erkenntnisse über die Migration von monolithischen Anwendungen hin zu Microservices in die Industrie bringen soll, hat das \gls{ese} erste Schritte unternommen, um die erwähnte Lücke zwischen Industrie und Forschung zu schließen.

Bei der Anwendung des auf dem Framework basierenden Tools in der Industrie besteht jedoch noch Verbesserungsbedarf.
Praktische Anwendungen sind für die Weiterentwicklung und Verbesserung des Frameworks und des Tools wichtig.
In der Arbeit von \Citet{master-marvin-knodel} wurde eine erste industrielle Fallstudie zu dem Framework und dem Tool mit der (Teil-) Migration eines Monolithen durchgeführt.
Diese Bachelorarbeit bildet die Fortsetzung der Fallstudien zum \gls{mmf}.
Das Framework wird erstmalig für das Refactoring der bereits vorhandenen \gls{msa} des realen Produkts \jf verwendet.
Die Ergebnisse dieses Refactorings werden zur Bewertung der Effektivität des Frameworks und des Tools verwendet, wodurch diese in Zukunft weiter verbessert werden sollen.

Bei der Analyse von \jf mithilfe des Frameworks werden primär drei Designaspekte von Microservices betrachtet:
\begin{enumerate}
	\item[{[1]}] Die optimale Service-Granularität. Bei dieser muss zwischen Overhead durch zu viele Einheiten und Overhead durch zu große Einheiten abgewogen werden.
	\item[{[2]}] Paradigma der Ansteuerung der einzelnen Einheiten, Kommunikationsmodell zwischen den Einheiten.
	\item[{[3]}] Reduktion des IO-Flaschenhalses, Art des Payload-Transfers. % TODO ausführlicher beschreiben
	Der IO-Flaschenhals entsteht dadurch, dass die einzelnen Einheiten, bevor sie ihre Arbeit verrichten können, immer die Eingangsdokumente herunterladen und anschließend die Ergebnisse wieder hochladen müssen.
\end{enumerate}

Daraus folgt die Forschungsfrage, die in dieser Thesis untersucht wird:

\label[forschungsfrage]{forschungsfrage:1} \textbf{FF:} Wie kann eine bereits bestehende \acrlong{msa} mit Hilfe des \acrfull{mmf} hinsichtlich konkreter Qualitätsaspekte weiter optimiert werden?
\begin{enumerate}
	\item[1.1] Welche Refactoring-Verfahren eignen sich zur Bestimmung der optimalen Service-Gra\-nu\-la\-ri\-tät einer bestehenden \acrlong{msa}?
	\item[1.2] Welche Ansätze, Patterns oder Best Practices eignen sich zur Optimierung des Kom\-mu\-ni\-ka\-tions\-mo\-dells und der Verringerung des IO-Flaschenhalses zwischen den einzelnen Services?
\end{enumerate}


%Um die definierten Forschungsfragen zu beantworten, werden zwei Ziele für diese Arbeit definiert:
%Im ersten Teil wird das \gls{mmf} benutzt, um die Architektur von \jf zu überarbeiten.
%Dabei wird innerhalb von drei Phasen ein Plan für eine neue Architektur des Produkts entworfen.
%Diese Phasen kategorisieren und verweisen auf eine Sammlung wissenschaftlich untersuchter Methoden zur Migration zu Microservices sowie passender Patterns und Best Practices.
%
%Im zweiten Teil wird diese neue Architektur praktisch umgesetzt.
%Da ein komplettes Refactoring des Systems nicht dem Rahmen dieser Arbeit angemessen wäre, wird lediglich ein Prototyp mit der neuen Architektur umgesetzt.
%Bestandteil des zweiten Teils ist deswegen ebenfalls, einen geeigneten Umfang für diesen Prototyp zu finden, sowie Messmethoden festzulegen, die einen Vergleich des Produkts vor und nach dem Refactoring ermöglichen.

Um die Forschungsfragen zu beantworten, ist diese Thesis in die folgenden Kapitel aufgeteilt:

\textbf{\cref{chap:theoretischer-hintergrund}} führt in den theoretischen Hintergrund ein, der für diese Arbeit relevant ist.
Dabei werden auch das \gls{mmf} und \jf kurz beschrieben und ähnliche Arbeiten vorgestellt.
\textbf{\cref{chap:methodik}} beschreibt die Methodik, mit der diese Thesis durchgeführt wird und die Forschungsfrage beantwortet wird.
Unter Verwendung des \gls{mmf} wird in \textbf{\cref{chap:anwendung}} am Produkt \jf die Planung eines Refactorings durchgeführt.
Dabei werden die drei Phasen des Frameworks ausgeführt, welche einen Großteil der Arbeit dieser Thesis ausmachen.
Eine Bewertung der Ergebnisse dieser Anwendung wird in \textbf{\cref{chap:auswertung}} beschrieben.
Abschließend werden in \textbf{\cref{chap:gueltigkeit}} Einschränkungen und die Gültigkeit dieser Arbeit diskutiert und in \textbf{\cref{chap:fazit}} ein Fazit gezogen sowie ein Ausblick gegeben.
