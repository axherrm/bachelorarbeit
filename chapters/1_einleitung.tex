\chapter{Einleitung}
\label{chap:einleitung}

In einer Zeit, in der viele Softwarelösungen auf Cloud Computing basieren, ist auch die Microservices-Architektur immer gebräuchlicher geworden.
Viele Produkte wurden innerhalb des letzten Jahrzehnts schon in diese Architektur übergeführt, hauptsächlich mit dem Ziel, von Vorteilen wie besserer Skalierbarkeit, Wartbarkeit und Flexibilität zu profitieren~\cite{Fritzsch_2019,taibi2017processmotivations}.
Abhängig von dem Ausmaß und der Komplexität monolithischer Systeme, kann die Migration in eine Microservices-Architektur jedoch sehr zeitaufwendig, kostenintensiv und kompliziert sein.
Es liegen aktuell eine Vielzahl akademischer Forschungsarbeiten zum Thema Microservices vor, die jedoch nur selten in der Industrie benutzt werden~\cite{fritzsch2022architecturecentric}.

Um diese Migration zu vereinfachen und die wissenschaftliche Expertise der Industrie leichter zugänglich zu machen, hat das ESE das \gls{mmf} entwickelt.
Mit der Entwicklung des \gls{mmf}, das Erkenntnisse über die Migration von monolithischen Anwendungen hin zu Microservices in die Industrie bringen soll, hat das ESE erste Schritte unternommen, um die erwähnte Lücke zwischen Industrie und Forschung zu schließen.

Offen ist bisher jedoch die Anwendung des auf dem Framework basierenden Werkzeugs in der Industrie, durch die das Framework und Werkzeug weiterentwickelt und verbessert werden kann.
Im Rahmen dieser Bachelorarbeit wird das \acrshort{mmf} daher erstmalig für das Refactoring eines realen Produkts, \emph{jadice flow}, zur Optimierung der bestehenden Microservices-basierten Architektur evaluiert.
Dazu wird das Framework zum Refactoring des Produktes verwendet.
Die Ergebnisse dieses Refactorings werden zur Bewertung der Effektivität des Frameworks und Werkzeuges verwendet, wodurch diese in Zukunft weiter verbessert werden sollen.

Bei der Analyse des Produkts mithilfe des Frameworks werden vor allem drei Designaspekte von Microservices betrachtet:
[1] Die optimale Service Granularität, bei der zwischen Overhead durch zu viele Einheiten und Overhead durch zu große Einheiten abgewogen werden muss.
[2] Paradigma der Ansteuerung der einzelnen Einheiten, Kommunikationsmodell zwischen den Einheiten.
[3] Reduktion des IO-Flaschenhalses, Art des Payload-Transfers.
Der IO-Flaschenhals entsteht dadurch, dass die einzelnen Einheiten, bevor sie ihre Arbeit verrichten können, immer die Eingangsdokumente herunterladen und anschließend die Ergebnisse wieder hochladen müssen.

Daraus folgen die Forschungsfragen, die in dieser Thesis untersucht werden:

\begin{itemize}
	\item \label[forschungsfrage]{forschungsfrage:1} \emph{RQ:} Wie kann eine bereits bestehende Microservices-Architektur mit Hilfe des \acrfull{mmf} hinsichtlich konkreter Qualitätsaspekte weiter optimiert werden?
	\begin{enumerate}[i.]
		\item Welche Refactoring-Verfahren eignen sich zur Bestimmung der optimalen Service Granularität einer bestehenden Microservices-Architektur?
		\item Welche Ansätze, Patterns oder Best Practices eignen sich zur Optimierung des Kommunikationsmodells und der Verringerung des IO-Flaschenhalses zwischen den einzelnen Services?
	\end{enumerate}
\end{itemize}

Um die definierten Forschungsfragen zu beantworten, werden zwei Ziele für diese Arbeit definiert:
Im ersten Teil wird das \gls{mmf} benutzt, um die Architektur von \emph{jadice flow} zu überarbeiten.
Dabei wird innerhalb von drei Phasen ein Plan für eine neue Architektur des Produkts entworfen.
Diese Phasen kategorisieren und verweisen auf eine Sammlung wissenschaftlich untersuchter Methoden zur Migration zu Microservices sowie passender Pattern und Best Practices.

Im zweiten Teil wird diese neue Architektur praktisch umgesetzt.
Da ein komplettes Refactoring des Systems nicht dem Rahmen dieser Arbeit angemessen wäre, wird lediglich ein Prototyp mit der neuen Architektur umgesetzt.
Bestandteil des zweiten Teils ist deswegen ebenfalls, einen geeigneten Umfang für diesen Prototyp zu finden, sowie Messmethoden festzulegen, die einen Vergleich des Produkts vor und nach dem Refactoring ermöglichen.

Um die gesetzten Ziele zu erreichen und damit die Forschungsfragen zu beantworten, ist diese Thesis in die folgenden Kapitel aufgeteilt:
Anfangs wird in \textbf{\cref{chap:theoretischer-hintergrund}} in den theoretischen Hintergrund eingeführt, der für diese Arbeit relevant ist. 
Es werden auch das \gls{mmf} und \emph{jadice flow} kurz beschrieben. 
Ebenfalls werden ähnliche Arbeiten vorgestellt.
In \textbf{\cref{chap:methodik}} wird die Methodik beschrieben, mit der diese Thesis durchgeführt werden soll und die Forschungsfragen beantwortet werden sollen.
Mit Verwendung des \gls{mmf} wird in \textbf{\cref{chap:anwendung}} am Produkt \emph{jadice flow} die Planung eines Refactorings durchgeführt. 
Es werden die drei Phasen des Frameworks ausgeführt, wodurch schlussendlich ein neuer Architekturplan entsteht, der in \textbf{\cref{chap:implementierung}} prototypisch implementiert wird.
Der entstandene Prototyp dient dann in \textbf{\cref{chap:auswertung}} zum quantitativen Vergleich zwischen dem Produkt vor und nach Refactoring.
Abschließend werden in \textbf{\cref{chap:gueltigkeit}} Einschränkungen und Gültigkeit dieser Arbeit diskutiert und in \textbf{\cref{chap:fazit}} ein Fazit gezogen und der Ausblick diskutiert.  