\chapter{Leitfaden Experteninterviews}
\label{chap:expert-interviews-leitfaden}

Dieser Leitfaden beschreibt den Plan, nach dem in den Experteninterviews vorgegangen wird.

\section{Übersicht}

Eine Übersicht über den Rahmen der Experteninterviews ist in \cref{tab:expert-interviews-übersicht} gegeben sowie ein Zeitplan in \cref{tab:expert-interviews-zeitplan}.

\begin{table}[!ht]
  \centering
  \begin{tabular}{|l | p{9cm}|}
    \hline
    Ziel & Bewertung der Nützlichkeit des MMF/ARH bei Migration zu Microservices im Spezialfall jadice flow \\ \hline
    Art des Interviews & Semi-Strukturiert, Einzelinterviews \\ \hline
    Interviewte & Product Owner, Softwarearchitekt, Softwareentwickler \\ \hline
    Interviewer & Axel Herrmann \\ \hline
    Dauer &ca. 45 min \\ \hline
    Datum & 26.02.2024 \& 29.02.2024 \\ \hline
    Ort & Büro levigo/ Online-Meeting TODO \\ \hline
  \end{tabular}
  \caption[Übersicht Experteninterviews]{
    Übersicht über die Experteninterviews
  }
  \label{tab:expert-interviews-übersicht}
\end{table}

\begin{table}[!ht]
	\centering
	\begin{tabular}{m{4.1cm} m{8cm} c}
		\toprule
		\textbf{Schritt} & \textbf{Ziel} & \textbf{Dauer} \\ \midrule
		Einleitung & Begrüßung, Zustimmung zur Verarbeitung und Auf\-zeich\-nung, Erfassung grundlegender Daten & 5 min \\
		Gewünschte Qualitäten ei\-nes \gls{mmf} allgemein & Einschätzung ohne Bezug auf MMF/ARH & 10 min \\
		Verständnisfragen zu MMF/\-ARH & Verständnisprobleme eliminieren, damit spätere Fra\-gen auf vergleichbarem Wissensstand durchgeführt werden können & 10 min \\
		Evaluation MMF/ARH & Hauptteil. Bewertung der in dieser Arbeit durch\-ge\-führ\-ten Anwendung des ARH auf \jf & 20 min \\
		\bottomrule
	\end{tabular}
	\caption[Zeitplan Experteninterviews]{
		Zeitplan der Experteninterviews
	}
	\label{tab:expert-interviews-zeitplan}
\end{table}


\section{Einleitung}

Zu Beginn oder je nach Betrachtung vor Beginn der Experteninterviews werden einige Formalitäten geklärt.
Dabei wird nach einem klaren, strukturiertem Plan vorgegangen:

\begin{enumerate}
	\item Zustimmung zur Aufzeichnung des Interviews und zur Verarbeitung der Daten
	\item Übersicht über die Präambel
	\item Start der Aufzeichnung
	\item Dem Interviewten seine Rolle nennen (da einzelne Interviewte teilweise mehrere Rollen innehaben)
	\item Professionelle Erfahrung in der Informatik in Jahren
	\item Erfahrung mit Microservices in Jahren
	\item Arbeit an dem Produkt (\emph{jadice server} zählt als Vorgänger ebenfalls zu \jf) in Jahren
\end{enumerate}

\section{Evaluation}

Nach der Einleitung kann der Hauptteil der Experteninterviews beginnen.
Im Gegensatz zu den Aussagen und Fragen in der Einleitung ist bei den folgenden Fragen der Wortlaut wichtig, da die Fragen teilweise offen sind.
Deswegen werden die Fragen so wie hier beschrieben gestellt.

\subsection{Gewünschte Qualitäten eines MMF allgemein}

Versetzen Sie sich zurück in die Zeit, in der die Planung der Migration von \emph{jadice server} zu einer \acrlong{msa} bevorstand, aus der dann \jf entstanden ist.
Sie suchen nach einem Werkzeug, einer Software in einer beliebigen Form, das Sie bei der Migration zu Microservices unterstützen soll.

\begin{enumerate}
	\item Welche Aufgaben soll ein solches Werkzeug für Sie übernehmen bzw. bei welchen soll es Sie unterstützen?
	\item Falls noch nicht in der letzten Frage erläutert: Welche Qualitäten soll das Werkzeug mich sich bringen?
\end{enumerate}

\subsection{Verständnisfragen zu MMF/ARH}

\begin{enumerate}
	\item Hatten Sie Probleme, Teile des Informationsmaterials zu verstehen, soll ich bestimmte Details genauer erläutern?
	\item Falls Probleme bestehen: Was sollte am Informationsmaterial verbessert werden?
	\item Falls nicht durch Besprechung der Probleme bereits geschehen: Kurze Zusammenfassung über die zwei enthaltenen Migrationsverfahren.
\end{enumerate}

\subsection{Evaluation MMF/ARH}

\begin{enumerate}
	\item Wie schätzen Sie die Nützlichkeit der vorgestellten Methoden im Bezug auf das Refactoring von \jf mit diesen mit Ziel der Verbesserung der Granularität, des Kommunikationsmodells und des Netzwerk-Overheads ein?
	\item Wie passend schätzen Sie die folgenden Best Practices und Patterns für \jf ein?
	\item Würden Sie nach den Ihnen verfügbaren Informationen die Verwendung des MMF/ARH für \jf zum jetzigen Zeitpunkt als sinnvoll und potentiell gewinnbringend betrachten?
	\item Hätten Sie nach den Ihnen verfügbaren Informationen die Verwendung des MMF/ARH für die Migration von \emph{jadice server} zu Microservices in der Vergangenheit  als sinnvoll und potentiell gewinnbringend betrachten?
\end{enumerate}





\chapter{Präambel Experteninterviews}
\label{chap:expert-interviews-preamble}

In diesem Dokument werden die allgemeinen Richtlinien und ethischen Erwägungen für die Experteninterviews dieser Fallstudie beschrieben.

\section{Ziel der Studie}

Wir erforschen das toolunterstützte Refactoring von Microservices mit dem Tool \emph{\acrlong{arh}} nach dem \emph{\acrlong{mmf}}.
Ziel dieser Fallstudie ist es, diese am Produkt \jf der Firma \emph{levigo solutions} zu testen und diesen Test zu evaluieren.
Damit soll auf der einen Seite eine potentielle Überarbeitung von \jf angestoßen werden und auf der anderen Seite Erkenntnisse über die Effektivität des Frameworks und Tools gewonnen werden, die in Zukunft bei der Weiterentwicklung dieser helfen sollen.

Das Framework hat allgemein das Ziel, die Lücke zwischen Industrie und Akademie zu verringern, indem wissenschaftliche Methoden für die Migration zu Microservices-Architekturen in einem webbasierten Tool kategorisiert und so für die Industrie zugänglich gemacht werden.

\section{Ziel der Experteninterviews}

In diesen Interviews soll Evaluation der Anwendung des Frameworks und Tools auf \jf in dieser Arbeit stattfinden.
Dabei soll sowohl eine allgemeine Bewertung des Konzepts des Frameworks durchgeführt werden als auch die spezielle Anwendung des Tools mit \jf, die in dieser Arbeit stattgefunden hat.

\section{Vertraulichkeit und Anonymität}
Alles, was in den Interviews gesagt wird, wird streng vertraulich behandelt und ohne die Erlaubnis der Teilnehmer an niemanden außerhalb unseres Forschungsteams weitergegeben.
Die Ergebnisse werden vor der Veröffentlichung aggregiert und anonymisiert.
Einzelne Aussagen können in keinster Weise auf einen Mitarbeiter zurückgeführt werden.
Auch nach dem Interview haben die Teilnehmer die Möglichkeit, Aussagen vor der Auswertung auszuschließen oder zu schwärzen.
Die vollständigen Interviewtranskripte werden NICHT veröffentlicht.

\section{Interviewprozess und Analyse}
Die Befragten werden über das allgemeine Thema der Studie informiert, so dass sie damit vertraut sind und sich vorab informieren können, falls sie dies wünschen.
Die Dauer des Interviews wird auf 45 Minuten geschätzt.
Ich bitte Sie um Ihre Erlaubnis, das Interview für die Transkription aufzuzeichnen.
Die schriftliche Abschrift wird dem Teilnehmer zur endgültigen Genehmigung zurückgeschickt.
Dies ist die letzte Möglichkeit, Aussagen aus dem Interview zu entfernen oder zu kürzen.
Danach wird die Tonaufnahme vernichtet, und das möglicherweise redigierte und genehmigte Transkript wird die einzige Grundlage für die qualitative Inhaltsanalyse sein.

\chapter{Material Experteninterviews}







\chapter{Feldnotizen}
\label{chap:field-notes}

%\fieldnote
%{F1\_P2\_11\_20\_2023\_Activity}
%{test}
%{Datum, Uhrzeit}
%{Ort}
%{Personen}
%{Phase Schritt MMF}
%{
%	aaaaaaa aaaaaa aaaaaa aaaaaaa aaaaaa aaaaaa aaaaaaa aaaaaa aaaaaa aaaaaaa aaaaaa aaaaaa aaaaaaa aaaaaa aaaaaa aaaaaaa aaaaaa aaaaaa aaaaaaa aaaaaa aaaaaa aaaaaaa aaaaaa aaaaaa aaaaaaa aaaaaa aaaaaa aaaaaaa aaaaaa aaaaaa aaaaaaa aaaaaa aaaaaa aaaaaaa aaaaaa aaaaaa aaaaaaa aaaaaa aaaaaa aaaaaaa aaaaaa aaaaaa aaaaaaa aaaaaa aaaaaa aaaaaaa aaaaaa aaaaaa aaaaaaa aaaaaa aaaaaa aaaaaaa aaaaaa aaaaaa aaaaaaa aaaaaa aaaaaa aaaaaaa aaaaaa aaaaaa aaaaaaa aaaaaa aaaaaa aaaaaaa aaaaaa aaaaaa aaaaaaa aaaaaa aaaaaa aaaaaaa aaaaaa aaaaaa aaaaaaa aaaaaa aaaaaa aaaaaaa aaaaaa aaaaaa aaaaaaa aaaaaa aaaaaa aaaaaaa aaaaaa aaaaaa aaaaaaa aaaaaa aaaaaa aaaaaaa aaaaaa aaaaaa aaaaaaa aaaaaa aaaaaa aaaaaaa aaaaaa aaaaaa aaaaaaa aaaaaa aaaaaa aaaaaaa aaaaaa aaaaaa aaaaaaa aaaaaa aaaaaa aaaaaaa aaaaaa aaaaaa aaaaaaa aaaaaa aaaaaa aaaaaaa aaaaaa aaaaaa aaaaaaa aaaaaa aaaaaa aaaaaaa aaaaaa aaaaaa aaaaaaa aaaaaa aaaaaa aaaaaaa aaaaaa aaaaaa aaaaaaa aaaaaa aaaaaa aaaaaaa aaaaaa aaaaaa aaaaaaa aaaaaa aaaaaa aaaaaaa aaaaaa aaaaaa aaaaaaa aaaaaa aaaaaa aaaaaaa aaaaaa aaaaaa aaaaaaa aaaaaa aaaaaa aaaaaaa aaaaaa aaaaaa aaaaaaa aaaaaa aaaaaa aaaaaaa aaaaaa aaaaaa aaaaaaa aaaaaa aaaaaa aaaaaaa aaaaaa aaaaaa aaaaaaa aaaaaa aaaaaa aaaaaaa aaaaaa aaaaaa aaaaaaa aaaaaa aaaaaa aaaaaaa aaaaaa aaaaaa aaaaaaa aaaaaa aaaaaa aaaaaaa aaaaaa aaaaaa aaaaaaa aaaaaa aaaaaa aaaaaaa aaaaaa aaaaaa aaaaaaa aaaaaa aaaaaa aaaaaaa aaaaaa aaaaaa aaaaaaa aaaaaa aaaaaa aaaaaaa aaaaaa aaaaaa aaaaaaa
%}
%{
%	Kommentare
%}
%{
%	Was lief gut
%}
%{
%	Probleme
%}
%{
%	Empfindungen
%}

\input{field_notes/F01_P2_11_20_2023_FilterAuswahl}
\input{field_notes/F02_P2_11_22_2023_FilterAuswahl}
\fieldnote
{F03\_P2\_11\_27\_2023\_FilterPrio}
{3}
{27.11.2023, 11:00-11:10}
{Holzgerlingen, Büro levigo}
{Axel Herrmann, Product Owner}
{2, Einstellen der Filter des \gls{arh}}
{
	Die Einstellung der Filter für die Suche nach Refactoring-Methoden mit dem \gls{arh} wurde fortgesetzt.
	Dabei sollten Filter in zwei Prioritäten entstehen, sodass später eine Suche mit allen Filtern und eine mit nur den wichtigsten Filtern durchgeführt werden kann.
	Größtenteils wurden die bereits ausgewählten Filter in die zwei Prioritäten unterteilt, aber in zwei Fällen auch neue, vorher nicht ausgewählte Filter zur geringeren Priorität hinzugefügt.
	Die neuen Filter sind:
	\begin{tabular}{|p{3cm}|p{3.5cm}|p{3cm}|}
		\hline
		\textbf{Kategorie} & \textbf{Filter Prio 1} & \textbf{Filter Prio 2} \\ \hline
		Quality Preferences, System Properties & Autonomy, Granularity, Technology Heterogenity & Complexity, Isolation \\ \hline
		Input Preferences, Domain Artifacts & Human Expertise & Version Control System \\ \hline
		Input Preferences, Runtime Artifacts & & Log Traces \\ \hline
		Input Preferences, Executables & API/Interface, Source Code (No specifica\-tion) & Source Code (Java) \\ \hline
		Process Preferences, Process Strategy & Refactor & \\ \hline
		Output Preferences, Representation & List of services, Split\-ting recommendations & Guideline/ Workflow \\ \hline
		Usability Preferences & - & \\ \hline
	\end{tabular}
}
{
	Die Filter sind vorerst finalisiert. Nach einer Durchsicht der Ergebnisse der Refactoring-Methoden könnten weitere Anpassung der Filter erfolgen.
}
{
	Es konnte sich bei jedem Filter schnell geeinigt werden und die Aufgabe war schnell erledigt.
}
{}
{
	Sicherheit bei der Aufgabe, da das Vorgehen nun klarer ist und alles mit dem Betreuer besprochen wurde.
	Außerdem mehr Sicherheit mit Dokumentation in Feldnotizen.
}

\fieldnote
{F04\_P2\_12\_13\_2023\_Ergebnisbetrachtung}
{4}
{13.12.2023, 13:00-15:00}
{Zuhause}
{Axel Herrmann}
{2, Betrachtung des ersten Suchergebnisses}
{
	Die Arbeit von \Citet{arh-result-no-filter-1} wurde durchgelesen und zusammengefasst.
}
{
	Nach der Betrachtung der anderen Ergebnisse kann die Arbeit erneut und genauer betrachtet werden.
	Eine Bewertung hinsichtlich des Nutzens für \emph{jadice flow} erfolgt später, wenn mit anderen Ergebnissen verglichen werden kann.
}
{
	Der Artikel ist kurz und prägnant.
}
{}
{
	Erster Eindruck: Die Methode wirkt sehr oberflächlich und wenig spezifisch und scheint sich daher weniger zu eignen.
}

\input{field_notes/F05_P2_12_20_2023_Ergebnisbetrachtung_2}
\fieldnote
{F06\_P2\_12\_20\_2023\_Ergebnisbetrachtung\_3}
{6}
{20.12.2023, 17:30-18:30}
{Zuhause}
{Axel Herrmann}
{2, Betrachtung des dritten Suchergebnisses}
{
	Die Arbeit von \Citet{arh-result-no-filter-2} wurde durchgelesen und zusammengefasst.
}
{
	Nach der Betrachtung der anderen Ergebnisse kann die Arbeit erneut und genauer betrachtet werden.
	Eine Bewertung hinsichtlich des Nutzens für \emph{jadice flow} erfolgt später, wenn mit anderen Ergebnissen verglichen werden kann.
}
{
%  Bereits das Abstract des Artikels bietet genug Informationen für eine abstrakte Zusammenfassung und ein grundlegendes Verständnis der Methode.
  Erster Eindruck: Die Methode ist bisher am ausführlichsten.
  Im Vergleich zur 2. berücksichtigt sie mehr Metriken und die Eingabe einer vorhandenen \gls{arh} ist auch passend.
  Die Visualisierung der Architektur mit Anzeige der Metriken wäre sehr angenehm.
}
{}
{
  Froh, einen weiteren relativ gut passenden Ansatz gefunden zu haben.
}

\fieldnote
{F07\_P2\_12\_21\_2023\_Ergebnisbetrachtung\_4}
{7}
{21.12.2023, 08:30-09:30}
{Zuhause, Online-Meeting}
{Axel Herrmann, Product Owner}
{2, Betrachtung der ersten drei Suchergebnisse}
{
	Die Arbeiten von \Citet{arh-result-no-filter-1}, \Citet{arh-result-no-filter-3} und \Citet{arh-result-no-filter-2} wurden dem Product Owner zusammengefasst und gemeinsam die Möglichkeit der Anwendung auf \emph{jadice flow} diskutiert.
  Dabei wurde eine Ordnung der Ergebnisse hinsichtlich ihrer Nützlichkeit für diese Fallstudie vorgenommen, wobei \Citet{arh-result-no-filter-2} mit Platz 1, \Citet{arh-result-no-filter-3} mit Platz 2 und \Citet{arh-result-no-filter-3} mit Platz 3 abschneidet.
}
{
	Die aufgestellte Platzierung ist in Anbetracht dessen, dass erst 3 der 6 Ergebnisse betrachtet wurden, noch nicht vollständig.
}
{
  Alle der Methoden beinhalten eine nützliche Übersichts-Grafik, wodurch das Beschreiben der Methoden gut funktioniert.
  Bei der Platzierung der Methoden kann einfach und unkopliziert Konsens erreicht werden.
}
{}
{
	Der erste Eindruck des Product Owners stimmt mit der des Autors überein.
}

\fieldnote
{F08\_P2\_01\_10\_2024\_Ergebnisbetrachtung\_5}
{8}
{10.01.2024, 10:00-10:30}
{Büro levigo}
{Axel Herrmann}
{2, Betrachtung des vierten Suchergebnisses}
{
  Die Arbeit von \Citet{arh-result-no-filter-4} wurde durchgelesen und zusammengefasst.
}
{
  Die Betrachtung dieser Arbeit ist im Gegensatz zu den anderen Ergebnissen vermutlich final, da die Verwendung dieser Methode schnell ausgeschlossen werden konnte.
}
{
  Bereits das Abstract des Artikels bietet genug Informationen für eine abstrakte Zusammenfassung und der Ausschluss der Verwendung dieser kann schnell erfolgen.
}
{}
{
  Erster Eindruck: Die Ziele der Methode passen nicht zu unseren und sie ist technisch vermutlich auch nicht anwendbar.
}

\fieldnote
{F09\_P2\_01\_10\_2024\_Ergebnisbetrachtung\_6}
{9}
{10.01.2024, 11:00-11:30}
{Büro levigo}
{Axel Herrmann}
{2, Betrachtung des fünften Suchergebnisses}
{
  Die Arbeit von \Citet{arh-result-no-filter-5} wurde durchgelesen und zusammengefasst.
}
{
  Nach der Betrachtung der anderen Ergebnisse kann die Arbeit erneut und genauer betrachtet werden.
}
{
  Erster Eindruck: Das Werkzeug kann nicht verwendet werden, die Methode kommt infrage.
}
{
  Erster Eindruck: Verwendung bringt vermutlich erhöhten Aufwand mit sich, da das fehlende Werkzeug manuell nachgeahmt werden muss.
}
{
}

\fieldnote
{F10\_P2\_01\_11\_2024\_Ergebnisbetrachtung\_7}
{10}
{11.01.2024, 11:00-11:30}
{Büro levigo}
{Axel Herrmann}
{2, Betrachtung des sechsten Suchergebnisses}
{
  Die Arbeit von \Citet{arh-result-important-filter-4} wurde durchgelesen und zusammengefasst.
}
{
  Nach der Absprache mit dem \gls{po} kann die Arbeit erneut und genauer betrachtet werden.
}
{
}
{
  Erster Eindruck: Der Kontext von \gls{iiot} passt nicht zu \emph{jadice flow}, ob die Methode trotzdem sinnvoll sein könnte, bleibt offen.
}
{
}

\fieldnote
{F11\_P2\_01\_11\_2024\_Ergebnisbetrachtung\_8}
{11}
{11.01.2024, 14:00-14:30}
{Büro levigo}
{Axel Herrmann, Product Owner}
{2, Vergleich der Suchergebnisse}
{
  Die Arbeiten von \Citet{arh-result-no-filter-4}, \Citet{arh-result-no-filter-5} und \Citet{arh-result-important-filter-4} wurden dem Product Owner zusammengefasst und gemeinsam die Möglichkeit der Anwendung auf \emph{jadice flow} diskutiert.
  Dabei wurde eine Ordnung aller Ergebnisse hinsichtlich ihrer Nützlichkeit für diese Fallstudie vorgenommen, wobei immer noch \Citet{arh-result-no-filter-2} mit Platz 1 und \Citet{arh-result-no-filter-3} mit Platz 2 abschneiden.
}
{
}
{
}
{}
{
  Der erste Eindruck des Product Owners stimmt mit der des Autors überein.
}

\fieldnote
{F12\_P2\_01\_17\_2024\_Ergebnisbetrachtung\_9}
{12}
{17.01.2024, 11:00-11:30}
{Zuhause}
{Axel Herrmann}
{2, Betrachtung des siebten Suchergebnisses}
{
  Die Arbeit von \Citet{arh-result-important-filter-7} wurde durchgelesen und zusammengefasst.
}
{
}
{
}
{}
{
  Erster Eindruck: Einsatz wäre vermutlich möglich, Funktionen als atomare Einheiten scheinen allerdings sehr klein-granular und möglicherweise zu aufwendig.
}

\fieldnote
{F13\_P2\_01\_18\_2024\_Ergebnisbetrachtung\_10}
{13}
{18.01.2024, 10:30-11:00}
{Zuhause}
{Axel Herrmann}
{2, Betrachtung des achten Suchergebnisses}
{
  Die Arbeit von \Citet{arh-result-no-qas} wurde durchgelesen und zusammengefasst.
}
{
}
{
  Erster Eindruck: Einsatz wäre prinzipiell möglich, aber
}
{
  Klassen als atomare Einheiten scheinen allerdings sehr klein-granular und möglicherweise zu aufwendig.
}
{
}

\fieldnote
{F14\_P2\_01\_18\_2024\_Ergebnisbetrachtung\_11}
{14}
{18.01.2024, 11:00-11:30}
{Zuhause, Online-Meeting}
{Axel Herrmann, Product Owner}
{2, Vergleich der Suchergebnisse}
{
  Die Arbeiten von \Citet{arh-result-important-filter-7} und \Citet{arh-result-no-qas} wurden dem Product Owner zusammengefasst und gemeinsam die Möglichkeit der Anwendung auf \emph{jadice flow} diskutiert.
  Da nun voraussichtlich alle potentiellen Methoden betrachtet wurden, wurde diskutiert, welche der Verfahren als erstes fokussiert werden sollte.
  Die ersten zwei Plätze belegen dabei \Citet{arh-result-no-filter-2} und \Citet{arh-result-no-filter-3}.
}
{
}
{
  Schnelles Übereinstimmen bei der Bewertung und Sortierung der Verfahren
}
{
  Nur oberfläches Wissen über die Verfahren reicht nicht aus, um final zu entscheiden, welcher Anstatz verwendet wird.
  Einige Verfahren gelten als semi-automatisch, allerdings sind keine der in den Verfahren beschriebenen Tools zu finden, was den zeitlichen Aufwand bei Einsatz der Verfahren schwerer einschätzbar macht.
}
{
  Unsicherheit, welches der Verfahren wirklich im zeitlichen Rahmen realistisch anwendbar ist
}

\fieldnote
{F15\_P3\_01\_30\_2024\_Methodenanwendung}
{15}
{30.01.2024, 11:00-17:40}
{Zuhause}
{Axel Herrmann}
{3, Anwendung der ersten Migrationsmethode}
{
  Das Ergebnis \result{3} wurde in vorherigen Schritten als favorisierte Methode ausgewählt.
  Der Artikel dazu wurde genauer betrachtet und das beschriebene Tool \acrfull{mb} gesucht.
  Es konnte keine Veröffentlichung des Tools gefunden werden, lediglich ein GitHub Projekt der Autoren, das den richtigen Namen trug.
  Das Projekt scheint veraltet zu sein, es dauerte einige Zeit, es überhaupt zum Laufen zu bringen.
  Auch nach erfolgreichem Start von \gls{mb} in einem Docker Container konnte das Webtool nicht verwendet werden, da abgesehen von der Startseite jede Sub-Seite Fehler auslöste und nicht aufgerufen werden konnte.
  Da der manuelle Aufwand der Umsetzung dieser Methode als zu hoch bewertet wurde, wurde dieses Verfahren letztendlich verworfen.
}
{
  Nach diesem favorisiertem Verfahren besteht noch bei einem weiteren Verfahren Hoffnung, dass es verwendet werden könnte.
  Dieses wird als nächstes betrachtet.
}
{
}
{
  Das beschriebene Tool ist nicht öffentlichv verlinkt, das gefundene GitHub Projekt nicht benutzbar und die Methode manuell in dieser Thesis nicht umsetzbar.
}
{
  Frustration über das Fehlschlagen dieses Verfahrens.
}

\fieldnote
{F16\_P3\_02\_07\_2024\_Methodenanwendung\_2}
{16}
{07.02.2024, 10:00-14:00}
{Zuhause}
{Axel Herrmann}
{3, Anwendung der zweiten Migrationsmethode}
{
  Der Algorithmus \emph{cHAC} nach \result{2} wurde im Detail betrachtet und nach Top-Down Methode angefangen umzusetzen.
  Die Implementierung wurde in Java vorgenommen.
  Für gewisse Details des Pseudo-Codes wurden Java-spezifische Konstrukte verwendet statt der im Algorithmus angegebenen.
  Beispielsweise wurde die Synchronisationsvariable @SIC des Algorithmus durch eine CyclicBarrier\footnote{\url{https://docs.oracle.com/javase/8/docs/api/java/util/concurrent/CyclicBarrier.html}} umgesetzt.
}
{
}
{
  Obwohl der Algorithmus relativ komplex ist, wurde er auf einer abstrakten Ebene verstanden und auf dieser auch umgesetzt.
}
{
  Der Algorithmus ist relativ kompliziert und die Beschreibung dazu fällt kurz aus. Viele Teile mussten sehr oft gelesen werden, bevor Verständnis erlangt werden konnte.
}
{
  Unsicher, ob der Algorithmus umgesetzt werden kann.
}

\fieldnote
{F17\_P3\_02\_08\_2024\_Methodenanwendung\_3}
{17}
{08.02.2024, 10:45-17:00}
{Zuhause}
{Axel Herrmann}
{3, Anwendung der zweiten Migrationsmethode}
{
  Der Algorithmus \emph{cHAC} nach \result{2} wurde weiter versucht umzusetzen.
  Die grobe Struktur war duch den vorherigen Tag bereits vorhanden.
  Nun sollten die einzelnen Schritte, Funktionen und Formeln des Artikels umgesetzt werden.
  Dabei konnte kein Erfolg verzeichnet werden; einige Funktionen waren nicht bis ins Detail beschrieben und ließen Freiraum, einige Formeln definierten Eingabevariablen nicht genau und in einer Formel wurde auch ein Schreibfehler vermutet.
  Dadurch dass die Formeln lediglich genannt, nicht aber begründet oder beschrieben wurden, konnten Unklarheiten nicht geklärt und der Algorithmus so nicht umgesetzt werden.
}
{
}
{
}
{
  Kein Erfolg: Arbeit am Algorithmus muss eingestellt werden.
}
{
  Frustration: Obwohl 2 Suchergebnisse vielversprechend klangen, konnte schlussendlich keines davon verwendet werden.
}

\fieldnote
{F18\_P3\_02\_09\_2024\_Besprechung\_Methodenanwendung}
{18}
{09.02.2024, 14.00-14:15}
{Zuhause, Online-Meeting}
{Axel Herrmann, Product Owner}
{3, Besprechung Anwendung Migrationsverfahren}
{
  Die Ergebnisse der erfolgslosen Anwendung der Migrationsmethoden wurden besprochen und gemeinsam beschlossen, die Arbeit des Refactorings deswegen einzustellen.
  Die anderen Suchergebnisse des \gls{arh} wurden bereits in vorherigen Schritten als wesentlich weniger passend bewertet und deswegen und aus Zeitgründen keine weitere Anwendung in Betracht gezogen.
}
{
}
{
}
{
  Kein Erfolg in dieser Phase: Arbeit am Refactoring muss eingestellt werden.
}
{
  Frustration: Obwohl 2 Suchergebnisse vielversprechend klangen, konnte schlussendlich keines davon verwendet werden.
}

