\chapter{Fazit}
\label{chap:fazit}

Das Ziel dieser Arbeit bestand darin, die Verwendung des \acrlongpl{mmf} (\acrshort{mmf}) und des \acrlongpl{arh} (\acrshort{arh}) der Abteilung \acrfull{ese} des \acrfull{iste} mit einer bestehenden \acrfull{msa} in einer industriellen Fallstudie zu evaluieren.
Dazu wurden das Framework und das Tool, welche in vorausgegangen wissenschaftlichen Arbeiten entwickelt wurden, zum Refactoring des Produkts \jf der Firma \emph{levigo solutions} verwendet.
Der Fokus lag dabei auf der Beantwortung der Forschungsfrage:

\textbf{FF:} Wie kann eine bereits bestehende Microservices-Architektur mit Hilfe des \acrfull{mmf} hinsichtlich konkreter Qualitätsaspekte weiter optimiert werden?
\begin{enumerate}[i.]
	\item Welche Refactoring-Verfahren eignen sich zur Bestimmung der optimalen Service Granularität einer bestehenden Microservices-Architektur?
	\item Welche Ansätze, Patterns oder Best Practices eignen sich zur Optimierung des Kom\-mu\-ni\-ka\-tions\-mo\-dells und der Verringerung des IO-Flaschenhalses zwischen den einzelnen Services?
\end{enumerate}
%TODO:  Aufzählung der Kapitel überdenken
Um diese Fragen zu beantworten, wurde nach einer thematischen Übersicht in \cref{chap:theoretischer-hintergrund} und der Planung der Methodik in \cref{chap:methodik} in \cref{chap:anwendung} mit der Anwendung des \gls{mmf} auf \jf begonnen.
In der ersten Phase des Frameworks wurde ein Architekturreview durchgeführt, um die gewünschten \glspl{qa} des Systems szenarienbasiert zu sammeln.
Zusätzlich wurde am Anfang der zweiten Phase eine Konfiguration von weiteren Suchfiltern vorgenommen.
Mit vier verschiedenen Variationen dieser Ausgangsdaten wurden dann vier Suchen nach Migrationsverfahren durchgeführt und die Ergebnisse dessen zu einer Liste aggregiert.
Die ersten acht Ergebnisse dieser Liste wurden anschließend vom Autor detaillierter betrachtet und zusammen mit dem \gls{po} von \jf hinsichtlich ihrer Anwendbarkeit in dieser Thesis auf \jf evaluiert.
Daraufhin wurde versucht, die daraus resultierenden zwei bevorzugten Verfahren im verfügbaren Zeitrahmen umzusetzen.
Aufgrund von Zeitmangel und Problemen mit der Umsetzung der Verfahren ist dieser Teil fehlgeschlagen.
Eine Auswertung der Fallstudie durch Evaluation konkreter Ergebnisse eines Refactorings war daher nicht möglich.

Die Auswertung wurde stattdessen mittels Experteninterviews zur Einschätzung der Anwendbarkeit beider Verfahren durchgeführt.
Zusätzlich wurden während der Durchführung ab der zweiten Phase strukturierte Feldnotizen nach \Citet{seaman2008qualitative} gesammelt.
Zusammen mit den Experteninterviews bilden diese Methoden die gesamte Methodik zur Auswertung dieser Fallstudie.
%In \cref{sec:auswertung-diskussion} wurde die Forschungsfrage anhand der beschriebenen Methodik erörtert.
%Dabei wurde sowohl die erste als auch die zweite Unterfrage der Forschungsfrage positiv beantwortet.
Obwohl die Anwendung der Migrationsverfahren in dieser Arbeit nicht vollendet werden konnte, zeigt die Auswertung der Feldnotizen, dass die Teile des Prozesses, in die der \gls{arh} involviert war, gut verlaufen sind.
%Für das Scheitern der Anwendung der Verfahren wurde dem \gls{arh} nur wenig Schuld zugerechnet.
Des Weiteren wurde in den Experteninterviews untersucht, inwieweit die zwei favorisierten Verfahren in einer zukünftigen, weniger zeitlich und personell beschränkten Umgebung nützlich wären.
Die Verfahren wurden von allen Experten als potenziell gewinnbringend für das Produkt eingeschätzt.
Es wurde angegeben, dass eine Verbesserung der Granularität von \jf mit diesen für möglich gehalten wird.
Da die vom \gls{arh} vorgeschlagene Liste von \bpp in der Thesis nicht zur Anwendung kam, wurde hier ebenfalls eine Bewertung der Experten eingeholt.
Auch dabei ergab sich eine positive Rückmeldung.
Dies bezog sich auch auf die in der Forschungsfrage thematisierten Qualitäten des Kommunikationsmodells und Netzwerk-Overheads.
Abschließend wurde von den Experten die generelle Funktionsweise des Frameworks evaluiert. 
Dabei wurden die einzelnen Funktionen sowie das Vorgehen in Phasen positiv hervorgehoben.
%Sowohl die positive Evaluation der Funktionsweise des Frameworks in drei Phasen als auch die positiv evaluierten Ergebnisse in Phasen 2 und 3 des \gls{arh} zeigen, dass der \gls{arh}
%Die Ergebnisse der Untersuchung zeigen, dass der \gls{arh} ein vielversprechendes Tool für die Migration von Monolithen zu Microservices ist.
%Das Tool bietet vefrschiedene Funktionen, die den Migrationsprozess unterstützen, z. B. die Erfassung von Qualitätsattributen, die Suche nach Migrationsverfahren und die Auswahl von Best Practices und Patterns.
%
%Es wurde gezeigt, dass der \gls{arh} in diesem Einzelfall das Finden von geeignete Migrationsverfahren ermöglicht hat.
%Außerdem wurde die Liste der geeigneten \bpp des \gls{arh} von Experten als wertvoll bewertet.

Zusammenfassend kann festgestellt werden, dass der \gls{arh} ein vielversprechendes Tool für die Migration von Monolithen zu \glspl{msa} darstellt und auch Potential zur Optimierung bestehender \glspl{msa} gesehen wird.

\section{Ausblick}

Es gibt noch einige Herausforderungen, die in Zukunft angegangen werden müssen, um die Effizienz und Effektivität des \gls{arh} zu verbessern.
In der Arbeit wurden Verbesserungsvorschläge für das Framework und das Tool genannt, deren Umsetzung die Qualität erhöhen könnte.
Außerdem ist die Datenbasis des \gls{arh} einer der größten Einflussfaktoren auf seine Qualität.
Daher ist es wichtig, die Auswahl an Migrationsverfahren, \bpp in Zukunft zu erweitern und auf dem neuesten Stand zu halten.
Außerdem kann der Filterungsmechanismus der Suchen in Phasen 2 und 3, neben den konkreten Vorschlägen in dieser Arbeit, allgemein immer weiter verbessert werden.
Es könnte beispielsweise interessant sein, \gls{ki} zu verwenden, um die Verwendung der Filterfunktionen zu erleichtern sowie ihre Verfügbarkeit zu erhöhen und die damit verbundene Komplexität für den Nutzer trotzdem gering zu halten.

Außerdem sollte die Nutzung des \gls{arh} weiter untersucht werden, da die vorhandenen zwei Fallstudien vermutlich nur einen kleinen Teil der realen Anwendungsfälle repräsentieren und die externe Validität dieser somit eingeschränkt ist.
Insbesondere der spezielle Anwendungsfall, bei dem eine \gls{msa} mit dem Tool überarbeitet wird, verursacht einige Risikofaktoren, die die Anwendung schwieriger und möglicherweise nicht in jedem Fall sinnvoll machen.
Es gilt, diese Risikofaktoren weiter zu untersuchen.

Im Hinblick auf \jf wurde in dieser Arbeit gezeigt, dass eine Verbesserung der Ar\-chi\-tek\-tur durch die Anwendung des \gls{arh} sowie konkreter die Anwendung der ausgewählten Migrationsmethoden als gewinnbringend angesehen wird.

In dieser Arbeit wurden wichtige Schritte zur Weiterentwicklung von \jf angestoßen.
Im Rahmen der Recherche und der teilweisen Anwendung des Frameworks und des Tools wurden an vielen Stellen neue Lösungen für vorhandene Probleme erkannt.
Diese sollten in Zukunft weiterverfolgt werden, um die Architektur von \jf zu optimieren.

