\chapter{Ähnliche Forschung}
\label{chap:aehnliche-forschung}

In dieser Thesis wird sehr vordergründig das Microservices Migration Framework des ESE  untersucht.
Dieses ist aufgrund vorheriger Arbeiten wie \Citet{10.1007/978-3-030-06019-0_10} entstanden.

In 2019 kategorisieren \Citet{10.1007/978-3-030-06019-0_10} zehn verschiedene Migrations-Methoden.
Dabei wird ein Mängel an praktisch anwendbaren Methoden hervorgehoben, die gute Werkzeugunterstützung und Metriken zur Verifikation der Ergebnisse bieten.

Als Folge darauf stellen \Citet{fritzsch2022architecturecentric} 2022 dann die Planung ein solchen Frameworks vor. Der Migrationsprozess mit Framework und dessen Arbeitsweise in drei Phasen wird bereits beschrieben. Auch wird die Entwicklung eines Web-basierten Werkzeugs erwähnt, das das Framework benutzt und Entwickler bei der Migration unterstützt.

\Citet{10.1145/3242163.3242164} beschreiben geläufige Gefahren bei der Migration zu Microservices-Architekturen.
Dabei  werden fünf architektonische Fehlermuster und vier Fehlermuster bei der Migration aufgeführt.
Außerdem werden Lösungsmöglichkeiten für jedes Muster beschrieben.

%\section{Refactoring}
%
%Obwohl in dieser Thesis ein Framework zur Migration zu Microservices im Mittelpunkt steht, sieht man schnell, dass die Migration eng verwandt mit dem Refactoring ist.
%Deswegen werden folgend ein paar Quellen vorgestellt, die das Refactoring von Microservices-Architekturen behandeln.
%
%\Citet{https://doi.org/10.1002/spe.2974} beschreiben in einem Artikel zur $\mu$TOSCA toolchain, wie sie mit dem  $\mu$TOSCA Modell Architekturen von Microservices-basierten Systemen mit dem OASIS Standard TOSCA darstellen können.
%Außerdem beschreiben sie die Möglichkeit, das $\mu$TOSCA Modell eines Systems durch das zugehörige Kubernetes Deployment automatisch zu ermitteln.
%Das $\mu$TOSCA Modell einer Applikation soll dann dazu genutzt werden, potentielle architektonische Probleme zu finden.
%Für das Ermitteln des $\mu$TOSCA Modells und das Identifizieren von Architektur Smells stellen sie zwei Werkzeug Prototypen ($\mu$Miner und $\mu$Freshener) vor.