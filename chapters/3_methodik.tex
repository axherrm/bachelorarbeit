\chapter{Methodik}
\label{chap:methodik}

Die Thesis ist als industrielle Fallstudie in Zusammenarbeit mit \emph{levigo solutions} konzipiert und dient zur Evaluation des \gls{mmf} an einer bestehenden Microservices-Architektur.

Um den Nutzen des Frameworks zu analysieren, soll es zur Erstellung eines Konzepts [1] zum Refactoring der gesamten Anwendung auf Basis eines oder mehrerer (identifizierter) Refactoring-Verfahren; [2] für einen Ansatz zur Optimierung des Kommunikationsmodells und der Verringerung des IO-Flaschenhalses; angewendet werden.
Der genaue Aufbau dieses Experiments wird im Folgenden explizit beschrieben.

\section{Aufbau}

Dieses Refactoring soll dann im Rahmen eines minimalen Prototyps umgesetzt und simuliert werden, da ein Refactoring des kompletten Systems im zeitlichen Rahmen dieser Arbeit nicht möglich wäre.
Dabei entstehende Prototypen könnten Vorläufer von \emph{jadice flow 2.0} werden.
Im Nachgang sollen die Ergebnisse dieses Experiments ausgewertet werden. Um quantitativ vergleichbare Ergebnisse zu erhalten, sollen im Vorlauf des Experiments Vergleichskriterien und Messmethoden definiert werden, die es möglich machen, gleichartige Anwendungsfälle von \emph{jadice flow 1.0} und neuen Prototypen zu vergleichen.

Um den Migrationsprozess als Ganzes bewerten zu können, werden die qualitativen Methoden der strukturierten Feldnotizen, des Interviews oder der Beobachtungen verwendet \cite{seaman2008qualitative}. 
Während der Migration und Anwendung des Frameworks bzw. Tools werden systematisch durchgeführte Aktivitäten mit gemachten Erfahrungen und Herausforderungen dokumentiert und am Ende ausgewertet. 
Dazu können die involvierten zwei Systemarchitekten, sowie vier Entwickler zur Rate gezogen werden.

\section{Vergleichskriterien \& Messmethoden}
