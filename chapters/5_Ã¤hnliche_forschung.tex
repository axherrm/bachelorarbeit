\chapter{Ähnliche Forschung}
\label{chap:aehnliche-forschung}

Es liegt bereits einige Forschung zum Architekturprinzip der Microservices vor.
Folgend soll sowohl auf grundlegende Literatur zu Microservices verwiesen werden, als auch verwandte Arbeiten erwähnt werden.
Dabei wird in Literatur zur Migration von monolithischen Systemen zu Microservices-basierten und in Literatur zum Refactoring von Microservices-basierten System unterschieden werden.

\section{Migration}

\Citet{10.1145/3242163.3242164} beschreiben geläufige Gefahren bei der Migration zu Microservices-Architekturen.
Dabei  werden fünf architektonische Fehlermuster und vier Fehlermuster bei der Migration aufgeführt.
Außerdem werden Lösungsmöglichkeiten für jedes Muster beschrieben.

\Citet{10.1007/978-3-030-06019-0_10} kategorisieren zehn verschiedene Refactoring-Methoden.
Dabei wird ein Mängel an praktisch anwendbaren Methoden hervorgehoben, die gute Werkzeugunterstützung und Metriken zur Verifikation der Ergebnisse bieten.

\section{Refactoring}

\Citet{https://doi.org/10.1002/spe.2974} beschreiben in einem Artikel zur $\mu$TOSCA toolchain, wie sie mit dem  $\mu$TOSCA Modell Architekturen von Microservices-basierten Systemen mit dem OASIS Standard TOSCA darstellen können.
Außerdem beschreiben sie die Möglichkeit, das $\mu$TOSCA Modell eines Systems durch das zugehörige Kubernetes Deployment automatisch zu ermitteln.
Das $\mu$TOSCA Modell einer Applikation soll dann dazu genutzt werden, potentielle architektonische Probleme zu finden.
Für das Ermitteln des $\mu$TOSCA Modells und das Identifizieren von Architektur Smells stellen sie zwei Werkzeug Prototypen ($\mu$Miner und $\mu$Freshener) vor.
