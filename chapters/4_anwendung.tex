\chapter{Anwendung des Frameworks auf jadice flow} % TODO reconsider chap name
\label{chap:anwendung}

In diesem Kapitel wird ein Architektur-Refactoring am Produkt \emph{jadice flow} geplant.
Dazu wird das \gls{mmf} benutzt, das in \cref{sec:mmf} beschrieben wird.
Unterteilt ist der Prozess in drei Phasen, deren Durchführung im Folgenden beschrieben wird.

\section{Phase 1 - Architekturreview}

In dieser Phase wird ein Architekturreview nach \Citet{SVAHNBERG20071893} durchgeführt.
Dabei handelt es sich um eine Architekturbewertungsmethode, die auf Szenarien-basierten Methoden wie \gls{saam} (\Citet{saam}) und \gls{atam} (\Citet{kazman_2000}) aufbaut.
Stakeholder definieren gemeinsam gewünschte Szenarien und ordnen diesen \glspl{qa} zu.
Außerdem wird den verschiedenen Szenarien jeweils ein Grad der Schwierigkeit und ein Grad der Wichtigkeit zugewiesen.
Die resultierenden Szenarien können in den \gls{arh} eingegeben werden.

Die Szenarien, die sich dabei im konkreten Fall zu \emph{jadice flow} ergeben haben, sind in Abbildung TODO zu sehen.

TODO nach Durchfürhung von Phase 1: 
 - Beschreibung der Szenarien und QAs
 - Explizite Änderungsvorschläge


\section{Phase 2 - Strategieplanung}
\section{Phase 3a - Architekturplanung}
\section{Herausforderungen bei der Durchführung}
%\section{Analyse der Ergebnisse}
%\section{Erstellung eines Refactoring-Konzepts}
%\subsection{Bestimmung der optimalen Granularität}
%\subsection{Optimierung des Kommunikationsmodells und der Verringerung des IO-Flaschenhalses}
