\chapter{Einleitung}
\label{chap:einleitung}

In dieser Thesis sollen das Framework und Werkzeug mit dem Produkt \emph{jadice flow} evaluiert werden.

Bei der Analyse des Produkts mithilfe des Frameworks sollen vor allem drei Design-Aspekte von Microservices betrachtet werden:
[1] Die optimale Service Granularität, bei der zwischen Overhead durch zu viele Einheiten und Overhead durch zu große Einheiten abgewägt werden muss.
[2] Paradigma der Ansteuerung der einzelnen Einheiten, Kommunikationsmodell zwischen den Einheiten.
[3] Verringerung des IO-Flaschenhalses, Art des Payload-Transfers.
Der IO-Flaschenhals entsteht dadurch, dass die einzelnen Einheiten, bevor sie ihre Arbeit verrichten können, immer die Input-Dokumente herunterladen und danach die Ergebnisse wieder hochladen müssen.

\section{Refactoring Framework des ESE}
\section{jadice flow}

\emph{jadice flow} ist ein Produkt der Firma \emph{levigo solutions}\footnote{\url{https://solutions.levigo.de/}}.
\emph{levigo solutions} beschäftigt über 30 Mitarbeiter, von denen ungefähr sechs seit vier Jahren an \emph{jadice flow} arbeiten.
\emph{jadice flow} basiert bereits auf einer Microservices-Architektur, ist jedoch erst die erste Generation nach der Überführung des vorherigen Monoliths \emph{jadice server}\footnote{\url{https://jadice.com/produkte/server/}}.
Beide Produkte bieten Workflow-basierte Dokumentenverarbeitung an, bei der mit komplexen Datenströmen umgegangen werden kann.
Durch den Umstieg auf eine Microservices-Architektur können in der neuen Generation einzelne Verarbeitungsschritte skaliert werden.
Der Großteil der Services ist in Java geschrieben, doch durch den Betrieb in Containern ist jadice flow prinzipiell unabhängig von der Programmiersprache.

Einen beispielhaften Workflow stellt der \emph{E-Mail converter} dar.
Dabei werden E-Mails in Einzelteile aufgeteilt (Körper, Anhänge) und diese dann einzeln in das Zielformat konvertiert.
Die einzelnen Arbeitsschritte dabei können parallelisiert werden und separat skaliert werden.
Am Ende werden die Ergebnisse kombiniert.
Im Gegensatz zu einfacher PDF-Darstellung von E-Mails in geläufigen Programmen bietet \emph{jadice flow} zusätzliche Features, wie die tiefe Analyse und Darstellung von Archiven in Anhängen.

Da \emph{jadice flow} die erste Generation in Microservices-Architektur ist, wird noch einiges Verbesserungspotential vermutet.
Durch die Analyse von \emph{jadice flow 1.0} mithilfe des Frameworks und weiterer Forschung im Rahmen dieser Bachelorarbeit soll ein Refactoring-Prozess für \emph{jadice flow 2.0} angestoßen werden.

Da das Framework für die Anwendung auf Monolithen gedacht ist, soll in dieser Arbeit die Anwendung des Frameworks auf ein bereits migriertes System angewendet werden, um dessen Eigenschaften weiter zu verbessern.
Es ist zu erforschen, inwiefern sich Framework und Werkzeug auf eine bereits existierende Microservice-Architektur anwenden lassen.
Falls es deshalb für diesen Anwendungsfall zielführend ist, wird zusätzlich die ursprüngliche monolithische Anwendung von \emph{jadice flow} (\emph{jadice server}) hinzugezogen.
Die Ergebnisse dessen könnten dann mit \emph{jadice flow} verglichen werden.
