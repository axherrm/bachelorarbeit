\fieldnote
{F16\_P3\_02\_07\_2024\_Methodenanwendung\_2}
{16}
{07.02.2024, 10:00-14:00}
{Zuhause}
{Axel Herrmann}
{3, Anwendung der zweiten Migrationsmethode}
{
  Der Algorithmus \emph{cHAC} nach \result{2} wurde im Detail betrachtet und nach Top-Down Methode angefangen umzusetzen.
  Die Implementierung wurde in Java vorgenommen.
  Für gewisse Details des Pseudo-Codes wurden Java-spezifische Konstrukte verwendet statt der im Algorithmus angegebenen.
  Beispielsweise wurde die Synchronisationsvariable @SIC des Algorithmus durch eine CyclicBarrier\footnote{\url{https://docs.oracle.com/javase/8/docs/api/java/util/concurrent/CyclicBarrier.html}} umgesetzt.
}
{
}
{
  Obwohl der Algorithmus relativ komplex ist, wurde er auf einer abstrakten Ebene verstanden und auf dieser auch umgesetzt.
}
{
  Der Algorithmus ist relativ kompliziert und die Beschreibung dazu fällt kurz aus. Viele Teile mussten sehr oft gelesen werden, bevor Verständnis erlangt werden konnte.
}
{
  Unsicher, ob der Algorithmus umgesetzt werden kann.
}
