\fieldnote
{F01\_P2\_11\_20\_2023\_FilterAuswahl}
{1}
{20.11.2023, 10:00-10:30}
{Holzgerlingen, Büro levigo}
{Axel Herrmann, Product Owner}
{2, Einstellen der Filter des \gls{arh}}
{
	Es wurden Filter für die Suche nach Refactoring-Methoden mit dem \gls{arh} eingestellt.
	Dabei wurden die \glspl{sp} als am wichtigsten betrachtet, weswegen dort einige Eigenschaften inkludiert wurden.
	Bei den anderen Kategorien wurde versucht, nur die wichtigsten Filter zu verwenden, da mit zu vielen Filtern eine geringere Aussagekraft dieser vermutet wird.
	Vor allem \emph{Process Preferences} und \emph{Usability Preferences}, also Eigenschaften, die die interne Funktionsweise der Methode betreffen, wurden als relativ unwichtig und vor allem auch unabhängig vom betreffenden System eingestuft und demnach nur ein einziger dieser Filter verwendet.
	Am wichtigsten neben den \glspl{sp} wurden \emph{Input Preferences} und \emph{Output Preferences} eingestuft, da diese ebenfalls stark vom System abhängen und Ausgangslage und gewünschtes Ergebnis betreffen.
	Daraus sind die folgenden Filter resultiert:
	\begin{tabular}{|p{4.5cm}|p{5.5cm}|}
		\hline
		\textbf{Kategorie} & \textbf{Filter} \\ \hline
		Quality Preferences, System Properties & Autonomy, Complexity, Granularity, Isolation, Technology Heterogenity \\ \hline
		Input preferences, Domain Artifacts & Human Expertise, Version Control System \\ \hline
		Input preferences, Execut-ables & API/Interface, Source Code (Java) \\ \hline
		Process preferences, Process Strategy & Refactor \\ \hline
		Output preferences, Repre-sentation & List of services, Splitting recom-mendations \\ \hline
		Usability Preferences & - \\ \hline
	\end{tabular}
}
{
	Die Filter sind noch nicht unbedingt final. Nach einer Durchsicht der Ergebnisse der Refactoring-Methoden kann eine weitere Anpassung der Filter erfolgen.
}
{
	Die Auswahl der Filter mit dem ARH ist simpel, bietet gute UX.
}
{
	Nicht alle Filterkriterien sind direkt verständlich und beinhalten oft noch keine Beschreibung. Beispiel: \emph{Validation Methods} könnte auch so verstanden werden, dass es auf die Validierung des Systems nach Migration bezieht, bezieht sich aber darauf, wie jeweiligen Methoden bereits validiert wurden.
}
{
	Bei manchen Filterkriterien unsicher bezüglich der Bedeutung, legt sich allerdings nach Diskussion darüber.
	Außerdem etwas unsicher, wie genau die Feldnotizen funktionieren werden, da es die erste Anwendung dieser ist.
}
