\fieldnote
{F3\_P2\_11\_27\_2023\_FilterPrio}
{3}
{27.11.2023, 11:00-11:10}
{Holzgerlingen, Büro levigo}
{Axel Herrmann, Product Owner}
{2, Einstellen der Filter des \gls{arh}}
{
	Die Einstellung der Filter für die Suche nach Refactoring-Methoden mit dem \gls{arh} wurde fortgesetzt.
	Dabei sollten Filter in zwei Prioritäten entstehen, sodass später eine Suche mit allen Filtern und eine mit nur den wichtigsten Filtern durchgeführt werden kann.
	Größtenteils wurden die bereits ausgewählten Filter in die zwei Prioritäten unterteilt, aber in zwei Fällen auch neue, vorher nicht ausgewählte Filter zur geringeren Priorität hinzugefügt.
	Die neuen Filter sind:
	\begin{tabular}{|p{3cm}|p{3.5cm}|p{3cm}|}
		\hline
		\textbf{Kategorie} & \textbf{Filter Prio 1} & \textbf{Filter Prio 2} \\ \hline
		Quality Preferences, System Properties & Autonomy, Granularity, Technology Heterogenity & Complexity, Isolation \\ \hline
		Input preferences, Domain Artifacts & Human Expertise & Version Control System \\ \hline
		Input preferences, Runtime Artifacts & & Log Traces \\ \hline
		Input preferences, Executables & API/Interface, Source Code (No specifica-tion) & Source Code (Java) \\ \hline
		Process preferences, Process Strategy & Refactor & \\ \hline
		Output preferences, Representation & List of services, Split-ting recommendations & Guideline/ Workflow \\ \hline
		Usability Preferences & - & \\ \hline
	\end{tabular}
}
{
	Die Filter sind vorerst finalisiert. Nach einer Durchsicht der Ergebnisse der Refactoring-Methoden könnten weitere Anpassung der Filter erfolgen.
}
{
	Es konnte sich bei jedem Filter schnell geeinigt werden und die Aufgabe war schnell erledigt.
}
{}
{
	Sicherheit bei der Aufgabe, da das Vorgehen nun klarer ist und alles mit dem Betreuer besprochen wurde.
	Außerdem mehr Sicherheit mit Dokumentation in Feldnotizen.
}
