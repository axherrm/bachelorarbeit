\fieldnotelong
{F1\_P2\_11\_20\_2023\_Activity}
{1}
{20.11.2023, 10:00-10:30}
{Holzgerlingen, Büro levigo}
{Axel Herrmann, Product Owner}
{2, Einstellen der Filter des \gls{arh}}
{
	Es wurden Filter für die Suche nach Refactoring-Methoden mit dem ARH eingestellt. Daraus sind die folgenden Filter resultiert:
	\begin{tabular}{|p{3.3cm}|p{4cm}|}
		\hline
		\textbf{Kategorie} & \textbf{Filter} \\ \hline
		Quality Preferences, System Properties & Autonomy, Complexity, Granularity, Isolation, Technology Heterogenity \\ \hline
		Input preferences, Domain Artifacts & Human Expertise, Version Control System \\ \hline
		Input preferences, Executables & API/Interface, Source Code(Java) \\ \hline
		Process preferences, Process Strategy & Refactor \\ \hline
		Output preferences, Representation & List of services, Splitting recommendations \\ \hline
		Usability Preferences & - \\ \hline
	\end{tabular}
	% System Properties: Autonomy, Complexity, Granularity, Isolation, Technology Heterogenity
}
{Die Filter sind noch nicht unbedingt final. Nach einer Durchsicht der Ergebnisse der Refactoring-Methoden kann eine weitere Anpassung der Filter erfolgen.}
{Die Auswahl der Filter mit dem ARH ist simpel, bietet gute UX}
{Nicht alle Filterkriterien sind direkt verständlich und beinhalten oft noch keine Beschreibung. Beispiel: "Validation Methods" könnte sich auch auf die Validierung des Systems nach Migration beziehen, bezieht sich aber darauf, wie jeweilige Methoden bereits validiert wurden.}
{Bei manchen Filterkriterien unsicher bezüglich der Bedeutung, legt sich allerdings nach Diskussion darüber.}