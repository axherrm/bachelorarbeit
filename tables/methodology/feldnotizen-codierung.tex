\begin{table}[!h]
  \centering
  \begin{tabular}{l l m{7cm}}
    \toprule
    \textbf{Kodierung} & \textbf{Subkodierungen} & \textbf{Bedeutung} \\ \midrule
    Gut gelaufen            &              & Feldnotiz hat Eintrag bei \glqq Gut gelaufen\grqq{} \\ \hline
    Schlecht Gelaufen       &              & Feldnotiz hat Eintrag bei \glqq Schlecht gelaufen\grqq{} \\ \hline
    Mehr gut gelaufen       &              & Nach subjektiver Bewertung des Autors bei Betrachtung der Einträge in \glqq Gut gelaufen\grqq{} und \glqq Schlecht gelaufen\grqq{} ist mehr gut gelaufen \\ \hline
    Mehr schlecht gelaufen  &              & Nach subjektiver Bewertung des Autors bei Betrachtung der Einträge in \glqq Gut gelaufen\grqq{} und \glqq Schlecht gelaufen\grqq{} ist mehr schlecht ge\-laufen \\ \hline
    Gutes Gefühl            &              & Eintrag eines guten Gefühls in \glqq Em\-pfin\-dun\-gen\grqq{} \\ \hline
    \multirow{2}{*}[0cm]{Schlechtes Gefühl} & Unsicherheit & Eintrag eines Gefühls von Unsicherheit in \glqq Em\-pfindungen\grqq{} \\
                            & Frustration  & Eintrag eines Gefühls von Frustration in \glqq Em\-pfindungen\grqq{} \\ \hline
    Verbesserungsvorschlag  &              & In der Feldnotiz ist direkt oder in\-di\-rekt in einem beliebigen Feld ein Ver\-bes\-se\-rungs\-vorschlag für den \gls{arh} enthalten \\ \hline
    Einzelarbeit            &              & Axel Herrmann ist alleinige beteiligte Person an der Aktivität der Feldnotiz \\ \hline
    Besprechung             &              & Die Aktivität der Feldnotiz wurde mit anderen Personen in einer Besprechung durchgeführt. Gegenteil von \glqq Einzelarbeit\grqq{} \\
    \bottomrule
  \end{tabular}
  \caption[Kodierung Feldnotizen]{
    Verwendete Kodierung für die Auswertung der Feldnotizen.
  }
  \label{tab:methodik-feldnitzen-kodierung}
\end{table}
