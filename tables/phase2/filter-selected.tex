\begin{table}[!h]
  \centering
  \begin{tabular}{m{3cm} l m{4cm} m{3.5cm}}
    \toprule
    \textbf{Kategorie} & \textbf{Subkategorie} & \textbf{Filter Priorität 1} & \textbf{Filter Priorität 2} \\ \midrule
    Quality Preferences & System Properties & Autonomy, Granularity, Tech\-nology Heterogenity & Complexity, Isolation \\ \hline
%    \multirow{3}{=}[-0.3cm]{Input Preferences} In die mittlere Zeile schreiben sollte es besser zentrieren
    & Domain Artifacts & Human Expertise & Version Control Sys\-tem \\
    Input Preferences & Runtime Artifacts & - & Log Traces \\
     & Executables & API/Interface, Source Code (No specifica\-tion) & Source Code (Java) \\ \hline
    Process Preferences & Process Strategy & Refactor & - \\ \hline
    Output Preferences & Representation & List of services, Split\-ting recommendations & Guideline/ Workflow \\ \hline
    Usability Preferences & - & - & - \\ \midrule
    \multicolumn{2}{l}{\textbf{Summe}} &\textbf{9} & \textbf{6} \\ \bottomrule
  \end{tabular}
  \caption[Gewählte Filter des \acrshort{arh}]{
    Gewählte Filter des \gls{arh} in Phase 2 bei der Suche nach Migrationsstrategien.
    Unterteilt in zwei Prioritätsstufen, wobei Filter der Priorität 1 in einer Evaluationsrunde von Autor und \gls{po} als wichtiger eingestuft wurden.
%    Filter mit Priorität 2 werden nur bei der Suche mit allen Filtern verwendet.
%    Mit 1 priorisierte Filter werden sowohl bei der Suche mit allen als auch in der Suche mit den wichtigsten Filtern verwendet.
  }
  \label{tab:phase2-selected-filter}
\end{table}
