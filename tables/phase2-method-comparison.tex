\begin{table}[!h]
  \centering
  \begin{tabular}{l m{1.9cm} m{3.1cm} m{3.1cm} m{3.1cm}}
    \toprule
    \textbf{Methode}                     & \textbf{Eingabe} & \textbf{Dekompositions-Funktionsweise} & \textbf{Vorteile} & \textbf{Nachteile} \\ \midrule
    \cite{arh-result-no-filter-2}        & User Stories & Genetischer Algorithmus, Semantic Grouping, Manuell & Ausgangsarchitektur passend, Eingabe liegt bei \emph{jadice flow} vor, mehrere Dekompositions-Methoden, Visualisierung möglich &  \\ \hline
    \cite{arh-result-no-filter-3}        & \glspl{bp} & Clustering & & \glspl{bp} liegen nicht vor und müssten erst erstellt werden  \\ \hline
    \cite{arh-result-no-filter-5}        & Quellcode & Clustering & & Werkzeug kann nicht verwendet wer\-den \\ \hline
    \cite{arh-result-important-filter-4} & Quellcode, Daten\-bank\-schema & Clustering & & Datenbank als Ein\-ga\-be unpassend, Kontext \gls{iiot} unpassend \\ \hline
    \cite{arh-result-important-filter-7} & Klassen\-dia\-gramm oder Quell\-code auf Funk\-tions\-ebene & Evolutionärer Algorithmus \emph{NSGA-III} & & Kein Tool dazu, Extraktion der Funktionen manuell, \emph{NSGA-III} ist in Python implementiert und könnte verwendet werden \\ \bottomrule
  \end{tabular}
  \caption[Vergleich der Suchergebnisse in Phase 2 des \gls{mmf}]{
    Vergleich der Suchergebnisse in Phase 2 des \gls{mmf}.
    Diese sind nach der subjektiven Platzierung der Methoden sortiert (\cref{feldnotiz:7,feldnotiz:11}).
  }
  \label{tab:phase2-comparison}
\end{table}