\begin{table}[!h]
  \centering
  \begin{tabular}{ m{2,5cm} m{6cm} m{0.7cm} m{2,5cm} p{0.7cm} }
    \toprule
    \textbf{Name} & \textbf{Beschreibung} & \textbf{W/S} & \textbf{\glspl{qa}} & \textbf{MS} \\
    \midrule
    Dynamische Skalierbarkeit & Wenn die Nutzlast steigt, kann das System dynamisch skalieren & A/B & Scalability, Resource Utilization, Adaptability, Execution Cost & \advantage \\ \hline
    Statische Skalierbarkeit & Bei geplanten Anwendungsfällen soll passend zur Einhaltung einer Zeitgrenze skaliert werden & B/C & Scalability, Resource Utilization, Time Behavior & \advantage \\  \hline
    Jobtemplates& Einzelne Worker sollen modular zu Services kombinierbar sein & A/C & Modularity, Reusability & - \\ \hline
    Neue Worker& Neue Worker können schnell und unkompliziert zum Produkt hinzugefügt werden & A/C & Modularity, Reusability & \advantage  \\ \hline
    Schnelle Abarbeitung & Effiziente Ressourcennutzung durch schnelle Abarbeitung der Worker-Aufrufe  & A/A & Time Behavior, Resource Utilization & \disadvantage \\ \hline
    \glqq Echtzeit\grqq{}-Verarbeitung & Ein Nutzer soll möglichst schnell das Ergebnis seines Auftrags sehen & A/B & Time Behavior & \disadvantage \\ \hline
    Einfaches Deployment & Ein neues Produkt kann aus Funktionsbausteinen innerhalb eines Tages zusammengestellt werden &C/C & Deployability, Modularity, Agility & \disadvantage \\ \hline
    Platform-unabhängigkeit& Das Produkt kann auf vielen Systemen deployed werden & B/A & Deployability, Installability & \advantage \\ \hline
    Fehlertoleranz Massen-verarbeitung & Wenn 1 Dokument von tausenden einen Fehler verursacht, soll nicht die ganze Verarbeitung abbrechen & A/B & Fault-Tolerance &\advantage \\ \hline
    Erholen nach Systemausfall & Bei dem Ausfall des Systems soll die Verarbeitung möglichst dort wieder einsteigen, wo sie aufgehört hat & A/C & Recoverability & - \\
    \bottomrule
  \end{tabular}
  \caption[Im Architekturreview ermittelte Qualitätsanforderungen und Szenarien]{
    Szenarien, die aus dem in Phase 1 durchgeführten Architekturreview resultieren.
    \emph{W/S} gibt die Wichtigkeit/Schwierigkeit der Szenarien in drei Stufen an (A steht für sehr wichtig und sehr schwierig).
    \emph{\glspl{qa}} gibt die Assoziation der Szenarien zu bestimmten \acrfullpl{qa} an; zuerst genannte sind primäre \glspl{qa}, folgende sekundäre \glspl{qa}.
    \emph{MS} gibt die Einschätzung darüber an, ob das jeweilige Szenario von einer Microservices-Architektur profitiert.
  }
  \label{tab:scenarios}
\end{table}
