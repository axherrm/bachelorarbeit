\begin{table}[!ht]
  \centering
  \begin{tabular}{m{2.8cm} | c c c | c}
    \toprule
    \multirow{2}{*}[0cm]{\textbf{Code}} & \multicolumn{3}{c|}{\textbf{Feldnotiz}} & \multirow{2}{*}[0cm]{\textbf{Summe} (3)} \\
     & \textbf{\fn{1}} & \textbf{\fn{2}} & \textbf{\fn{3}} & \\ \midrule
    Gut gelaufen                        & \checkmark & \checkmark & \checkmark & 3 (100\%)  \\ \hline
    Schlecht gelaufen                   & \checkmark &                       & & 1 (33\%)  \\ \hline
    Mehr gut gelaufen                   & \checkmark & \checkmark & \checkmark & 3 (100\%)  \\ \hline
    Mehr schlecht \:\:\:\:\:\: gelaufen & &                                  & & 0 (0\%)    \\ \hline
    Gutes Gefühl                        & & \checkmark & \checkmark            & 2 (67\%)   \\ \hline
    Schlechtes Gefühl                   & \checkmark & &                       & 1 (33\%)   \\
    Unsicherheit                        & \checkmark & &                       & 1 (33\%)   \\
    Frustration                         & & &                                  & 0 (0\%)    \\ \hline
    Verbesserungs\-vorschlag            & \checkmark & &                       & 1 (33\%)   \\ \hline
    Einzelarbeit                        & & &                                  & 0 (0\%)    \\ \hline
    Besprechung                         & \checkmark & \checkmark & \checkmark & 3 (100\%) \\
    \bottomrule
  \end{tabular}
  \caption[Auswertung Kodierung Feldnotizen Filterwahl]{
    Auswertung der Kodierung der Feldnotizen in Phase 2 beim übergeordneten Schritt \emph{Filterwahl}.
    Prozentuale Angaben hinter der Summe beziehen sich auf den Anteil des Auftreten des Codes zu der Anzahl der Feldnotizen.
  }
  \label{tab:auswertung-feldnotizen-1}
\end{table}
