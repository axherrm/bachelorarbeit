\begin{table}[!ht]
  \centering
  \begin{tabular}{m{2.8cm} | c c c c | c}
    \toprule
    \multirow{2}{*}[0cm]{\textbf{Code}} & \multicolumn{4}{c|}{\textbf{Feldnotiz}} & \multirow{2}{*}[0cm]{\textbf{Summe} (4)} \\
     & \textbf{\fn{15}} & \textbf{\fn{16}} & \textbf{\fn{17}} & \textbf{\fn{18}} & \\ \midrule
    Gut gelaufen                        &  & \checkmark &  &                                & 1 (25\%)  \\ \hline
    Schlecht gelaufen                   & \checkmark & \checkmark & \checkmark & \checkmark & 4 (100\%) \\ \hline
    Mehr gut gelaufen                   &  &  &  &                                          & 0 (0\%)   \\ \hline
    Mehr schlecht \:\:\:\:\:\: gelaufen & \checkmark & \checkmark & \checkmark & \checkmark & 3 (75\%)  \\ \hline
    Gutes Gefühl                        &  &  &  &                                          & 0 (0\%)   \\ \hline
    Schlechtes Gefühl                   & \checkmark & \checkmark & \checkmark & \checkmark & 4 (100\%) \\
    Unsicherheit                        &  & \checkmark &  &                                & 1 (25\%)  \\
    Frustration                         & \checkmark &  & \checkmark & \checkmark           & 3 (75\%)  \\ \hline
    Verbesserungs\-vorschlag            &  &  &  &                                          & 0 (0\%)   \\ \hline
    Einzelarbeit                        & \checkmark & \checkmark & \checkmark &            & 3 (75\%)  \\ \hline
    Besprechung                         &  &  &  & \checkmark                               & 1 (25\%)  \\
    \bottomrule
  \end{tabular}
  \caption[Auswertung Kodierung Feldnotizen Methodenanwendung]{
    Auswertung der Kodierung der Feldnotizen in Phase 3 beim übergeordneten Schritt \emph{Methodenanwendung}.
    Prozentuale Angaben hinter der Summe beziehen sich auf den Anteil des Auftretens des Codes zu der Anzahl der Feldnotizen.
  }
  \label{tab:auswertung-feldnotizen-3}
\end{table}
