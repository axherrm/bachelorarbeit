% !TeX document-id = {8e8e29d3-1f25-424a-acf7-0f37323948d8}
% !TeX spellcheck = de-DE
% !TeX encoding = utf8
% !TeX program = lualatex
% !BIB program = biber
% -*- coding:utf-8 mod:LaTeX -*-

% vv  scroll down to line 200 for content  vv


% Template von: https://github.com/latextemplates/scientific-thesis-template/tree/main
% Example des Tempaltes: https://latextemplates.github.io/scientific-thesis-template/main-german.pdf

\let\ifdeutsch\iftrue
\let\ifenglisch\iffalse
\input{pre-documentclass}
\documentclass[
  ngerman,
  % fontsize=11pt is the standard
  a4paper,  % Standard format - only KOMAScript uses paper=a4 - https://tex.stackexchange.com/a/61044/9075
  twoside,  % we are optimizing for both screen and two-side printing. So the page numbers will jump, but the content is configured to stay in the middle (by using the geometry package)
  bibliography=totoc,
  %               idxtotoc,   %Index ins Inhaltsverzeichnis
  %               liststotoc, %List of X ins Inhaltsverzeichnis, mit liststotocnumbered werden die Abbildungsverzeichnisse nummeriert
  headsepline,
  cleardoublepage=empty,
  parskip=half,
  %               draft    % um zu sehen, wo noch nachgebessert werden muss - wichtig, da Bindungskorrektur mit drin
  draft=false
]{scrbook}
\input{config}

% EN: The package scientific-thesis-cover (https://ctan.org/pkg/scientific-thesis-cover) was added to CTAN on January 1, 2018.
%     It is available in recent texlive and miktex installations
\usepackage[
title={Toolunterstütztes Refactoring von Microservices-Architekturen: Eine industrielle Fallstudie},
author={Axel Herrmann},
type=bachelor,
institute={Institut für Software Engineering\\Empirical Software Engineering\\~\\Universitätsstraße 38\\D–70569 Stuttgart},
course={Software Engineering},
examiner={Prof.\ Dr.\ Stefan Wagner},
supervisor={Jonas Fritzsch,\\Frank Intorp},
startdate={25.\ September 2023},
enddate={25.\ März 2024},
logo=uni-logo.png
]{scientific-thesis-cover}
\usepackage{cleveref}
%\usepackage{amssymb} % moved to config.tex before import of "unicode-math"

% Hier stehen alle Abkürzungen
\newacronym{mmf}{MMF}{Microservices Migration Framework}
\newacronym{ese}{ESE}{Empirisches Software Engineering}
\newacronym{arh}{ARH}{Architecture Refactoring Helper}
\newacronym{rest}{REST}{Restful State Transfer}
\newacronym{ci}{CI}{Continuous Integration}
\newacronym{cd}{CD}{Continuous Delivery}

\crefname{forschungsfrage}{Forschungsfrage}{Forschungsfragen}

\makeindex

\begin{document}

%tex4ht-Konvertierung verschönern
\iftex4ht
  % tell tex4ht to create picures also for formulas starting with '$'
  % WARNING: a tex4ht run now takes forever!
  \Configure{$}{\PicMath}{\EndPicMath}{}
  %$ % <- syntax highlighting fix for emacs
  \Css{body {text-align:justify;}}

  %conversion of .pdf to .png
  \Configure{graphics*}
  {pdf}
  {\Needs{"convert \csname Gin@base\endcsname.pdf
      \csname Gin@base\endcsname.png"}%
    \Picture[pict]{\csname Gin@base\endcsname.png}%
  }
\fi

%\VerbatimFootnotes %verbatim text in Fußnoten erlauben. Geht normalerweise nicht.

% DE: wird fuer Tabellen benötigt (z.B. >{centering\RBS}p{2.5cm} erzeugt einen zentrierten 2,5cm breiten Absatz in einer Tabelle
\newcommand{\RBS}{\let\\=\tabularnewline}

% EN: To avoid issues with Springer's \mathplus
%     See also http://tex.stackexchange.com/q/212644/9075
\providecommand\mathplus{+}

% DE: typoraphisch richtige Abkürzungen
\newcommand{\zB}{z.\,B.\xspace}
\newcommand{\bzw}{bzw.\xspace}
\newcommand{\usw}{usw.\xspace}
\renewcommand{\dh}{d.\,h.\xspace}

% EN: from hmks makros.tex - \indexify
\newcommand{\toindex}[1]{\index{#1}#1}

% DE: Tipp aus "The Comprehensive LaTeX Symbol List"
\newcommand{\dotcup}{\ensuremath{\,\mathaccent\cdot\cup\,}}

% DE: Anstatt $|x|$ $\abs{x}$ verwenden.
%     Die Betragsstriche skalieren automatisch, falls "x" etwas größer sein sollte...
\newcommand{\abs}[1]{\left\lvert#1\right\rvert}

% DE: für Zitate
\newcommand{\citeS}[2]{\cite[S.~#1]{#2}}
\newcommand{\citeSf}[2]{\cite[S.~#1\,f.]{#2}}
\newcommand{\citeSff}[2]{\cite[S.~#1\,ff.]{#2}}
\newcommand{\vgl}{vgl.\ }
\newcommand{\Vgl}{Vgl.\ }

% EN: For the algorithmic package
\newcommand{\commentchar}{\ensuremath{/\mkern-4mu/}}
\algrenewcommand{\algorithmiccomment}[1]{\hfill $\commentchar$ #1}

% DE: Seitengrößen - Gegen Schusterjungen und Hurenkinder...
\newcommand{\largepage}{\enlargethispage{\baselineskip}}
\newcommand{\shortpage}{\enlargethispage{-\baselineskip}}

\newcommand{\initialism}[1]{%
  \ifdeutsch%
    \textsc{#1}\xspace%
  \else%
    \textlcc{#1}\xspace%
  \fi%
}
\newcommand{\OMG}{\initialism{OMG}}
\newcommand{\BPEL}{\initialism{BPEL}}
\newcommand{\BPMN}{\initialism{BPMN}}
\newcommand{\UML}{\initialism{UML}}

% green arrow up for improvement
\newcommand{\advantage}{\raisebox{-0.35\height}{\includegraphics[width=.04\textwidth]{up\_icon.png}}}
%\newcommand{\advantage}{\textcolor{green}{{\Large $\mathbf{\uparrow}$}}}
% red arrow down for
\newcommand{\disadvantage}{\raisebox{-0.35\height}{\includegraphics[width=.04\textwidth]{down\_icon.png}}}
%\newcommand{\disadvantage}{\textcolor{red}{$\mathbf{\downarrow}$}}

% Trick to use more than 9 args: https://tex.stackexchange.com/a/509766
\newcommand{\neworrenewcommand}[1]{\providecommand{#1}{}\renewcommand{#1}}

\newcommand{\fieldnote}[9]{
	\neworrenewcommand{\fieldnoteextended}[2]{
		\begin{longtable}{|p{2.5cm}|p{11cm}|}
			%\caption{#1} \label{tab:field-notes-long-#2} \\
			\caption{#1} \label[feldnotiz]{feldnotiz:#2} \\
			
			\hline \textbf{Eintragstyp} & \textbf{Inhalt} \\ \hline
			\endfirsthead
			
			\multicolumn{2}{c}%
			{{\bfseries \tablename\ \thetable{} -- von der vorherigen Seite weitergeführt}} \\
			\hline \textbf{Eintragstyp} & \textbf{Inhalt}
			\endhead
			
			\hline \multicolumn{2}{|r|}{{Wird auf der n\"achsten Seite fortgef\"uhrt}} \\ \hline
			\endfoot
			
			\hline
			\endlastfoot
			
			Feldnotitz Nr. & #2 \\ \hline
			Datum, Uhrzeit & #3 \\ \hline
			Ort &  #4 \\ \hline
			Beteiligte Personen & #5 \\ \hline
			Phase \& Schritt des MMF & #6 \\ \hline
			Aktionen und Entscheidungen &  #7 \\ \hline
			Kommentare &  #8 \\ \hline
			Was lief gut &  #9 \\ \hline
			Probleme &  ##1 \\ \hline
			Empfindungen & ##2 \\ \hline
			
		\end{longtable}
	}
	\fieldnoteextended
}

%\newcommand{\fieldnote}[9]{
%	\neworrenewcommand{\fieldnoteextended}[2]{
%		\begin{table}
%			\centering
%			\begin{tabular}{|l|p{9cm}|}
%				\hline \textbf{Eintragstyp} & \textbf{Inhalt} \\ \hline
%				Feldnotitz Nr. & #2 \\ \hline
%				Datum, Uhrzeit & #3 \\ \hline
%				Ort &  #4 \\ \hline
%				Beteiligte Personen & #5 \\ \hline
%				Phase \& Schritt des MMF & #6 \\ \hline
%				Aktionen und Entscheidungen &  #7 \\ \hline
%				Kommentare &  #8 \\ \hline
%				Was lief gut &  #9 \\ \hline
%				Probleme &  ##1 \\ \hline
%				Empfindungen & ##2 \\ \hline
%			\end{tabular}
%			\caption{#1} \label{tab:field-notes-#2}			
%		\end{table}
%	}
%	\fieldnoteextended
%}

% https://tex.stackexchange.com/a/136050
\newenvironment{shortitemize}{
	\begin{itemize}
		\setlength{\itemsep}{0pt}
		\setlength{\parskip}{0pt}
		\setlength{\parsep}{0pt}     }
	{ \end{itemize}                  } 

% Mark content of the filter table with prio 1 and prio 2
\newcommand{\prioOne}[1]{\ul{\textbf{#1}}}
\newcommand{\prioTwo}[1]{\ul{#1}}

% shortcut for jadice flow
\newcommand{\jf}{\emph{jadice flow}\xspace}

% shortcut for jadice flow
\newcommand{\bpp}{Best Practices und Patterns\xspace}

% commands for results of search
\newcommand{\resultref}[1]{%
\hyperref[res:#1]{#1}\xspace%
}

\newcommand{\result}[1]{%
\hyperref[res:#1]{Er\-geb\-nis #1}%
\ifx1#1 \cite{arh-result-no-filter-1}\fi%
\ifx2#1 \cite{arh-result-no-filter-3}\fi%
\ifx3#1 \cite{arh-result-no-filter-2}\fi%
\ifx4#1 \cite{arh-result-no-filter-4}\fi%
\ifx5#1 \cite{arh-result-no-filter-5}\fi%
\ifx6#1 \cite{arh-result-important-filter-4}\fi%
\ifx7#1 \cite{arh-result-important-filter-7}\fi%
\ifx8#1 \cite{arh-result-no-qas}\fi%
\xspace%
}

% remove whitespace before paragraph
\let\oldparagraph\paragraph
\renewcommand{\paragraph}[1]{\vspace*{-0.5cm}\oldparagraph{#1}}
\pagenumbering{arabic}
\Titelblatt

%Eigener Seitenstil fuer die Kurzfassung und das Inhaltsverzeichnis
\deftriplepagestyle{preamble}{}{}{}{}{}{\pagemark}
%Doku zu deftriplepagestyle: scrguide.pdf
\pagestyle{preamble}
\renewcommand*{\chapterpagestyle}{preamble}



%Kurzfassung / abstract
%auch im Stil vom Inhaltsverzeichnis
\ifdeutsch
  \section*{Kurzfassung}
\else
  \section*{Abstract}
\fi

Aufgrund zahlreicher Vorteile gegenüber monolithischen Anwendungen hat sich das Architekturprinzip von Microservices im letzten Jahrzehnt immer weiter verbreitet.
Faktoren wie die einfachere Skalierbarkeit oder größere Flexibilität bei der Kombination von Technologien können es für viele Produkte sinnvoll machen, von einer monolithischen zu einer Microservices-basierten Architektur zu migrieren.
Die grundlegende Architektur eines Produkts zu ändern, ist jedoch ein sehr zeit- und kostenintensiver Prozess, der viele Risiken birgt.
Eine klare Vorgehensweise für diesen Migrationsprozess zu finden, ist eine schwierige Aufgabe.
Da viele Entwickler dabei nicht auf wissenschaftliche Literatur zurückgreifen, wurde in früheren Arbeiten ein Werkzeug entwickelt, das den Migrationsprozess mit wissenschaftlichen Erkenntnissen unterstützen soll.
Im Rahmen dieser Thesis wird dieses Werkzeug erstmalig für das Refactoring eines realen Produktes hinsichtlich spezifischer Designaspekte eingesetzt.
Dabei konnten quantitative Ergebnisse durch den Vergleich der resultierenden Prototypen mit dem Ausgangsprodukt über vordefinierte Vergleichskriterien und Messmethoden erfasst werden.
Außerdem wurde der Migrationsprozess als Ganzes qualitativ evaluiert.

\cleardoublepage


% BEGIN: Verzeichnisse

\iftex4ht
\else
  \microtypesetup{protrusion=false}
\fi

%%%
% Literaturverzeichnis ins TOC mit aufnehmen, aber nur wenn nichts anderes mehr hilft!
% \addcontentsline{toc}{chapter}{Literaturverzeichnis}
%
% oder zB
%\addcontentsline{toc}{section}{Abkürzungsverzeichnis}
%
%%%

%Produce table of contents
%
%In case you have trouble with headings reaching into the page numbers, enable the following three lines.
%Hint by http://golatex.de/inhaltsverzeichnis-schreibt-ueber-rand-t3106.html
%
%\makeatletter
%\renewcommand{\@pnumwidth}{2em}
%\makeatother
%
\tableofcontents

% Bei einem ungünstigen Seitenumbruch im Inhaltsverzeichnis, kann dieser mit
% \addtocontents{toc}{\protect\newpage}
% an der passenden Stelle im Fließtext erzwungen werden.

\listoffigures
\listoftables

%Wird nur bei Verwendung von der lstlisting-Umgebung mit dem "caption"-Parameter benoetigt
%\lstlistoflistings
%ansonsten:
\ifdeutsch
  \listof{Listing}{Verzeichnis der Listings}
\else
  \listof{Listing}{List of Listings}
\fi

%mittels \newfloat wurde die Algorithmus-Gleitumgebung definiert.
%Mit folgendem Befehl werden alle floats dieses Typs ausgegeben
\ifdeutsch
  \listof{Algorithmus}{Verzeichnis der Algorithmen}
\else
  \listof{Algorithmus}{List of Algorithms}
\fi
%\listofalgorithms %Ist nur für Algorithmen, die mittels \begin{algorithm} umschlossen werden, nötig

% Abkürzungsverzeichnis
\printnoidxglossaries

\iftex4ht
\else
  %Optischen Randausgleich und Grauwertkorrektur wieder aktivieren
  \microtypesetup{protrusion=true}
\fi

% END: Verzeichnisse


% Headline and footline
\renewcommand*{\chapterpagestyle}{scrplain}
\pagestyle{scrheadings}
\pagestyle{scrheadings}
\ihead[]{}
\chead[]{}
\ohead[]{\headmark}
\cfoot[]{}
\ofoot[\usekomafont{pagenumber}\thepage]{\usekomafont{pagenumber}\thepage}
\ifoot[]{}


%% vv  scroll down for content  vv %%

%%%%%%%%%%%%%%%%%%%%%%%%%%%%%%%%%%%%%%%%%%%%%%%%%%%%%%%%%%%%%%%%%%%%%%%%%%%%%%
%
% Main content starts here
%
%%%%%%%%%%%%%%%%%%%%%%%%%%%%%%%%%%%%%%%%%%%%%%%%%%%%%%%%%%%%%%%%%%%%%%%%%%%%%%

\chapter{Einleitung}
\label{chap:einleitung}

In einer Zeit, in der viele Softwarelösungen auf Cloud Computing basieren, ist auch die Microservices-Architektur immer gebräuchlicher geworden.
Viele Produkte wurden innerhalb des letzten Jahrzehnts schon in diese Architektur übergeführt, hauptsächlich mit dem Ziel, von Vorteilen wie besserer Skalierbarkeit, Wartbarkeit und Flexibilität zu profitieren~\cite{Fritzsch_2019,taibi2017processmotivations}.
Abhängig von dem Ausmaß und der Komplexität monolithischer Systeme, kann die Migration in eine Microservices-Architektur jedoch sehr zeitaufwendig, kostenintensiv und kompliziert sein.
Es liegen aktuell eine Vielzahl akademischer Forschungsarbeiten zum Thema Microservices vor, die jedoch nur selten in der Industrie benutzt werden~\cite{fritzsch2022architecturecentric}.

Um diese Migration zu vereinfachen und die wissenschaftliche Expertise der Industrie leichter zugänglich zu machen, hat das ESE das \gls{mmf} entwickelt.
Mit der Entwicklung des \gls{mmf}, das Erkenntnisse über die Migration von monolithischen Anwendungen hin zu Microservices in die Industrie bringen soll, hat das ESE erste Schritte unternommen, um die erwähnte Lücke zwischen Industrie und Forschung zu schließen.

Offen ist bisher jedoch die Anwendung des auf dem Framework basierenden Werkzeugs in der Industrie, durch die das Framework und Werkzeug weiterentwickelt und verbessert werden kann.
Im Rahmen dieser Bachelorarbeit wird das \acrshort{mmf} daher erstmals an einem realen Produkt, \emph{jadice flow}, evaluiert.
Dazu wird das Framework zum Refactoring des Produktes verwendet.
Die Ergebnisse dieses Refactorings werden zur Bewertung der Effektivität des Frameworks und Werkzeuges verwendet, wodurch diese in Zukunft weiter verbessert werden sollen.

Bei der Analyse des Produkts mithilfe des Frameworks werden vor allem drei Designaspekte von Microservices betrachtet:
[1] Die optimale Service Granularität, bei der zwischen Overhead durch zu viele Einheiten und Overhead durch zu große Einheiten abgewogen werden muss.
[2] Paradigma der Ansteuerung der einzelnen Einheiten, Kommunikationsmodell zwischen den Einheiten.
[3] Reduktion des IO-Flaschenhalses, Art des Payload-Transfers.
Der IO-Flaschenhals entsteht dadurch, dass die einzelnen Einheiten, bevor sie ihre Arbeit verrichten können, immer die Eingangsdokumente herunterladen und anschließend die Ergebnisse wieder hochladen müssen.

Daraus folgen die Forschungsfragen, die in dieser Thesis untersucht werden:

\begin{itemize}
	\item \label[forschungsfrage]{forschungsfrage:1} \emph{RQ 1:} Wie kann eine bereits bestehende Microservices-Architektur mit Hilfe des \acrfull{mmf} hinsichtlich konkreter Qualitätsaspekte weiter optimiert werden?
	\begin{enumerate}[i.]
		\item Welche Refactoring-Verfahren eignen sich zur Bestimmung der optimalen Service Granularität einer bestehenden Microservices-Architektur?
		\item Welche Ansätze, Patterns oder Best Practices eignen sich zur Optimierung des Kommunikationsmodells und der Verringerung des IO-Flaschenhalses zwischen den einzelnen Services?
	\end{enumerate}
\end{itemize}

Um die definierten Forschungsfragen zu beantworten, werden zwei Ziele für diese Arbeit definiert:
Im ersten Teil wird das \gls{mmf} benutzt, um die Architektur von \emph{jadice flow} zu überarbeiten.
Dabei wird innerhalb von drei Phasen ein Plan für eine neue Architektur des Produkts entworfen.
Diese Phasen beinhalten eine Sammlung von wissenschaftlich untersuchten Methoden zur Migration zu Microservices sowie passender Pattern und Best Practices.

Im zweiten Teil wird diese neue Architektur praktisch umgesetzt.
Da ein komplettes Refactoring des Systems nicht dem Rahmen dieser Arbeit angemessen wäre, wird lediglich ein Prototyp mit der neuen Architektur umgesetzt.
Bestandteil des zweiten Teils ist deswegen ebenfalls, einen geeigneten Umfang für diesen Prototyp zu finden, sowie Messmethoden festzulegen, die einen Vergleich des Produkts vor und nach dem Refactoring ermöglichen.

Um die gesetzten Ziele zu erreichen und damit die Forschungsfragen zu beantworten, ist diese Thesis in die folgenden Kapitel aufgeteilt:
Anfangs wird in \textbf{\cref{chap:theoretischer-hintergrund}} in den theoretischen Hintergrund eingeführt, der für diese Arbeit relevant ist. 
Es werden auch das \gls{mmf} und \emph{jadice flow} kurz beschrieben. 
Ebenfalls werden ähnliche Arbeiten vorgestellt.
In \textbf{\cref{chap:methodik}} wird die Methodik beschrieben, mit der diese Thesis durchgeführt werden soll und die Forschungsfragen beantwortet werden sollen.
Mit Verwendung des \gls{mmf} wird in \textbf{\cref{chap:anwendung}} am Produkt \emph{jadice flow} die Planung eines Refactorings durchgeführt. 
Es werden die drei Phasen des Frameworks ausgeführt, wodurch schlussendlich ein neuer Architekturplan entsteht, der in \textbf{\cref{chap:implementierung}} prototypisch implementiert wird.
Der entstandene Prototyp dient dann in \textbf{\cref{chap:auswertung}} zum quantitativen Vergleich zwischen dem Produkt vor und nach Refactoring.
Abschließend werden in \textbf{\cref{chap:gueltigkeit}} Einschränkungen und Gültigkeit dieser Arbeit diskutiert und in \textbf{\cref{chap:fazit}} ein Fazit gezogen und der Ausblick diskutiert.  
\chapter{Theoretischer Hintergrund}
\label{chap:theoretischer-hintergrund}

Es liegt bereits eine Vielzahl an akademischen Publikationen zum Architekturprinzip der Microservices vor.
In diesem Kapitel wird in den theoretischen Hintergrund der für diese Thesis relevanten Literatur eingeführt.
Zuerst wird ein kleiner Vergleich zwischen dem häufig genutzten Legacy-Architekturprinzip \emph{Monolith} und dem behandelten neuen Architekturprinzip \emph{Microservices} gezogen.
Danach beschreiben wir aktuelle Arbeiten zur Migration zu Microservices-Architekturen.
Außerdem führen wir in das Werkzeug \gls{arh} ein, das großer Bestandteil und Untersuchungsobjekt dieser Arbeit ist.
Abschließend wird ein Überblick über das Produkt gegeben, das mithilfe des \gls{arh} überarbeitet werden soll.

\section{Vergleich von monolithischen und Microservices-Systemen}

Die monolithische Architektur von Systemen hat lange Zeit einen Großteil der Softwareprodukte ausgemacht und ist der einfachste Weg, Software zu entwickeln.
Die gesamte Funktionalität eines Monolithen befindet sich in einer Applikation, die in sich selbst geschlossen ist. 
Diese Architektur hat durchaus Stärken.
Da es eine sehr einfache Architektur ist, können kleine Anwendungen sehr schnell erstellt, getestet und lauffähig gemacht werden.
Es wird kein kompliziertes Kommunikationsmodell benötigt, da Monolithen oft in einer Programmiersprache und sogar mit nur einem Framework geschrieben sind.

Je größer Monolithen werden, desto stärken zeigen sich allerdings die Schwächen dieser Architektur.
Es entstehen riesige Codebasen und kleine Änderungen erzwingen erneutes Bauen und Testen der gesamten Applikation.
Das steht auch dem modernen agilen Arbeitsprinzip im Weg.
Falls nur einzelne Komponenten eines Monolithen hohe Last erfahren, muss trotzdem das ganze System skaliert werden, was zu verschwendeter Rechenleistung führen kann und die Applikationen ineffizient macht.

Die meisten dieser Probleme werden durch die Microservices-Architektur gelöst.
Dabei besteht ein System aus vielen kleinen (Micro) Services.
Diese zeichnen sich dadurch aus, dass sie möglichst klein gehalten werden und genau eine Aufgabe haben.
Durch die Entkopplung der Services wird im Gegensatz zu Monolithen ein Kommunikationsmodell zwischen den Services nötig.
Dafür existieren verschiedene Lösungen, die im Rahmen dieser Arbeit noch näher untersucht werden, doch ein sehr häufig benutztes ist beispielsweise \gls{rest}.

Die Vorteile dieser Architektur gegenüber Monolithen sind vielseitig.
Da die Services unabhängig voneinander entworfen sind, können sie einzeln implementiert, getestet und gebaut werden.
Dadurch ist die Automatisierung des Bauens und Testens durch \gls{ci} / \gls{cd} einfacher möglich.
Außerdem können Services je nach Last einzeln skaliert werden, was die Effizienz in manchen Szenarien stark erhöht.
Des weiteren sind Entwickler flexibel in der Wahl der Programmiersprache und des Frameworks für einzelne Services, da die einzige Anforderung die Berücksichtigung des gewählten Kommunikationsmodells ist.

Wie bereits erwähnt gibt es allerdings auch Punkte, in denen Monolithen vorteilhafter sind, wodurch eine Microservices-Architektur nicht universell den Monolithen vorzuziehen ist.
Dazu gehört vor allem die Komplexität des Systems, die mit Microservices-Architektur wesentlich erhöht ist.
Das liegt an der Aufteilung in einzelne Services, die ein ausgefeiltes Kommunikationsmodell benötigen.
Die Kommunikation zwischen den Komponenten ist in jedem Fall langsamen und ineffizienter als das Pendant dazu in Monolithen, wo lediglich Funktionen im selben ausgeführten Programm aufgerufen werden.

\section{Migration zu Microservices}

Aufgrund der genannte Vorteile der Microservices-Architektur gegenüber Monolithen, besteht großes Interesse daran, Monolithen zu Microservices zu migrieren.
Der Eingriff in tiefe Architekturprinzipien eines Systems ist natürlich keine einfache Operation.
Deswegen ist die Migration zu Microservices seit einiger Zeit ein stark untersuchtes Thema.

In dieser Thesis wurde sehr vordergründig das Microservices Migration Framework des ESE  untersucht.
Dieses ist aufgrund vorheriger Arbeiten wie \Citet{10.1007/978-3-030-06019-0_10} entstanden.

In 2019 kategorisieren \Citet{10.1007/978-3-030-06019-0_10} zehn verschiedene Migrations-Methoden.
Dabei wird ein Mängel an praktisch anwendbaren Methoden hervorgehoben, die gute Werkzeugunterstützung und Metriken zur Verifikation der Ergebnisse bieten.

Als Folge darauf stellen \Citet{fritzsch2022architecturecentric} 2022 dann die Planung ein solchen Frameworks vor. Der Migrationsprozess mit Framework und dessen Arbeitsweise in drei Phasen wird bereits beschrieben. Auch wird die Entwicklung eines Web-basierten Werkzeugs erwähnt, das das Framework benutzt und Entwickler bei der Migration unterstützt.

\Citet{10.1145/3242163.3242164} beschreiben geläufige Gefahren bei der Migration zu Microservices-Architekturen.
Dabei  werden fünf architektonische Fehlermuster und vier Fehlermuster bei der Migration aufgeführt.
Außerdem werden Lösungsmöglichkeiten für jedes Muster beschrieben.

%\section{Refactoring}
%
%Obwohl in dieser Thesis ein Framework zur Migration zu Microservices im Mittelpunkt steht, sieht man schnell, dass die Migration eng verwandt mit dem Refactoring ist.
%Deswegen werden folgend ein paar Quellen vorgestellt, die das Refactoring von Microservices-Architekturen behandeln.
%
%\Citet{https://doi.org/10.1002/spe.2974} beschreiben in einem Artikel zur $\mu$TOSCA toolchain, wie sie mit dem  $\mu$TOSCA Modell Architekturen von Microservices-basierten Systemen mit dem OASIS Standard TOSCA darstellen können.
%Außerdem beschreiben sie die Möglichkeit, das $\mu$TOSCA Modell eines Systems durch das zugehörige Kubernetes Deployment automatisch zu ermitteln.
%Das $\mu$TOSCA Modell einer Applikation soll dann dazu genutzt werden, potentielle architektonische Probleme zu finden.
%Für das Ermitteln des $\mu$TOSCA Modells und das Identifizieren von Architektur Smells stellen sie zwei Werkzeug Prototypen ($\mu$Miner und $\mu$Freshener) vor.

\section{\acrfull{arh}}

Aufgrund der beschriebenen Vorteile von Microservices Architekturen ist es im Interesse vieler Unternehmen und Entwickler, existierende Applikationen und Systeme mit monolithischer Architektur zu Microservices-basierten Systemen zu migrieren.
Der Planungsprozess dieser Migration ist allerdings sehr zeit- und kostenintensiv und durch große Risiken geprägt, da dabei Systeme sehr tiefliegend verändert werden.
Erschwert wird er durch das Fehlen klarer und strukturierter Anleitungen.
Es wurde bereits durch mehrere Metastudien versucht, eine Übersicht über verschiedene Refactoring- und Migrationsverfahren zu geben, die für Entwickler hilfreich sein kann.
Jedoch bleibt dabei das Problem bestehen, dass sich der aktuelle Stand der Literatur schnell ändert und Metastudien dadurch teilweise obsolet werden.
Dieses Problem könnte durch die Umsetzung der Anleitung in Form eines Werkzeugs, das Entwickler bei der Migration unterstützt, gelöst werden.
Im Gegensatz zu Metastudien könnte dieses Werkzeug dynamisch an den neuesten Forschungsstand angepasst werden.

Aus diesem Grund hat das \gls{ese} das Werkzeug \gls{arh} \cite{arh-github} entwickelt, das in diesem Kapitel vorgestellt wird.
Während \Citet{fritzsch2022architecturecentric} den Entwurf des Frameworks für das Werkzeug beschreiben, wurde das Werkzeug im Rahmen von zwei Masterarbeiten entwickelt \cite{master-daniel-koch}\cite{master-daniel-koch2}\cite{master-tobias-haller}.
Der \gls{arh} ist eine web-basierte Anleitung, die Entwickler durch drei Phasen der Migrationsplanung führt.
Eine Übersicht über diese Phasen und die Funktionsweise des Werkzeugs ist in \cref{fig:arh-overview} zu sehen.
Im Folgenden werden diese drei Phasen genauer beschrieben: 

\begin{itemize}
	\item \textbf{Phase 1: System Comprehension:}
	In Phase 1 wird das Verständnis des Systems, das migriert werden soll, angestrebt. 
	Mit den Stakeholdern sollen dabei strategische Ziele für Produkt und Unternehmen definiert, sowie Qualitätsmerkmale und gewünschte Szenarien identifiziert werden.
	Es sollen sich dadurch mögliche Treiber für die neue Microservices-Architektur herauskristallisieren.
	Durch das resultierende Verständnis des Systems wird ein Vergleich der alten monolithischen Architektur und einer potentiellen neuen Microservices-Architektur möglich.
	Architekten sollen dann mithilfe dieser Informationen eine Entscheidung für oder gegen die Migration zu Microservices fällen.
	\item \textbf{Phase 2: Strategy Definition:}
	Falls sich in Phase 1 für die Migration zu Microservices entschieden wurde, wird in Phase 2 die Planung der Migration begonnen.
	Eine der zwei hauptsächlichen Aufgaben in dieser Phase ist die konkrete Definition, welche Migrationsstrategie benutzt werden soll, um das System zu modernisieren.
	Dafür kann zwischen (momentan) 22 verschiedenen Methoden gewählt werden, die aus akademischen Publikationen stammen.
	Das Werkzeug kann Entwickler dabei unterstützen, indem es aufgrund der Eingaben aus Phase 1 empfohlene Methoden vorschlägt.
	Der zweite Teil dieser Phase besteht darin, einen Service-Identifikationsansatz zu wählen.
	Durch diesen kann in der nächsten Phase eine sinnvolle Aufteilung in einzelne Services bestimmt werden.
	\item \textbf{Phase 3a: Architecture Definition:}
	Phase 3 ist in zwei Teile aufgeteilt. 
	Im ersten Abschnitt wird die neue Architektur definiert. 
	Hier wird, basierend auf der Methode zur Migration, die in der Planungsphase gewählt wurde, die Identifikation von Microservice-Kandidaten realisiert.
	Außerdem wird allgemeiner die Architektur des gesamten Systems geplant, wobei das Werkzeug Entwickler durch eine Liste von vorgeschlagenen Patterns und Best Practices unterstützen kann.
	 Phase 3a und 3b sind eng verbunden, denn durch Erkenntnisse in der Implementierung kann die Planung häufig noch mehrmals überarbeitet werden und dadurch eine neue Implementierung begonnen werden. 
	\item \textbf{Phase 3b: Service Implementation:} In Phase 3b startet dann ein Zyklus von Implementierungen der in Phase 3a definierten Services.
	Bei Implementierung selbst kann das Werkzeug Entwickler nicht unterstützen.
	Doch Bestandteil der Entwicklungszyklen ist auch, dass das entstehende System anhand der Qualitätsmerkmale zu bewerten.
	Dadurch kann eine unpassende Architekturdefinition oder auch eine unpassende Aufteilung in Services frühzeitig erkannt und korrigiert werden.
	Ist diese Phase abgeschlossen, ist eine erste Version des migrierten Systems fertig.
\end{itemize}

\begin{figure}
	\centering
	\includegraphics[width=\textwidth]{figures/ese-mmf-overview.png}
	\caption[Architecture Refactoring Helper Übersicht]{
		Übersicht über den \acrlong{arh} des \gls{ese} \cite{fritzsch2022architecturecentric}. Es sind die drei Phasen des Prozesses zu erkennen, sowie einige Schritte der einzelnen Phasen.
	}
	\label{fig:arh-overview}
\end{figure}

\section{jadice flow}

Das vorgestellte Werkzeug soll in dieser Thesis nun erstmals in der Industrie benutzt werden.
Deswegen wird in diesem Abschnitt näher auf das Produkt \emph{jadice flow} eingegangen, das damit überarbeitet werden soll.

\emph{jadice flow} ist ein Produkt der Firma \emph{levigo solutions}\footnote{\url{https://solutions.levigo.de/}}.
\emph{levigo solutions} beschäftigt über 30 Mitarbeiter, von denen ungefähr sechs seit vier Jahren an \emph{jadice flow} arbeiten.
\emph{jadice flow} basiert bereits auf einer Microservices-Architektur, ist jedoch erst die erste Generation nach der Überführung des vorherigen Monoliths \emph{jadice server}\footnote{\url{https://jadice.com/produkte/server/}}.
Beide Produkte bieten Workflow-basierte Dokumentenverarbeitung an, bei der mit komplexen Datenströmen umgegangen werden kann.
Durch den Umstieg auf eine Microservices-Architektur können in der neuen Generation einzelne Verarbeitungsschritte skaliert werden.
Der Großteil der Services ist in Java geschrieben, doch durch den Betrieb in Containern ist jadice flow prinzipiell unabhängig von der Programmiersprache.

Einen beispielhaften Workflow stellt der \emph{E-Mail converter} dar.
Dabei werden E-Mails in Einzelteile aufgeteilt (Körper, Anhänge) und diese dann einzeln in das Zielformat konvertiert.
Die einzelnen Arbeitsschritte dabei können parallelisiert werden und separat skaliert werden.
Am Ende werden die Ergebnisse kombiniert.
Im Gegensatz zu einfacher PDF-Darstellung von E-Mails in geläufigen Programmen bietet \emph{jadice flow} zusätzliche Features, wie die tiefe Analyse und Darstellung von Archiven in Anhängen.

Da \emph{jadice flow} die erste Generation in Microservices-Architektur ist, wird noch einiges Verbesserungspotential vermutet.
Durch die Analyse von \emph{jadice flow 1.0} mithilfe des Werkzeugs und weiterer Forschung im Rahmen dieser Bachelorarbeit soll ein Refactoring-Prozess für \emph{jadice flow 2.0} angestoßen werden.

Da das Framework für die Anwendung auf Monolithen gedacht ist, soll in dieser Arbeit die Anwendung des Frameworks auf ein bereits migriertes System angewendet werden, um dessen Eigenschaften weiter zu verbessern.
Dadurch wird erforscht, inwiefern sich Framework und Werkzeug auf eine bereits existierende Microservice-Architektur anwenden lassen.
Falls es deshalb für diesen Anwendungsfall zielführend ist, wird zusätzlich die ursprüngliche monolithische Anwendung von \emph{jadice flow} (\emph{jadice server}) hinzugezogen.
Die Ergebnisse dessen könnten dann mit \emph{jadice flow} verglichen werden.




\chapter{Methodik}
\label{chap:methodik}

In diesem Kapitel wird der methodische Aufbau dieser Thesis beschrieben und begründet.
Die Arbeit ist als industrielle Fallstudie in Zusammenarbeit mit \emph{levigo solutions} konzipiert und dient der Evaluation des \gls{mmf} an einer bestehenden Microservices-Architektur.
Diese Fallstudie besteht aus zwei Hauptbestandteilen.
Im ersten Teil wird ein Refactoring des Produktes \emph{jadice flow} nach Anleitung des Frameworks durchgeführt.
Im zweiten Teil wird das Ergebnis dieses Refactorings verwendet, um eine Evaluation des Frameworks und Werkzeugs abzuschließen.

\begin{figure}[!ht]
	\centering
	\includegraphics[width=0.7\textwidth]{methodology.drawio}
	\caption[Überblick über die Methodik dieser Fallstudie]{
		Überblick über die Methodik dieser Fallstudie.
	}
	\label{fig:methodology}
\end{figure}

\section{Anwendung des MMF}

Die Vorgehensweise beim Refactoring ist durch das Framework sehr genau vorgegeben.
Daher wird das Refactoring in dieser Thesis in die gleichen drei Phasen unterteilt wie in \cref{sec:mmf} beschrieben.
In der ersten Phase wird gemeinsam mit den Stakeholdern des Produkts ein Architekturreview nach \Citet{SVAHNBERG20071893} durchgeführt.
Diese Methode wird in \cref{sec:methodik-architekturreview} näher beschrieben.
Das wesentliche Ergebnis dieses Reviews sind die gewünschten \glspl{qa} des Systems.
Diese Attribute sind die Basis für Phase 2 und 3 und werden in den \gls{arh} eingegeben.
Dieser führt Berechnungen mit den \glspl{qa} durch und kann so eine Liste von Refactoring-Methoden vorschlagen, die nach ihrer Eignung für die \glspl{qa} sortiert sind.
Als weitere Eingabe für diese Suche dienen bestimmte Filter, die in \cref{sec:durchführung-phase2} näher erläutert werden.
Aus der resultierenden Liste von Migrationsmethoden werden die besten Vorschläge betrachtet und bewertet, bevor einer davon ausgewählt und in den folgenden Phasen verwendet wird.
Analog dazu wird in Phase 3a auf Basis der \glspl{qa} und Filter aus Phase 2 eine sortierte Liste von Patterns und Best Practices vorgeschlagen.
Das Vorgehen bei der Auswahl der Filter sowie die spätere Inspektion der Vorschläge des \gls{arh} werden in Form von strukturierten Feldnotizen protokolliert.
Deren genauere Funktion und Form dieser wird in \cref{sec:structured-field-notes} näher erläutert.

\subsection{Architekturreview}
\label{sec:methodik-architekturreview}

Die Methodik eines Architekturreviews wird in dieser Thesis nach Vorschlag des \gls{mmf}/\gls{arh} dazu verwendet, Qualitätsanforderungen an das System, sogenannte \acrfullpl{qa}, zu sammeln.
Für Konformität mit dem Werkzeug \gls{arh}, das als Ergebnis des Architekturreviews Szenarien benötigt, die \acrfullpl{qa} beschreiben, beschränkt sich die Auswahl möglicher Verfahren für die erste Phase auf szenarienbasierte Architekturreviews.
Das bedeutet, dass die gewünschten \glspl{qa} des Systems in Form von Szenarien erfasst und dokumentiert werden.
Ein Szenario beschreibt dabei eine Interaktion eines Stakeholders mit dem Produkt \cite{kazman_2000}.
Diese Konkretisierung von Qualitäten soll vor allem die Verständlichkeit fördern.
Dabei werden mit jedem Szenario zugehörige \glspl{qa} assoziiert, wodurch am Ende eine Abbildung aller \glspl{qa} auf ihre Priorität möglich ist.
Diese Abbildung ist das eigentliche Ergebnis, das der \gls{arh} als Eingabe für die weiteren Phasen benötigt.
Die Szenarien sind dabei lediglich ein Zwischenprodukt und sollen dabei helfen, eine genauere Vorstellung davon zu bekommen, welche spezifischen Situationen mit den \glspl{qa} verbunden sind.
Sie ermöglichen auch eine Bewertung der Wichtigkeit und technischen Schwierigkeit anhand konkreter Situationen, was bei abstrakten \glspl{qa} schwieriger wäre.

Folgend werden verschiedene akademische Methoden beschrieben und verglichen und anschließend die verwendete Methode als Abwandlung dieser Methoden erläutert.

\subsubsection{Vergleich verschiedener Methoden}
\label{sec:atam-saam-svahnberg}

Geläufige szenarienbasierte Verfahren sind die \gls{saam} nach \Citet{saam} und die \gls{atam} nach \Citet{kazman_2000}.
\gls{atam} ist der Nachfolger und eine Überarbeitung von \gls{saam}, weshalb folgend lediglich \gls{atam} näher betrachtet wird.
Die Autoren beschreiben mit \gls{atam} eine Methode, die aus insgesamt neun Schritten besteht.
Diese Schritte sollen mit den Stakeholdern des Produkts innerhalb von drei Arbeitstagen durchgeführt werden.
Da das Ausmaß dieser allerdings nicht angemessen für den Rahmen einer Thesis ist \cite{master-marvin-knodel}, würde für diese Arbeit höchstens eine stark abgewandelte Version dieser infrage kommen.
Stattdessen wurde sich in dieser Thesis für eine Modifikation des Architekturreviews nach \Citet{SVAHNBERG20071893} entschieden.
Es handelt sich dabei ebenfalls um eine szenarienbasierte Methode, die auf \gls{saam} und \gls{atam} basiert, jedoch spezieller für kleinere Anwendungsfälle entwickelt wurde.
Die Autoren erläutern die Verwendung dieser zur Evaluation von Studentenprojekten, die ein reales Softwareentwicklungsprojekt mit Industriekunden simulieren und einen Rahmen von 10-15 Teilnehmern und 20 Wochen Arbeitszeit vorsehen.

\subsubsection{Architekturreview nach \Citet{SVAHNBERG20071893}}

Das Architekturreview nach  \Citet{SVAHNBERG20071893} umfasst sechs Schritte (Übersicht in \cref{tab:svahnberg-plan}).
\begin{table}[!h]
  \centering
  \begin{tabular}{l l}
    \toprule
    \textbf{Aktivität} & \textbf{Zeit (in Minuten)} \\ \midrule
    1. Einleitung in das Projekt & 15 \\
    2. Identifikation von Qualitätsanforderungen & 20 \\
    3. Szenarienerhebung & 45 \\
    ~~~~Pause & 20 \\
    4. Architekturpräsentation & 20 \\
    5. Szenario- und Architekturanalyse & 90 \\
    6. Fazit & 15 \\
    \bottomrule
  \end{tabular}
  \caption[Zeitplan für ein Architekturreview nach \citeauthor{SVAHNBERG20071893}]{
    Zeitplan für ein Architekturreview nach \Citet{SVAHNBERG20071893}.
  }
  \label{tab:svahnberg-plan}
\end{table}

Diese sollen in einer etwa vierstündigen, mo\-de\-rierten Gruppendiskussion mit den wichtigsten Stakeholdern des Produkts durchgeführt werden.
Im ersten Schritt gibt der \gls{po} eine Einführung in das Projekt, in der Ziele für das Produkt sowie die Organisation geschildert werden.
Im zweiten Schritt wird in einer Brainstorming-Runde zusammen mit dem \gls{po}, End-Nutzern und Architekten die Identifikation von \glspl{qa} durchgeführt.
Hierfür dient die Liste der Qualitätskriterien gemäß ISO 9126~\cite{ISO-9126} als Vorlage für mögliche \glspl{qa}.
Im dritten Schritt ordnen \gls{po} und End-Nutzer die Liste der \glspl{qa} nach ihrer Wichtigkeit und fügen den drei wichtigsten jeweils zwei Szenarien hinzu.
Nach einer Pause präsentiert im vierten Schritt ein Softwarearchitekt die vorgeschlagene Architektur.
Anschließend werden im fünften Schritt die Szenarien sequenziell analysiert und bewertet, inwiefern sie durch die Architektur erfüllt werden und wo etwaige Probleme bestehen.
Im sechsten Schritt werden alle identifizierten Probleme und Aspekte, die mehr Aufmerksamkeit benötigen, final diskutiert und der gesamte Prozess zusammengefasst.

\subsubsection{Anpassung des Architekturreviews auf den \gls{arh}}

Ein volles Architekturreview, auf Deutsch auch mit Architekturbewertung zu übersetzen, ist generell bei der Verwendung des \gls{arh} und somit in dieser Thesis nicht das primäre Ziel.
Vielmehr entfällt das Hauptziel eines herkömmlichen Architekturreviews: die Bewertung.
Stattdessen wurde die in \cref{tab:architekturreview-plan} dargestellte Methodik zur Extraktion von Szenarien und \glspl{qa} entworfen.
\begin{table}[!h]
  \centering
  \begin{tabular}{m{4.4cm} m{7.6cm} p{1.3cm}}
    \toprule
    \textbf{Schritt nach \Citet{SVAHNBERG20071893}} & \textbf{Sub-Schritt} & \textbf{Zeit} \\ \midrule
    1. Einleitung in die Me\-tho\-dik & & 10 min\\ \hline
    \multirow{2}{=}[0cm]{2. Identifikation von Qualitätsanforderungen} & 1. Umfrage über wichtigste drei Sub-\glspl{qa} &  \multirow{3}{=}[0.2cm]{20 min}\\
    & 2. Diskussion über Priorisierung der Sub-\glspl{qa} & \\ \hline
    \multirow{3}{=}[-0.3cm]{3. Szenarienerhebung} & 1. Erstellen Szenarien zu wichtigsten Sub-\glspl{qa}& \multirow{3}{=}[-0.3cm]{60 min}\\
    & 2. Hinzufügen Wichtigkeit \& technische Schwie\-rig\-keit zu Szenarien  & \\
    & 3. Szenarien mit weiteren \glspl{qa} assoziieren  & \\ \hline
    5. Szenario- und Archi\-tek\-tur\-analyse & 1. Architekturbewertung anhand der Szenarien & 10 min \\
    \bottomrule
  \end{tabular}
  \caption[Zeitplan für das durchgeführte Architekturreview]{
    Zeitplan für das in Phase 1 dieser Thesis durchgeführte Architekturreview.
  }
  \label{tab:architekturreview-plan}
\end{table}


Wie in der Tabelle angemerkt, dienen als Vorlage für die Methodik die ersten Schritte des Ar\-chi\-tek\-turreviews nach \Citet{SVAHNBERG20071893}.
Außerdem werden auch einzelne Aspekte von \gls{atam} \cite{kazman_2000} verwendet.
Die einzelnen Schritte werden folgend näher beschrieben.

Im ersten Schritt ist die Einleitung in die Methodik geplant.
Dabei handelt es sich um eine Mo\-di\-fi\-ka\-ti\-on des nach \Citet{SVAHNBERG20071893} geplanten ersten Schritts, bei dem stattdessen eine Einleitung in das Projekt erfolgt.
Dieser stammt vermutlich aus dem Kontext der Studentenprojekte, bei denen mehrere Architekturreviews am selben Tag von den gleichen Personen durchgeführt werden und somit eine Einleitung in das Projekt nötig ist.
In den meisten Fällen in der Industrie, und so auch in diesem, ist dieser Schritt nicht notwendig, da die Teilnehmer der Besprechung, die Stakeholder, bereits mit dem Produkt vertraut sind.
Stattdessen wird (wie in \gls{atam} \cite{kazman_2000}) im ersten Schritt eine Einleitung in die verwendete Methodik gegeben, da diese dem Team noch unbekannt war.

Mit Schritt 2 beginnt das eigentliche Architekturreview.
Dabei werden die wichtigsten \glspl{qa} in zwei Sub-Schritten gesammelt:
Zunächst erfolgt eine Umfrage, bei der die Teilnehmer jeweils eine Einschätzung über die drei wichtigsten Sub-\glspl{qa} abgeben.
Anschließend wird in einer gemeinsamen Diskussion eine Priorisierung der Sub-\glspl{qa} finalisiert.
Als Basis-\glspl{qa} werden statt der ISO 9126~\cite{ISO-9126} nach \Citet{SVAHNBERG20071893} die von \Citet{master-daniel-koch} speziell für Microservices-Architekturen identifizierten \glspl{qa} verwendet.
Diese \glspl{qa} sind in \cref{fig:qas} dargestellt.
\begin{figure}[!h]
	\centering
	\includegraphics[width=0.7\textwidth]{qas}
	\includegraphics[width=0.3\textwidth]{qas-legend.drawio}
	\caption[Spezielle \acrlongpl{qa} für Microservices-Architekturen]{
		Spezielle \acrfullpl{qa} für Microservices-Architekturen nach \Citet{master-daniel-koch}.
	}
	\label{fig:qas}
\end{figure}
Es handelt sich hierbei größtenteils um die wichtigsten \glspl{qa} aus der ISO 25010 \cite{ISO-25010}, die durch einige spezifische \glspl{qa} für Microservices ergänzt wurden.
Aufgrund der Verwendung des \gls{arh} ist es die einzige sinnvolle Option, diese zu nutzen, da der \gls{arh} nur die Verwendung dieser zulässt.

Die im zweiten Schritt erstellte Priorisierung der \glspl{qa} wird im dritten Schritt dann verwendet, um Szenarien zu erstellen.
Dieser ist in drei Sub-Schritte unterteilt:
Erst werden für die wichtigsten \glspl{qa} Szenarien erstellt.
Diese werden in Form eines Utility festgehalten (siehe \cref{fig:utility-tree}).
Dann wird inspiriert von \gls{atam} \cite{kazman_2000} jedes Szenario erneut betrachtet und eine Bewertung hinsichtlich der Wichtigkeit und technischen Schwierigkeit vorgenommen.
Diese erfolgt auf einer Skala von A bis C, wobei A für die höchste Wichtigkeit und höchste technische Schwierigkeit steht.
Abschließend wird jedes Szenario ein drittes Mal betrachtet und die Assoziation mit anderen \glspl{qa} diskutiert und festgehalten.
Bei diesem Schritt stammt nur der erste Teil aus der Methodik von \Citet{SVAHNBERG20071893} und die anderen beiden Schritte stammen aus dem \gls{arh}.
Dass bei der Vorlage von \Citet{SVAHNBERG20071893} keine vergleichbare Bewertung der Szenarien und Assoziation der Szenarien mit weiteren \glspl{qa} beschrieben ist, liegt vermutlich am Kontext des Architekturreviews.
Im Kontext der qualitativen Bewertung durch Menschen sind diese nicht explizit notwendig.
Im Vergleich dazu können bei der automatisierten Auswertung durch ein Werkzeug wie den \gls{arh} die Szenarien nicht selbstständig eingeschätzt werden.
Außerdem stammt die Visualisierung durch einen Utility Tree aus \gls{atam} \cite{kazman_2000}.
Ein solcher soll dabei helfen, Qualitätsziele zu konkretisieren und priorisieren~\cite{kazman_2000}.
Dabei werden nach einem semantisch unbedeutenden Wurzelknoten von links nach rechts die \glspl{qa} in Sub-\glspl{qa} verfeinert, worauf dann auf dritter Ebene Szenarien folgen.
Wie in \cref{fig:utility-tree} zu sehen ist, enthält die in dieser Thesis verwendete Notation des Utility Trees auch die Einschätzung der Szenarien in Wichtigkeit und technische Schwierigkeit.
\begin{figure}[!h]
	\centering
	\includegraphics[width=\textwidth]{utility-tree.drawio}
	\caption[Utility Tree nach \acrshort{atam}]{
		Utility Tree nach \gls{atam} \cite{kazman_2000}, der auf die Verwendung mit dem \gls{arh} angepasst ist.
	}
	\label{fig:utility-tree}
\end{figure}


Im letzten Schritt erfolgt eine Bewertung, bei der jedem Szenario zugeordnet wird, ob eine \gls{msa} dafür vorteilhaft ist.
Diese Bewertung kann als Teil des fünften Schritts nach \Citet{SVAHNBERG20071893} gesehen werden.
Abgesehen davon wird allerdings der Teil der Bewertung der Architektur unter den Szenarien im Architekturreview komplett ausgelassen und der Rest der Schritte 4, 5 und 6 nach \Citet{SVAHNBERG20071893} nicht durchgeführt.
Während bei allen genannten szenarienbasierten Verfahren die Szenarien und \glspl{qa} lediglich ein Zwischenprodukt und Mittel zu Bewertung der Systemarchitektur sind, stellen sie bei dem im Rahmen dieser Thesis durchgeführten Architekturreview das gewünschte Endprodukt dar.

Die vorgestellte Methodik kann als neue Methodik zur Extraktion von Szenarien und \glspl{qa} betrachtet werden und könnte in Zukunft als Vorlage für die Anwendung mit dem \gls{arh} verwendet werden.
Trotz des Fehlens des Bewertungsschritts der Methodik wird sie folgend zur Vereinfachung als Architekturreview bezeichnet.

\subsection{Strukturierte Feldnotizen}
\label{sec:structured-field-notes}
Während Phase 1 mit der Methodik einer Fokusgruppe durchgeführt wird, findet die Bearbeitung in den fol\-gen\-den Phasen größtenteils in Einzelarbeit statt.
Um dabei strukturiert zu protokollieren, welche Aktivitäten im Rahmen dieser Phasen durchgeführt werden, werden systematisch durchgeführte Aktivitäten mit gemachten Erfahrungen und Herausforderungen dokumentiert. %und am Ende aus\-ge\-wer\-tet.
Dazu werden wie in der ähnlichen Arbeit von \Citet{master-marvin-knodel} strukturierte Feldnotizen nach \Citet{seaman2008qualitative} verwendet.

Jede Feldnotiz wurde zeitnah nach Beobachtung in einer zugehörigen Datei mit dem Na\-mens\-sche\-ma F\emph{Nummer}\_P\emph{Phase\gls{mmf}}\_\emph{Datum}\_\emph{Name}.tex erstellt.
Dabei wurden dementsprechend Nummer der Feldnotiz, Phase des \gls{mmf}, Datum (im Format \emph{MM\_DD\_JJJJ}) und Name der Feldnotiz im Dateinamen festgehalten.
\cref{feldnotiz:1} hat beispielsweise den Namen \emph{F01\_P2\_11\_20\_2023\_FilterAuswahl.tex}.
\cref{tab:field-notes} gibt einen Überblick über die Eintrags-Felder und welchem Feld nach \Citet{seaman2008qualitative} sie entsprechen.

\begin{table}[!h]
  \centering
  \begin{tabular}{l m{7cm}}
    \toprule
    \textbf{Eintrag} & \textbf{Eintrag nach \Citet{seaman2008qualitative}} \\ \midrule
    Feldnotiz Nr. & -  \\ \hline
	Datum, Uhrzeit & Zeit  \\ \hline
	Ort	 & Ort \\ \hline
	Beteiligte Personen	 & Teilnehmer der Beobachtung \\ \hline
	Phase \& Schritt des \gls{mmf} & Verfolgte Ziele  \\ \hline
	Aktionen und Entscheidungen & Stattgefundene Handlungen und Diskussionen\\ \hline
	Kommentare & Kommentare \\ \hline
	Was lief gut & Diskussion \\ \hline
	Probleme & Diskussion \\ \hline
	Empfindungen &Ton und Stimmung der Interaktionen \\
    \bottomrule
  \end{tabular}
  \caption[Aufbau der strukturierten Feldnotizen nach \citeauthor{seaman2008qualitative}]{
  	Aufbau der strukturierten Feldnotizen nach \Citet{seaman2008qualitative}.
  }
  \label{tab:field-notes}
\end{table}


\section{Evaluation des MMF}

Das Forschungsziel dieser Arbeit ist die Evaluation des \gls{mmf} und des \gls{arh}.
Nach dem Refactoring von \jf mit dem Framework wird evaluiert, als wie hilfreich sich dieses bei der Optimierung einer bereits vorhandenen \gls{msa} herausgestellt hat.
Dafür werden Experteninterviews durchgeführt sowie die im Verlauf der Anwendung erstellten Feldnotizen ausgewertet.

\subsection{Experteninterviews}
\label{sec:methodik-interviews}
Im Rahmen dieser Arbeit werden Experteninterviews durchgeführt, um die Nützlichkeit des \gls{mmf} und \gls{arh} bei der Migration zu Microservices im Spezialfall \jf zu evaluieren.
Interviews im Software Engineering allgemein werden oft genutzt, um Softwareprozesse zu bewerten \cite{seaman2008qualitative}.

\subsubsection{Aufbau}
Es gibt verschiedene Arten von Interviews sowie viele verschiedene Vorgehensweisen bei der Planung, Durchführung und Auswertung der Interviews.
In den nächsten Abschnitten wird beschrieben und begründet, welche Art und welche Vorgehensweisen verwendet werden.

\paragraph{Art:} Ein wichtiges Unterscheidungsmerkmal bei Interviews ist die Offenheit der Fragen.
\Citet{seaman2008qualitative} beschreibt zwei Ausprägungen: strukturierte und unstrukturierte Interviews.
Während struk\-tu\-rier\-te Interviews klare Antworten auf spezifische Fragen erwarten, haben unstrukturierte Interviews den Vorteil, dass auch unvorhergesehene Informationen zu den Antworten auf offene Fragen gehören können.
Um die Vorteile beider Varianten zu kombinieren, werden in dieser Arbeit semi-strukturierten Interviews \cite{seaman2008qualitative} verwendet.
Dabei werden spezifische Fragen gestellt, aber auch offene Fragen mit eingebunden, um die Flexibilität der potentiellen Ergebnisse zu erhöhen.

\paragraph{Experten:} Wie aus dem Namen hervorgeht, ist die Auswahl der Experten eine wichtige Entscheidung bei der Planung von Experteninterviews.
Die Expertise, die in diesem Fall für die Interviews von Bedeutung ist, ist \glqq Betriebswissen\grqq{} \cite{Meuser2009}.
Als Experten wurden Mitarbeiter mit guter Kenntnis des Produkts \jf ausgewählt, da die Fragen einen besonderen Bezug zur Nützlichkeit des Frameworks für das Produkt haben.
\Citet{Runeson2009} empfehlen, aufgrund der qualitativen Natur von Fallstudien, möglichst verschiedene Teilnehmer auszuwählen, anstatt nach Ähnlichkeiten zu suchen.
Aus diesem Grund wurde bei der Wahl der Mitarbeiter nach ihrer Rolle in der Produktentwicklung von \jf differenziert.
Für die abstrakteste Perspektive auf das Produkt wurde der \acrlong{po} (nach Definition des SCRUM von \Citet{SCRUM}) ausgewählt.
Für die technische Perspektive wurde der \gls{se} hinzugezogen.
Als weitere Meinung einer Person, die zwischen den beiden genannten Rollen liegt, wurde der \gls{sa} herangezogen.
Eine Übersicht über die drei ausgewählten Teilnehmer ist in \cref{tab:expert-interviewees} zu finden.
\begin{table}[!ht]
  \centering
  \begin{tabular}{|l|l| p{1.8cm} p{1.5cm} p{2cm}|c|}
    \toprule
    \multirow{2}{*}[0cm]{\textbf{ID}} & \multirow{2}{*}[0cm]{\textbf{Rolle}} & \multicolumn{3}{c|}{\textbf{Erfahrung in Jahren}} & \textbf{An Thesis beteiligt} \\
     & & Software-entwicklung & Micro-services & \emph{jadice flow/ jadice server} & \\ \midrule
    \acrshort{po} & \acrlong{po} & \multicolumn{1}{c}{20} & \multicolumn{1}{c}{6} & \multicolumn{1}{c|}{13} & Ja \\
    \acrshort{sa} & \acrlong{sa}       & \multicolumn{1}{c}{10} & \multicolumn{1}{c}{4} & \multicolumn{1}{c|}{7} & Nein \\
   \acrshort{se} & \acrlong{se}     & \multicolumn{1}{c}{10} & \multicolumn{1}{c}{4} & \multicolumn{1}{c|}{7} & Nein \\
    \bottomrule
  \end{tabular}
  \caption[Teilnehmer der Experteninterviews]{
    Teilnehmer der Experteninterviews.
    An Thesis beteiligt bedeutet, ob der Teilnehmer an der Wahl der Migrationsverfahren beteiligt war.
    Der \acrshort{se} war ebenfalls an Phase 1 des MMF beteiligt, das hat aber keine Relevanz für diese Interviews.
  }
  \label{tab:expert-interviewees}
\end{table}


\paragraph{Protokollierung:} Zwei geläufige Optionen für die Aufnahme eines Interviews sind die Pro\-to\-kol\-lie\-rung während des Interviews, entweder durch den Interviewer oder einen separaten Protokollant \cite{seaman2008qualitative,Runeson2009}.
Der Interviewer selbst ist besonders im Fall von Unerfahrenheit keine gute Wahl, da es kompliziert sein kann, zwei Aufgaben gleichzeitig zu bewältigen und die Leitung des Interviews somit leiden kann.
Die Lösung durch einen Protokollanten ist auch nicht optimal, da ein solcher nicht immer zur Verfügung steht.
Außerdem kann es zu Interpretationsunterschieden kommen, welche Informationen relevant sind.
Deswegen wird, wie von \Citet{seaman2008qualitative} sowie \Citet{Runeson2009} empfohlen, eine Aufzeichnung durch Audioaufnahme verwendet.

\paragraph{Leitfaden:} Eine weitere wichtige Vorbereitung für ein gut geplantes Interview ist die Erstellung eines Leitfadens \cite{seaman2008qualitative,hove-anda-2005}, welcher den Plan des Vorgehens enthält.
Dieser muss nicht so strikt gegliedert sein wie ein Fragebogen, sondern soll neben den Fragen Notizen für den Interviewer enthalten, die bei der Leitung des Interviews helfen.
Der für diese Interviews konzipierte Leitfaden ist in \cref{chap:expert-interviews-leitfaden} zu finden.
Er besteht aus hauptsächlich drei Teilen.
Im ersten Abschnitt werden Verständnisfragen gestellt.
Diese sollen absichern, dass die Informationen aus dem Informationsbogen ausreichend verstanden wurden, um die folgenden Fragen sinnvoll beantworten zu können.
Im zweiten Teil werden allgemeine inhaltliche Fragen zur Evaluation des \gls{mmf} gestellt, ohne Bezug zur Anwendung in dieser Thesis.
Mithilfe mehrerer Fragen sollen mögliche alternative Funktionsweisen, die Phasen des Frameworks und potenzielle Probleme dessen evaluiert werden.
Der letzte Teil dient der Bewertung der spezifischen Anwendung des Frameworks in dieser Thesis.
Dazu werden zunächst Fragen zu den mit dem \gls{arh} ermittelten Migrationsverfahren gestellt.
Da ein gutes Verständnis der Verfahren dafür nötig ist, werden die Fragen weiter spezifiziert, falls der Kandidat bei offenen Fragen in falsche Richtungen abschweift oder keine Antwort findet.
Außerdem wird für Bezug zur \hyperref[forschungsfrage:1]{Forschungsfrage} die Auswirkung der Verfahren auf die Granularität des Systems befragt.
Anschließend werden die mit \gls{arh} erhaltenen \bpp bewertet.
Auch hier werden zunächst offene Fragen gestellt und am Ende eine Frage mit Bezug zu den Qualitätsaspekten aus der \hyperref[forschungsfrage:1]{Forschungsfrage} gestellt.
Zum Abschluss werden dem Interviewten noch zwei Fragen zur Gesamtbewertung der Anwendung des Frameworks gestellt.
Dabei soll er seine Einzelbewertungen der letzten Schritte zusammenfassen und ein Fazit ziehen.

\paragraph{Vorbereitungsmaterial:}
Die Einleitung eines Interviews sollte nur einen kleinen Teil ausmachen, da Zeit in Interviews wertvoll und oft knapp bemessen ist \cite{Runeson2009,hove-anda-2005,seaman2008qualitative}.
Aus diesem Grund wurden für die Interviews eine Präambel und ein Informationsbogen erstellt, welche die Teilnehmer im Voraus erhalten haben.
%Diese sind in \cref{chap:expert-interviews-preamble,chap:expert-interviews-infobogen} zu finden.
Diese sind in den Anhängen \hyperref[chap:expert-interviews-preamble]{B} und \hyperref[chap:expert-interviews-infobogen]{C} zu finden.
Die Prä\-am\-bel fasst die generellen Richtlinien für die Experteninterviews zusammen.
Dabei diente die Präambel von Bogner et al.\footnote{\url{https://github.com/xJREB/research-microservices-interviews/blob/master/interview-preamble.md}} als Vorlage.
Da der thematische Kontext des Interviews durch die Voraussetzung des Verständnisses von zwei Migrationsmethoden relativ komplex ist, wurde außerdem ein Informationsbogen erstellt, der bei der thematischen Einarbeitung helfen soll.
Darin werden alle wichtigen Konzepte erklärt, die in den Interviews behandelt werden.
Der Bogen enthält einen generellen Überblick über  \gls{mmf}/\gls{arh} sowie die Funktionsweisen der zwei diskutierten Migrationsverfahren.
Es wird jedoch nicht erwartet, dass die Teilnehmer alle Inhalte im Voraus vollständig verstehen, da (wie im Leitfaden beschrieben) anfangs eine Runde zur Klärung der Verständnisfragen stattfindet.

\subsubsection{Analyse der Daten}

Die Analyse qualitativer Daten ist oft ein Prozess der parallel zu der Datensammlung stattfindet \cite{Runeson2009}.
Es ist wichtig, einen gut strukturierten und systematischen Plan für die Datenanalyse zu haben, da die Erkenntnisse in der Analyse die Sammlung der nächsten Daten beeinflussen können.
Wie \Citet{seaman2008qualitative} empfiehlt, wurde die Datenanalyse deswegen im Voraus geplant.
Obwohl es empfehlenswert ist, die Auswertung durch mehrere Forschende durchzuführen, um Voreingenommenheit zu reduzieren \cite{Runeson2009}, wird sie in dieser Arbeit aufgrund mangelnder Ressourcen nur vom Autor durchgeführt.

Die Rohdaten liegen in Form von drei Audioaufnahmen vor, die jeweils etwa eine Stunde lang sind.
Die Audioaufnahmen werden mithilfe der von \Citet{Mayring2019} vorgestellten Techniken  \emph{Zusammenfassen} und \emph{Explikation} getrennt transkribiert (siehe \cref{fig:datenanalyse}).
Die Ergebnisse dieser Transkription werden auf zwei verschiedene Arten in diese Thesis übernommen.
Zum einen wird eine \emph{Strukturierung} nach \Citet{Mayring2019} durchgeführt.
Dabei werden für jede gestellten Fragen die Antworten kategorisiert.
Das dafür entworfene \glqq ordinal geordnete Kategoriensystem \grqq{} nach \Citet{Mayring2019} ist in \cref{fig:kodierleitfaden} zu sehen.
Auf der anderen Seite werden die Transkripte, sowie die Kodierungen in textueller Form aggregiert.
Eine Übersicht über die dadurch entstandenen aggregierten Zusammenfassungen ist in \cref{tab:expert-interviews-analysis} gegeben.

%Diese enthalten Sammlungen von langen Texten, welche folgend auf einen geringeren Umfang gekürzt werden, ohne dabei relevante Informationen zu verlieren.
%Dabei werden die Techniken \emph{Zusammenfassen}, \emph{Explikation} und \emph{Strukturierung} nach \Citet{Mayring2019} verwendet.
%Das Ergebnis daraus ist jeweils eine textuelle Zusammenfassung der aggregierten Ergebnisse pro Fragenabschnitt:
%\begin{itemize}
%	\item Evaluation des \gls{mmf}/\gls{arh} allgemein
%	\item Evaluation der ausgewählten Migrationsmethoden für \jf
%	\item Evaluation der \bpp für \jf
%	\item Evaluation der Anwendung des \gls{mmf}/\gls{arh} auf \jf insgesamt
%\end{itemize}

\begin{figure}[!h]
	\centering
	\includegraphics[width=0.8\textwidth]{Datenanalyse.drawio}
	\caption[Vorgehen Datenanalyse Experteninterviews]{
		Vorgehen bei der Datenanalyse der Experteninterviews.
	}
	\label{fig:datenanalyse}
\end{figure}


\begin{table}[!ht]
  \centering
  \begin{tabular}{m{5cm} m{6cm} l}
    \toprule
    \textbf{Kategorie} & \textbf{Fragenabschnitt} & \textbf{Abschnitt} \\ \midrule
    Evaluation des \gls{mmf}/\gls{arh} all\-ge\-mein & & \cref{sec:evaluation-mmf-allgemein} \\ \hline
%    \multirow{3}{=}[-0.6cm]{Evaluation der Anwendung des \gls{mmf}/\gls{arh} auf \jf} 
    & Evaluation der ausgewählten Mi\-gra\-ti\-ons\-me\-tho\-den für \jf & \cref{sec:evaluation-mmf-anwendung-methoden} \\
    Evaluation der Anwendung des \gls{mmf}/\gls{arh} auf \jf & Evaluation der \bpp für \jf & \cref{sec:evaluation-mmf-anwendung-bp-patterns} \\
     & Evaluation der Anwendung des \gls{mmf}/ \gls{arh} auf \jf insgesamt & \cref{sec:evaluation-mmf-anwendung-insgesamt} \\
    \bottomrule
  \end{tabular}
  \caption[Abschnitte der Auswertung der Experteninterviews]{
    Abschnitte der Auswertung der Experteninterviews.
  }
  \label{tab:expert-interviews-analysis}
\end{table}


\begin{figure}
	\centering
	\includegraphics[width=\textwidth]{kodierleitfaden.drawio}
	\caption[Kodierleitfaden Auswertung Experteninterviews]{
		Kodierleitfaden für die Auswertung der Experteninterviews.
	}
	\label{fig:kodierleitfaden}
\end{figure}

\subsection{Strukturierte Feldnotizen}
\label{sec:methodik-auswertung-feldnotizen}

Nachdem alle Feldnotizen erstellt sind, werden diese qualitativ ausgewertet.
Auch hier wird die Strukturierung nach \Citet{Mayring2019} verwendet.
Dafür wurde eine Kodierung erstellt, welche in \cref{tab:methodik-feldnitzen-kodierung} zu sehen ist und an vielen Stellen von der Kodierung aus der ähnlichen Arbeit von \Citet{master-marvin-knodel} inspiriert ist.

\begin{table}[!h]
  \centering
  \begin{tabular}{l l m{7cm}}
    \toprule
    \textbf{Kodierung} & \textbf{Subkodierungen} & \textbf{Bedeutung} \\ \midrule
    Gut gelaufen            &              & Feldnotiz hat Eintrag bei \glqq Gut gelaufen\grqq{} \\ \hline
    Schlecht Gelaufen       &              & Feldnotiz hat Eintrag bei \glqq Schlecht gelaufen\grqq{} \\ \hline
    Mehr gut gelaufen       &              & Nach subjektiver Bewertung des Autors bei Betrachtung der Einträge in \glqq Gut gelaufen\grqq{} und \glqq Schlecht gelaufen\grqq{} ist mehr gut gelaufen \\ \hline
    Mehr schlecht gelaufen  &              & Nach subjektiver Bewertung des Autors bei Betrachtung der Einträge in \glqq Gut gelaufen\grqq{} und \glqq Schlecht gelaufen\grqq{} ist mehr schlecht ge\-laufen \\ \hline
    Gutes Gefühl            &              & Eintrag eines guten Gefühls in \glqq Em\-pfin\-dun\-gen\grqq{} \\ \hline
    \multirow{2}{*}[0cm]{Schlechtes Gefühl} & Unsicherheit & Eintrag eines Gefühls von Unsicherheit in \glqq Em\-pfindungen\grqq{} \\
                            & Frustration  & Eintrag eines Gefühls von Frustration in \glqq Em\-pfindungen\grqq{} \\ \hline
    Verbesserungsvorschlag  &              & In der Feldnotiz ist direkt oder in\-di\-rekt in einem beliebigen Feld ein Ver\-bes\-se\-rungs\-vorschlag für den \gls{arh} enthalten \\ \hline
    Einzelarbeit            &              & Axel Herrmann ist alleinige beteiligte Person an der Aktivität der Feldnotiz \\ \hline
    Besprechung             &              & Die Aktivität der Feldnotiz wurde mit anderen Personen in einer Besprechung durchgeführt. Gegenteil von \glqq Einzelarbeit\grqq{} \\
    \bottomrule
  \end{tabular}
  \caption[Kodierung Feldnotizen]{
    Verwendete Kodierung für die Auswertung der Feldnotizen.
  }
  \label{tab:methodik-feldnitzen-kodierung}
\end{table}


Bewusst wurde an einigen Stellen vom Original abgewichen.
Die Definition der Codes \glqq Mehr gut gelaufen\grqq{} und \glqq Mehr schlecht gelaufen\grqq{} über die Anzahl der positiven Aspekte im Vergleich zur Anzahl der negativen Aspekte wurde zum Beispiel nicht übernommen.
Es wurde sich stattdessen für eine Definition dieser über die Wahrnehmung des Autors nach Betrachtung der positiven und negativen Aspekte der jeweiligen Feldnotiz entschieden.
Das hat natürlich den Nachteil einer geringeren Objektivität.
Trotzdem wurde es als sinnvoller eingeschätzt, da sonst keine Gewichtung der positiven und negativen Aspekte stattfindet.
Dass die Ergebnisse je nach Definition dieser Kodierung abweichen wurde bei Betrachtung erkannt und sich deswegen für die angepasste Variante entschieden.


\chapter{Anwendung des Frameworks auf jadice flow}
\label{chap:anwendung}

In diesem Kapitel wird ein Architektur-Refactoring am Produkt \emph{jadice flow} geplant und in theoretischer Ebene durchgeführt.
Dazu wird das \gls{mmf} benutzt, das in \cref{sec:mmf} beschrieben wird.
Unterteilt ist der Prozess in drei Phasen, deren Durchführung im Folgenden beschrieben wird.

\section{Phase 1 - Architekturreview}

In dieser Phase wurde einer Fokusgruppe ein Architekturreview nach \Citet{SVAHNBERG20071893} wie in \cref{sec:methodik-architekturreview} beschrieben durchgeführt.
Teil dieser Gruppe waren vier Softwareentwickler beziehungsweise -Architekten
% Dabei handelt es sich um eine Architekturbewertungsmethode, die auf Szenarien-basierten Methoden wie \gls{saam} (\Citet{saam}) und \gls{atam} (\Citet{kazman_2000}) aufbaut.
% Stakeholder definieren gemeinsam gewünschte Szenarien und ordnen diesen \glspl{qa} zu.
% Außerdem wird den verschiedenen Szenarien jeweils ein Grad der Schwierigkeit und ein Grad der Wichtigkeit zugewiesen.
% Die resultierenden Szenarien können in den \gls{arh} eingegeben werden.

Die Szenarien, die sich dabei im konkreten Fall zu \emph{jadice flow} ergeben haben, sind in der \cref{tab:scenarios} zu sehen.

\begin{landscape}
	\begin{figure}
		\centering
		\includegraphics[width=1.5\textwidth]{scenarios.drawio}
		\caption[Utility Tree mit im Architekturreview ermittelten Qualitätsanforderungen und Szenarien]{
			Der Utility Tree mit Szenarien, die aus dem in Phase 1 durchgeführten Architekturreview resultieren.
			Von links nach rechts enthält der Baum folgende Elementarten: [1] Wurzel (ohne Bedeutung), [2] \gls{qa}, [3] Subattribut, [4] Beurteilung des Szenarios hinsichtlich Wichtigkeit und technischer Schwierigkeit, [5] Szenariobeschreibung.
		}
		\label{fig:scenarios}
	\end{figure}
\end{landscape}


\begin{table}
	\centering
	\begin{tabular}{ m{2,5cm} m{6cm} m{0.7cm} m{2,5cm} p{0.7cm} }
		\toprule
		\textbf{Name} & \textbf{Beschreibung} & \textbf{W/S} & \textbf{\glspl{qa}} & \textbf{MS} \\
		\midrule
		Dynamische Skalierbarkeit & Wenn die Nutzlast steigt, kann das System dynamisch skalieren & A/B & Scalability, Resource Utilization, Adaptability, Execution Cost & \advantage \\ \hline
		Statische Skalierbarkeit & Bei geplanten Anwendungsfällen soll passend zur Einhaltung einer Zeitgrenze skaliert werden & B/C & Scalability, Resource Utilization, Time Behavior & \advantage \\  \hline
		 Jobtemplates& Einzelne Worker sollen modular zu Services kombinierbar sein & A/C & Modularity, Reusability & - \\ \hline
		 Neue Worker& Neue Worker können schnell und unkompliziert zum Produkt hinzugefügt werden & A/C & Modularity, Reusability & \advantage  \\ \hline
		 Schnelle Abarbeitung & Effiziente Ressourcennutzung durch schnelle Abarbeitung der Worker-Aufrufe  & A/A & Time Behavior, Resource Utilization & \disadvantage \\ \hline
		 \glqq Echtzeit\grqq{}-Verarbeitung & Ein Nutzer soll möglichst schnell das Ergebnis seines Auftrags sehen & A/B & Time Behavior & \disadvantage \\ \hline
		 Einfaches Deployment & Ein neues Produkt kann aus Funktionsbausteinen innerhalb eines Tages zusammengestellt werden &C/C & Deployability, Modularity, Agility & \disadvantage \\ \hline
		 Platform-unabhängigkeit& Das Produkt kann auf vielen Systemen deployed werden & B/A & Deployability, Installability & \advantage \\ \hline
		 Fehlertoleranz Massen-verarbeitung & Wenn 1 Dokument von tausenden einen Fehler verursacht, soll nicht die ganze Verarbeitung abbrechen & A/B & Fault-Tolerance &\advantage \\ \hline
		 Erholen nach Systemausfall & Bei dem Ausfall des Systems soll die Verarbeitung möglichst dort wieder einsteigen, wo sie aufgehört hat & A/C & Recoverability & - \\
		\bottomrule
	\end{tabular}
	\caption[Im Architekturreview ermittelte Qualitätsanforderungen und Szenarien]{
		Szenarien, die aus dem in Phase 1 durchgeführten Architekturreview resultieren.
		\emph{W/S} gibt die Wichtigkeit/Schwierigkeit der Szenarien in drei Stufen an (A steht für sehr wichtig und sehr schwierig).
		\emph{\glspl{qa}} gibt die Assoziation der Szenarien zu bestimmten \acrfullpl{qa} an; zuerst genannte sind diejenigen, für die die Szenarien ursprünglich erstellt wurden.
		\emph{MS} gibt die Einschätzung darüber an, ob das jeweilige Szenario von einer Microservices-Architektur profitiert.
		}
	\label{tab:scenarios}
\end{table}

TODO nach Durchfürhung von Phase 1:
 - Beschreibung der Szenarien und QAs
 - Explizite Änderungsvorschläge


\section{Phase 2 - Strategieplanung}
\label{sec:durchführung-phase2}

\section{Phase 3a - Architekturplanung}
\section{Herausforderungen bei der Durchführung}
%\section{Analyse der Ergebnisse}
%\section{Erstellung eines Refactoring-Konzepts}
%\subsection{Bestimmung der optimalen Granularität}
%\subsection{Optimierung des Kommunikationsmodells und der Verringerung des IO-Flaschenhalses}

\chapter{Implementierung des Refactorings}
\label{chap:implementierung}

Die im \cref{chap:anwendung} geplanten Änderungen an \emph{jadice flow} werden in diesem Kapitel prototypisch durchgeführt.
Das soll im späteren \cref{chap:auswertung} einen quantitativen Vergleich zwischen dem Produkt vor und nach Refactoring ermöglichen.

\chapter{Auswertung}
\label{chap:auswertung}

Die Auswertung dieser Fallstudie erfolgt mithilfe drei verschiedener Methoden.
Zu Beginn wird eine qualitative Bewertung des Gesamtprozesses vom Autor vorgenommen. 
Diese kann subjektiv sein und soll daher nur als erste Orientierung dienen und durch Experteninterviews unterstützt werden.
Dabei wurden drei andere an \jf beteiligte Personen über die Verwendung des \gls{arh} mit dem Produkt befragt.
Außerdem werden abschließend die Feldnotizen ausgewertet.

\section{Qualitative Bewertung des Refactoringprozesses}

In diesem Abschnitt soll der gesamte Refactoringprozess, den das \gls{mmf} leitet, bewertet werden.
Dabei werden die einzelnen Funktionen des \gls{arh} qualitativ evaluiert.
Der Prozess besteht aus den folgenden wesentlichen Funktionen:
\begin{itemize}
	\item Erfassen der \glspl{qa} in Phase 1
	\item Einstellen von Suchfiltern in Phase 2
	\item Suche nach Migrationsverfahren in Phase 2
	\item Suche nach Patterns und Best Practices in Phase 3
\end{itemize}

Diese Qualität dieser Funktionen kann über die Ergebnisse der Suchen ausgewertet werden, da die ersten beiden Punkte in die Ergebnisse der letzten beiden Punkte einfließen.
Eine hypothetisch optimale Filterfunktion würde erreichen, dass Nutzer die Suchergebnisse nach Einstellen seiner Filter hinsichtlich der Nützlichkeit für sein Ziel genau gleich ordnen würde wie der \gls{arh} es tut.
Natürlich ist eine so genaue Filterfunktion unrealistisch und nicht erreichbar.
Im Folgenden wird versucht, von den Problemen mit Ergebnissen der Suchen Rückschlüsse auf Probleme in Filterfunktion durch Filter und \glspl{qa} zu ziehen.
Wenn beispielsweise hoch platzierte Suchergebnisse vom Nutzer als schlecht bewertet werden, könnte ein Zusammenhang mit fehlenden Filtermöglichkeiten oder zu stark bewerteten Filtern bestehen.

\subsection{Phase 2}

Bei Betrachtung der Ergebnisse in Phase 2 ist auffällig, dass das am höchsten bewertete Ergebnis von \Citet{arh-result-no-filter-1} eine der am wenigsten passenden Methoden beinhaltet.
Das Verfahren wurde aufgrund des Fehlens eines konkreten \gls{sia} ausgeschlossen.
Eine Filterfunktion für das Ziel der Verfahren wäre sinnvoll, um solche Ergebnisse gar nicht erst vorgeschlagen zu bekommen.

 - Fehlende quantitative Einordnung, wie stark die Methode den jeweiligen Filter erfüllt (Ergebnis 1 so oberflächlich, dass es viele SPs anreist, aber keins wirklich spezifisch lösen kann)

\subsection{Phase 3}


\section{Experteninterviews}

\section{Feldnotizen}


\chapter{Validität und Einschränkungen}
\label{chap:gueltigkeit}

Eine umfassende Berücksichtigung der Validitätseinschränkungen ist für die Bewertung der Relevanz einer empirischen Studie von großer Bedeutung. 
Jede Arbeit hat durch die verwendete Methodik Einschränkungen hinsichtlich ihrer Validität \cite{campbell2015experimental}.
Auch diese Arbeit hat gewisse Limitationen, die in diesem Kapitel erörtert werden.
\Citet{Runeson2009} stellen eine mögliche Klassifizierung der verschiedenen Faktoren vor, die Einfluss auf die Validität haben.
Dieses Kapitel ist entsprechend die vier Klassen \emph{konstruktionsbedingte Validität}, \emph{Interne Validität}, \emph{Reliabilität} und \emph{Externe Validität} strukturiert.

\section{Konstruktionsbedingte Validität}

Dieser Aspekt der Validität (original \emph{Construct validity}) drückt aus, inwieweit die verwendete Methodik tatsächlich die gestellte Forschungsfrage untersucht und ein Ergebnis liefert, das diese Forschungsfrage beantwortet \cite{Runeson2009}.

Die übergeordnete Methodik der gesamten Fallstudie besteht in erster Linie aus der Anwendung des \gls{mmf} auf \jf und der anschließenden Auswertung der Ergebnisse dessen.
Der erste Schritt (die Anwendung des \gls{mmf}) wird in der Forschungsfrage erwähnt und zielt direkt auf deren Beantwortung ab.
Im zweiten Schritt (der Evaluation der Ergebnisse der Anwendung) hängt die konstruktionsbedingte Validität von den verwendeten Methoden ab.

Die Frage in den Experteninterviews wurden so konzipiert, dass sie sehr gezielte Fragen mit starkem Bezug zu den Unterfragen der \ff enthalten.
Für die erste Teilfrage wurden in einem dedizierten Teil (mit Bezug zur Granularität) Fragen zu Migrationsverfahren und deren Anwendung auf \jf gestellt.
Für die zweite Unterfrage wurden in einem weiteren Teil Fragen zur Eignung der vom \gls{arh} vorgeschlagenen \bpp für \jf gestellt.
Auch hier bezieht sich eine Frage speziell auf den in der \ff genannten Qualitätsaspekt.
Obwohl viele Fragen einen direkten Bezug zur \ff haben, kann die konstruktionsbedingte Validität trotzdem eingeschränkt sein, falls Teilnehmer die Fragen anders verstanden haben.

Die Auswertung der Feldnotizen ist dagegen nur in indirekterer Form auf die \ff ausgerichtet.
Aus den Feldnotizen geht nicht direkt hervor, ob und wie gut die \gls{arh} für das Refactoring von \jf geeignet ist. 
Vielmehr geben sie einen Einblick in die Durchführung des Prozesses und die damit verbundenen Probleme, Emotionen und Anmerkungen. 
Diese Stellung der Methodik ist bei der Betrachtung der Ergebnisse zu berücksichtigen.

\section{Interne Validität}

Die interne Validität ist gegeben, wenn die untersuchten Kausalbeziehungen isoliert sind und andere Kausalitäten, die das Ergebnis ebenfalls beeinflussen, bekannt sind und berücksichtigt werden \cite{Runeson2009}.

In \cref{fig:limitations} wurde versucht, mögliche Auswirkungen von erwünschten Messfaktoren auf die Schritte in dieser Arbeit sowie von unerwünschten Störfaktoren auf diese zu skizzieren.
Als Hauptstörfaktoren werden die \emph{beteiligten Personen} sowie die \emph{verwendete Methodik und Vorgehensweise} vermutet. 
Dies sind jeweils nur Oberbegriffe für Sammlungen von möglichen Einflüssen.

\emph{Beteiligte Personen} bezieht sich primär darauf, welche Personen beteiligt waren.
Es sollte aber auch beinhalten, welche Probleme diese Personen mitbringen (Emotionen, Differenzen, Machtunterschiede).
Es kann auch den Grad der Erfahrung der Personen, menschliches Versagen und viele andere Einflüsse beinhaltet sein, die von den beteiligten Personen ausgehen.

Der Begriff \emph{Verwendete Methodik und Vorgehensweise} bezieht sich auf den gewählten Ansatz zur Durchführung der Maßnahme.
Im Falle der Wahl der \glspl{qa} ist dies die in \cref{sec:methodik-architekturreview} beschriebene Methodik des Architekturreviews.
In den anderen Schritten bezieht es sich auf weniger formale Vorgehensweisen, beispielsweise auf die gemeinsame Filterwahl oder die Art und Weise, wie verschiedene Suchergebnisse aggregiert werden.
All diese Vorgehensweisen können Auswirkungen auf die Ergebnisse haben, die es zu minimieren gilt. 

\begin{figure}
	\centering
	\includegraphics[width=1.0\textwidth]{limitations.drawio}
	\caption[Kausale Beziehungen der Fallstudie]{
			Kausale Beziehungen zum Ergebnis dieser Fallstudie und generell der Anwendung des \gls{arh}.
%			Einflüsse der Qualität des \gls{mmf} und \gls{arh} sowie andere Einflüsse auf das Ergebnis der Fallstudie.
		}
	\label{fig:limitations}
\end{figure}

\section{Reliabilität}

Für eine hohe Verlässlichkeit (original \emph{Reliability}) ist es wichtig, dass die Subjektivität des Autors möglichst wenig Einfluss auf die Ergebnisse der Studie hat \cite{Runeson2009}.
Im Falle dieser Arbeit ist die Reliabilität in einigen Punkten eingeschränkt.

Zum einen wurden die meisten Schritte in dieser Arbeit von nur einem Studenten durchgeführt, der nur über begrenzte Erfahrung in Software Engineering und wissenschaftlichem Arbeiten verfügt.
Dieses Risiko wurde versucht zu minimieren, indem in zumeist wöchentlichen Abständen mit dem \gls{po} von \jf und dem universitären Betreuer die durchgeführten und nächsten Schritte besprochen wurden.
Außerdem wurde das Architekturreview zur Erfassung der \glspl{qa}, die die späteren Ergebnisse maßgeblich beeinflussen, mit insgesamt fünf wichtigen Mitgliedern des Entwicklungsteams durchgeführt.
Damit sollte eine möglichst hohe Qualität der \glspl{qa} sichergestellt werden.
Insbesondere bei zu komplexen Details ist diese Strategie des Vier-Augen-Prinzips nicht möglich, da eventuell helfende Personen keine Zeit für die notwendige Einarbeitung hätten. 
Konkret ist dieses Problem in der Arbeit bei der Anwendung des Migrationsverfahrens in \cref{sec:anwendung-verfahren} aufgetreten.
Hier haben ein Algorithmus und Formeln auf einer sehr tiefen Detailebene zu Problemen geführt, die vom Autor nicht in der begrenzten Zeit gelöst werden konnten.
Gleichzeitig waren sie jedoch zu komplex für andere Personen, damit diese eventuell hätten helfen können.

Die gleiche Kritik gilt für die Methodik der Feldnotizen.
Diese wurden nur von einer Person erstellt und ausgewertet, die keine Erfahrung mit dieser Methode hatte.
Die Erstellung von Feldnotizen ist unabhängig von der Unerfahrenheit mit einem gewissen Maß an Subjektivität verbunden.
Um diese Einschränkung zu minimieren, wurden für die Erhebung und Auswertung der Feldnotizen etablierte Methoden ausgewählt und vor der Anwendung umfangreiche Recherchen durchgeführt.

Eine weitere Einschränkung der Verlässlichkeit ergibt sich bei der Durchführung der Interviews. 
Sowohl die Befragten als auch der Interviewer könnten an einer positiven Bewertung der Arbeit interessiert sein, wodurch gewollt oder ungewollt mehr positive Ergebnisse produziert werden könnten.
Außerdem können die persönlichen Beziehungen zu den Interviewteilnehmern und damit verbundene Sympathien und Antipathien Auswirkungen auf die Ergebnisse haben.
Um die Objektivität zu erhöhen, wurden die Befragten explizit aufgefordert, ehrliche Antworten zu geben. 
Trotzdem kann eine gewisse Einschränkung nicht ausgeschlossen werden.

Darüber hinaus stellt die Auswertung der Daten generell ein Risiko für die Zuverlässigkeit dar. 
Diese wurde nur von einer unerfahrenen Person durchgeführt. 
Insbesondere bei der Kodierung der Daten ist eine gewisse Subjektivität unvermeidlich.
Bei der Kodierung der Feldnotizen wurde versucht, einen möglichst objektiven Kodierleitfaden zu definieren.
Die Kodierung der Experteninterviews ist naturgemäß subjektiv.
Da die Kodierung in diesem Fall aber auch nicht das Hauptergebnis ist, sondern die textliche Auswertung der Interviews wesentlich genauer beschrieben und fokussiert wird, stellt dies kein relevantes Problem dar.

\section{Externe Validität}

Die externe Validität beschreibt das Ausmaß, in dem die Ergebnisse der Studie verallgemeinert werden können \cite{Runeson2009}. 
In diesem Fallbeispiel bedeutet dies konkret, inwieweit von den Ergebnissen dieser Arbeit auf mögliche andere Anwendungen der \gls{arh} geschlossen werden kann.
Externe Validität kann in der Regel nicht durch logische Maßnahmen sichergestellt werden, sondern nur durch die Annahme von Gesetzmäßigkeiten optimiert werden \cite{campbell2015experimental}. %TODO genauer beschreiben

Die externe Validität ist hoch, wenn andere Anwendungen unter ähnlichen Bedingungen statt\-fin\-den \cite{campbell2015experimental}.
Daher sind die variierenden Eingaben bei Verwendung des \gls{arh} eine wichtige Einschränkung.
Die Ergebnisse der Suche nach Migrationsverfahren und der Suche nach \bpp hängen vollständig von den gewählten \glspl{qa} und Filteroptionen ab.
Wenn andere Eingaben verwendet werden, kann dies zu einer anderen Reihenfolge der Suchergebnisse und folgend anderen Ergebnissen in der Anwendung führen.

Darüber hinaus wird die externe Validität im Bezug auf die allgemeine Anwendung des \gls{arh} durch den Spezialfall in dieser Thesis, dass die Ausgangsarchitektur eine \gls{msa} ist, weiter eingeschränkt.
Dadurch besteht eine geringere Ähnlichkeit zu anderen Anwendungsfällen mit Monolithen als Ausgangsarchitektur, was nach \Citet{campbell2015experimental} zu einer schlechteren externen Validität in diesen Fällen führt.
Auf der anderen Seite erhöht genau diese Bedingung die externe Validität in Bezug auf die Verwendung des \gls{arh} mit \glspl{msa} im Vergleich zu beispielsweise der Fallstudie von \Citet{master-marvin-knodel}.

Die einzige Maßnahme, die zur Verbesserung der externen Validität ergriffen werden kann, ist die Sicherstellung einer möglichst hohen Konstruktvalidität, internen Validität und Reliabilität, sodass zumindest die Voraussetzungen für die externe Validität optimiert sind.

%Neben der Qualität des \gls{mmf} und \gls{arh} gibt es viele andere Faktoren, die das Ergebnis im Einzelfall und auch in dieser Fallstudie beeinflussen können:
%
%\begin{itemize}
%	\item Die Auswahl der Migrationsmethode wird beeinflusst durch
%	\begin{itemize}
%		\item Qualität der \glspl{qa}
%		\item Qualität der Filter
%		\item Gleichgewicht zwischen \glspl{qa} und Filtern
%	\end{itemize}
%\end{itemize}
%
%\begin{figure}
%	\centering
%	\includegraphics[width=1.0\textwidth]{limitations.drawio}
%	\caption[Einflüsse auf das Ergebnis der Fallstudie]{
%		Einflüsse der Qualität des \gls{mmf} und \gls{arh} sowie andere Einflüsse auf das Ergebnis der Fallstudie.
%	}
%	\label{fig:limitations}
%\end{figure}
%
%Phase 1:
%Das Architekturreview ist für die Migration zu einem Microservices-System von Monolithen gedacht.
%Insbesondere existiert dabei das Zielprodukt noch nicht in der Ziel Architektur. 
%In unserem Fall dagegen war die Ausgangsarchitektur schon gleich der Zielarchitektur.
%Dadurch war es bei dem Architekturreview schwierig, die technische Schwierigkeit eines Szenarios einzuschätzen, da einige der Szenarios so schon umgesetzt sind und damit argumentiert werden kann, dass es dann nicht technisch schwierig ist, da es schon vorhanden ist.
%
%Nicht möglich gewesen, dass alle Stakeholder beim Architekturreview anwesend sind: Kein Kunde  
%- Arbeit mit \gls{arh} wurde von unerfahrenem Softwareentwickler durchgeführt und nur für größere Schritte und Zusammenfassungen der Einzelheiten erfahrenen und wichtigeren Stakeholder konsultiert
%
%\section{Interne Validität}
%
%- hohe Komplexität bei den Experteninterviews
%- Ein Experte hat an der Thesis mitgearbeitet
%- Fragen bei den Interviews teils sehr hypothetisch
%- Interviewer ohne Erfahrung
%
%\begin{itemize}
%	\item Begrenzter Rahmen durch Bachelorarbeit
%\end{itemize}
%
\chapter{Fazit}
\label{chap:fazit}

Das Ziel dieser Thesis war es, die \acrfull{msa} des Produkts \jf der Firma \emph{levigo solutions} mithilfe des \acrfull{mmf} der Abteilung \acrfull{ese} des \acrfull{iste} zu überarbeiten.
Insbesondere sollte dabei die Effektivität und Effizienz des \gls{mmf} mit zugehörigem Tool \acrfull{arh} untersucht werden.

\section{Ausblick}


\printbibliography

Alle URLs wurden zuletzt am %17.\,03.\,2018
TODO geprüft.

%\renewcommand{\appendixtocname}{Anhang}
%\renewcommand{\appendixname}{Anhang}
%\renewcommand{\appendixpagename}{Anhang}
\appendix
%\input{latexhints-german}

\pagestyle{empty}
\renewcommand*{\chapterpagestyle}{empty}
\Versicherung
\end{document}
