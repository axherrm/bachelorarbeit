% !TeX document-id = {8e8e29d3-1f25-424a-acf7-0f37323948d8}
% !TeX spellcheck = de-DE
% !TeX encoding = utf8
% !TeX program = lualatex
% !BIB program = biber
% -*- coding:utf-8 mod:LaTeX -*-

% vv  scroll down to line 200 for content  vv


% Template von: https://github.com/latextemplates/scientific-thesis-template/tree/main
% Example des Tempaltes: https://latextemplates.github.io/scientific-thesis-template/main-german.pdf

\let\ifdeutsch\iftrue
\let\ifenglisch\iffalse
\input{pre-documentclass}
\documentclass[
  ngerman,
  % fontsize=11pt is the standard
  a4paper,  % Standard format - only KOMAScript uses paper=a4 - https://tex.stackexchange.com/a/61044/9075
  twoside,  % we are optimizing for both screen and two-side printing. So the page numbers will jump, but the content is configured to stay in the middle (by using the geometry package)
  bibliography=totoc,
  %               idxtotoc,   %Index ins Inhaltsverzeichnis
  %               liststotoc, %List of X ins Inhaltsverzeichnis, mit liststotocnumbered werden die Abbildungsverzeichnisse nummeriert
  headsepline,
  cleardoublepage=empty,
  parskip=half,
  %               draft    % um zu sehen, wo noch nachgebessert werden muss - wichtig, da Bindungskorrektur mit drin
  draft=false
]{scrbook}
\input{config}

% EN: The package scientific-thesis-cover (https://ctan.org/pkg/scientific-thesis-cover) was added to CTAN on January 1, 2018.
%     It is available in recent texlive and miktex installations
\usepackage[
title={Toolunterstütztes Refactoring von Microservices-Architekturen: Eine industrielle Fallstudie},
author={Axel Herrmann},
type=bachelor,
institute={Institut für Software Engineering\\Empirical Software Engineering\\~\\Universitätsstraße 38\\D–70569 Stuttgart},
course={Software Engineering},
examiner={Prof.\ Dr.\ Stefan Wagner},
supervisor={Jonas Fritzsch,\\Frank Intorp},
startdate={25.\ September 2023},
enddate={25.\ März 2024},
logo=uni-logo.png
]{scientific-thesis-cover}
%\usepackage{amssymb} % moved to config.tex before import of "unicode-math"

% Hier stehen alle Abkürzungen
\newacronym{mmf}{MMF}{Microservices Migration Framework}
\newacronym{ese}{ESE}{Empirisches Software Engineering}
\newacronym{arh}{ARH}{Architecture Refactoring Helper}
\newacronym{rest}{REST}{Restful State Transfer}
\newacronym{ci}{CI}{Continuous Integration}
\newacronym{cd}{CD}{Continuous Delivery}

\makeindex

\begin{document}

%tex4ht-Konvertierung verschönern
\iftex4ht
  % tell tex4ht to create picures also for formulas starting with '$'
  % WARNING: a tex4ht run now takes forever!
  \Configure{$}{\PicMath}{\EndPicMath}{}
  %$ % <- syntax highlighting fix for emacs
  \Css{body {text-align:justify;}}

  %conversion of .pdf to .png
  \Configure{graphics*}
  {pdf}
  {\Needs{"convert \csname Gin@base\endcsname.pdf
      \csname Gin@base\endcsname.png"}%
    \Picture[pict]{\csname Gin@base\endcsname.png}%
  }
\fi

%\VerbatimFootnotes %verbatim text in Fußnoten erlauben. Geht normalerweise nicht.

% DE: wird fuer Tabellen benötigt (z.B. >{centering\RBS}p{2.5cm} erzeugt einen zentrierten 2,5cm breiten Absatz in einer Tabelle
\newcommand{\RBS}{\let\\=\tabularnewline}

% EN: To avoid issues with Springer's \mathplus
%     See also http://tex.stackexchange.com/q/212644/9075
\providecommand\mathplus{+}

% DE: typoraphisch richtige Abkürzungen
\newcommand{\zB}{z.\,B.\xspace}
\newcommand{\bzw}{bzw.\xspace}
\newcommand{\usw}{usw.\xspace}
\renewcommand{\dh}{d.\,h.\xspace}

% EN: from hmks makros.tex - \indexify
\newcommand{\toindex}[1]{\index{#1}#1}

% DE: Tipp aus "The Comprehensive LaTeX Symbol List"
\newcommand{\dotcup}{\ensuremath{\,\mathaccent\cdot\cup\,}}

% DE: Anstatt $|x|$ $\abs{x}$ verwenden.
%     Die Betragsstriche skalieren automatisch, falls "x" etwas größer sein sollte...
\newcommand{\abs}[1]{\left\lvert#1\right\rvert}

% DE: für Zitate
\newcommand{\citeS}[2]{\cite[S.~#1]{#2}}
\newcommand{\citeSf}[2]{\cite[S.~#1\,f.]{#2}}
\newcommand{\citeSff}[2]{\cite[S.~#1\,ff.]{#2}}
\newcommand{\vgl}{vgl.\ }
\newcommand{\Vgl}{Vgl.\ }

% EN: For the algorithmic package
\newcommand{\commentchar}{\ensuremath{/\mkern-4mu/}}
\algrenewcommand{\algorithmiccomment}[1]{\hfill $\commentchar$ #1}

% DE: Seitengrößen - Gegen Schusterjungen und Hurenkinder...
\newcommand{\largepage}{\enlargethispage{\baselineskip}}
\newcommand{\shortpage}{\enlargethispage{-\baselineskip}}

\newcommand{\initialism}[1]{%
  \ifdeutsch%
    \textsc{#1}\xspace%
  \else%
    \textlcc{#1}\xspace%
  \fi%
}
\newcommand{\OMG}{\initialism{OMG}}
\newcommand{\BPEL}{\initialism{BPEL}}
\newcommand{\BPMN}{\initialism{BPMN}}
\newcommand{\UML}{\initialism{UML}}

% green arrow up for improvement
\newcommand{\advantage}{\raisebox{-0.35\height}{\includegraphics[width=.04\textwidth]{up\_icon.png}}}
%\newcommand{\advantage}{\textcolor{green}{{\Large $\mathbf{\uparrow}$}}}
% red arrow down for
\newcommand{\disadvantage}{\raisebox{-0.35\height}{\includegraphics[width=.04\textwidth]{down\_icon.png}}}
%\newcommand{\disadvantage}{\textcolor{red}{$\mathbf{\downarrow}$}}

% Trick to use more than 9 args: https://tex.stackexchange.com/a/509766
\newcommand{\neworrenewcommand}[1]{\providecommand{#1}{}\renewcommand{#1}}

\newcommand{\fieldnote}[9]{
	\neworrenewcommand{\fieldnoteextended}[2]{
		\begin{longtable}{|p{2.5cm}|p{11cm}|}
			%\caption{#1} \label{tab:field-notes-long-#2} \\
			\caption{#1} \label[feldnotiz]{feldnotiz:#2} \\
			
			\hline \textbf{Eintragstyp} & \textbf{Inhalt} \\ \hline
			\endfirsthead
			
			\multicolumn{2}{c}%
			{{\bfseries \tablename\ \thetable{} -- von der vorherigen Seite weitergeführt}} \\
			\hline \textbf{Eintragstyp} & \textbf{Inhalt}
			\endhead
			
			\hline \multicolumn{2}{|r|}{{Wird auf der n\"achsten Seite fortgef\"uhrt}} \\ \hline
			\endfoot
			
			\hline
			\endlastfoot
			
			Feldnotitz Nr. & #2 \\ \hline
			Datum, Uhrzeit & #3 \\ \hline
			Ort &  #4 \\ \hline
			Beteiligte Personen & #5 \\ \hline
			Phase \& Schritt des MMF & #6 \\ \hline
			Aktionen und Entscheidungen &  #7 \\ \hline
			Kommentare &  #8 \\ \hline
			Was lief gut &  #9 \\ \hline
			Probleme &  ##1 \\ \hline
			Empfindungen & ##2 \\ \hline
			
		\end{longtable}
	}
	\fieldnoteextended
}

%\newcommand{\fieldnote}[9]{
%	\neworrenewcommand{\fieldnoteextended}[2]{
%		\begin{table}
%			\centering
%			\begin{tabular}{|l|p{9cm}|}
%				\hline \textbf{Eintragstyp} & \textbf{Inhalt} \\ \hline
%				Feldnotitz Nr. & #2 \\ \hline
%				Datum, Uhrzeit & #3 \\ \hline
%				Ort &  #4 \\ \hline
%				Beteiligte Personen & #5 \\ \hline
%				Phase \& Schritt des MMF & #6 \\ \hline
%				Aktionen und Entscheidungen &  #7 \\ \hline
%				Kommentare &  #8 \\ \hline
%				Was lief gut &  #9 \\ \hline
%				Probleme &  ##1 \\ \hline
%				Empfindungen & ##2 \\ \hline
%			\end{tabular}
%			\caption{#1} \label{tab:field-notes-#2}			
%		\end{table}
%	}
%	\fieldnoteextended
%}

% https://tex.stackexchange.com/a/136050
\newenvironment{shortitemize}{
	\begin{itemize}
		\setlength{\itemsep}{0pt}
		\setlength{\parskip}{0pt}
		\setlength{\parsep}{0pt}     }
	{ \end{itemize}                  } 

% Mark content of the filter table with prio 1 and prio 2
\newcommand{\prioOne}[1]{\ul{\textbf{#1}}}
\newcommand{\prioTwo}[1]{\ul{#1}}

% shortcut for jadice flow
\newcommand{\jf}{\emph{jadice flow}\xspace}

% shortcut for jadice flow
\newcommand{\bpp}{Best Practices und Patterns\xspace}

% commands for results of search
\newcommand{\resultref}[1]{%
\hyperref[res:#1]{#1}\xspace%
}

\newcommand{\result}[1]{%
\hyperref[res:#1]{Er\-geb\-nis #1}%
\ifx1#1 \cite{arh-result-no-filter-1}\fi%
\ifx2#1 \cite{arh-result-no-filter-3}\fi%
\ifx3#1 \cite{arh-result-no-filter-2}\fi%
\ifx4#1 \cite{arh-result-no-filter-4}\fi%
\ifx5#1 \cite{arh-result-no-filter-5}\fi%
\ifx6#1 \cite{arh-result-important-filter-4}\fi%
\ifx7#1 \cite{arh-result-important-filter-7}\fi%
\ifx8#1 \cite{arh-result-no-qas}\fi%
\xspace%
}

% remove whitespace before paragraph
\let\oldparagraph\paragraph
\renewcommand{\paragraph}[1]{\vspace*{-0.5cm}\oldparagraph{#1}}
\pagenumbering{arabic}
\Titelblatt

%Eigener Seitenstil fuer die Kurzfassung und das Inhaltsverzeichnis
\deftriplepagestyle{preamble}{}{}{}{}{}{\pagemark}
%Doku zu deftriplepagestyle: scrguide.pdf
\pagestyle{preamble}
\renewcommand*{\chapterpagestyle}{preamble}



%Kurzfassung / abstract
%auch im Stil vom Inhaltsverzeichnis
\ifdeutsch
  \section*{Kurzfassung}
\else
  \section*{Abstract}
\fi

Aus vielen verschiedenen Gründen ist das Architekturprinzip von Microservices im letzten Jahrzehnt immer verbreiteter geworden.
Für viele Produkte kann es sinnvoll sein, von einer monolithischen Architektur zu einer Microservices-basierten zu migrieren.
Doch eine grundlegende Architektur eines Produkts zu verändern, ist ein sehr zeit- und kostspieliger Prozess, der viele Risiken birgt.
Da viele Entwickler dabei nicht auf wissenschaftliche Literatur zurückgreifen, wurde in vorherigen Arbeiten ein Werkzeug entwickelt, das den Migrationsprozess mit wissenschaftlichen Erkenntnissen unterstützen soll.
Im Rahmen dieser Thesis wurde dieses Werkzeug erstmalig zum Refactoring eines reellen Produkts hinsichtlich spezieller Design-Aspekte angewandt.
Dabei konnten quantitative Ergebnisse durch den Vergleich der resultierenden Prototypen mit dem Ausgangsprodukt über vordefinierte Vergleichskriterien und Messmethoden erfasst werden.
Außerdem wurde der Migrationsprozess als Ganzes qualitativ bewertet.

\cleardoublepage


% BEGIN: Verzeichnisse

\iftex4ht
\else
  \microtypesetup{protrusion=false}
\fi

%%%
% Literaturverzeichnis ins TOC mit aufnehmen, aber nur wenn nichts anderes mehr hilft!
% \addcontentsline{toc}{chapter}{Literaturverzeichnis}
%
% oder zB
%\addcontentsline{toc}{section}{Abkürzungsverzeichnis}
%
%%%

%Produce table of contents
%
%In case you have trouble with headings reaching into the page numbers, enable the following three lines.
%Hint by http://golatex.de/inhaltsverzeichnis-schreibt-ueber-rand-t3106.html
%
%\makeatletter
%\renewcommand{\@pnumwidth}{2em}
%\makeatother
%
\tableofcontents

% Bei einem ungünstigen Seitenumbruch im Inhaltsverzeichnis, kann dieser mit
% \addtocontents{toc}{\protect\newpage}
% an der passenden Stelle im Fließtext erzwungen werden.

\listoffigures
\listoftables

%Wird nur bei Verwendung von der lstlisting-Umgebung mit dem "caption"-Parameter benoetigt
%\lstlistoflistings
%ansonsten:
\ifdeutsch
  \listof{Listing}{Verzeichnis der Listings}
\else
  \listof{Listing}{List of Listings}
\fi

%mittels \newfloat wurde die Algorithmus-Gleitumgebung definiert.
%Mit folgendem Befehl werden alle floats dieses Typs ausgegeben
\ifdeutsch
  \listof{Algorithmus}{Verzeichnis der Algorithmen}
\else
  \listof{Algorithmus}{List of Algorithms}
\fi
%\listofalgorithms %Ist nur für Algorithmen, die mittels \begin{algorithm} umschlossen werden, nötig

% Abkürzungsverzeichnis
\printnoidxglossaries

\iftex4ht
\else
  %Optischen Randausgleich und Grauwertkorrektur wieder aktivieren
  \microtypesetup{protrusion=true}
\fi

% END: Verzeichnisse


% Headline and footline
\renewcommand*{\chapterpagestyle}{scrplain}
\pagestyle{scrheadings}
\pagestyle{scrheadings}
\ihead[]{}
\chead[]{}
\ohead[]{\headmark}
\cfoot[]{}
\ofoot[\usekomafont{pagenumber}\thepage]{\usekomafont{pagenumber}\thepage}
\ifoot[]{}


%% vv  scroll down for content  vv %%

%%%%%%%%%%%%%%%%%%%%%%%%%%%%%%%%%%%%%%%%%%%%%%%%%%%%%%%%%%%%%%%%%%%%%%%%%%%%%%
%
% Main content starts here
%
%%%%%%%%%%%%%%%%%%%%%%%%%%%%%%%%%%%%%%%%%%%%%%%%%%%%%%%%%%%%%%%%%%%%%%%%%%%%%%

\chapter{Einleitung}
\label{chap:einleitung}

In einer Zeit, in der viele Softwarelösungen auf Cloud Computing basieren, ist auch die Microservices-Architektur immer gebräuchlicher geworden.
Viele Produkte wurden innerhalb des letzten Jahrzehnts schon in diese Architektur übergeführt, hauptsächlich mit dem Ziel, von Vorteilen wie besserer Skalierbarkeit, Wartbarkeit und Flexibilität zu profitieren~\cite{Fritzsch_2019,taibi2017processmotivations}.
Abhängig von dem Ausmaß und der Komplexität monolithischer Systeme, kann die Migration in eine Microservices-Architektur jedoch sehr zeitaufwendig, kostenintensiv und kompliziert sein.
Es liegen aktuell eine Vielzahl akademischer Forschungsarbeiten zum Thema Microservices vor, die jedoch nur selten in der Industrie benutzt werden~\cite{fritzsch2022architecturecentric}.

Um diese Migration zu vereinfachen und die wissenschaftliche Expertise der Industrie leichter zugänglich zu machen, hat das ESE das \gls{mmf} entwickelt.
Mit der Entwicklung des \gls{mmf}, das Erkenntnisse über die Migration von monolithischen Anwendungen hin zu Microservices in die Industrie bringen soll, hat das ESE erste Schritte unternommen, um die erwähnte Lücke zwischen Industrie und Forschung zu schließen.

Offen ist bisher jedoch die Anwendung des auf dem Framework basierenden Werkzeugs in der Industrie, durch die das Framework und Werkzeug weiterentwickelt und verbessert werden kann.
Im Rahmen dieser Bachelorarbeit wird das \acrshort{mmf} daher erstmals an einem realen Produkt, \emph{jadice flow}, evaluiert.
Dazu wird das Framework zum Refactoring des Produktes verwendet.
Die Ergebnisse dieses Refactorings werden zur Bewertung der Effektivität des Frameworks und Werkzeuges verwendet, wodurch diese in Zukunft weiter verbessert werden sollen.

Bei der Analyse des Produkts mithilfe des Frameworks werden vor allem drei Designaspekte von Microservices betrachtet:
[1] Die optimale Service Granularität, bei der zwischen Overhead durch zu viele Einheiten und Overhead durch zu große Einheiten abgewogen werden muss.
[2] Paradigma der Ansteuerung der einzelnen Einheiten, Kommunikationsmodell zwischen den Einheiten.
[3] Reduktion des IO-Flaschenhalses, Art des Payload-Transfers.
Der IO-Flaschenhals entsteht dadurch, dass die einzelnen Einheiten, bevor sie ihre Arbeit verrichten können, immer die Eingangsdokumente herunterladen und anschließend die Ergebnisse wieder hochladen müssen.

Daraus folgen die Forschungsfragen, die in dieser Thesis untersucht werden:

\begin{itemize}
	\item \label[forschungsfrage]{forschungsfrage:1} \emph{RQ 1:} Wie kann eine bereits bestehende Microservices-Architektur mit Hilfe des \acrfull{mmf} hinsichtlich konkreter Qualitätsaspekte weiter optimiert werden?
	\begin{enumerate}[i.]
		\item Welche Refactoring-Verfahren eignen sich zur Bestimmung der optimalen Service Granularität einer bestehenden Microservices-Architektur?
		\item Welche Ansätze, Patterns oder Best Practices eignen sich zur Optimierung des Kommunikationsmodells und der Verringerung des IO-Flaschenhalses zwischen den einzelnen Services?
	\end{enumerate}
\end{itemize}

Um die definierten Forschungsfragen zu beantworten, werden zwei Ziele für diese Arbeit definiert:
Im ersten Teil wird das \gls{mmf} benutzt, um die Architektur von \emph{jadice flow} zu überarbeiten.
Dabei wird innerhalb von drei Phasen ein Plan für eine neue Architektur des Produkts entworfen.
Diese Phasen beinhalten eine Sammlung von wissenschaftlich untersuchten Methoden zur Migration zu Microservices sowie passender Pattern und Best Practices.

Im zweiten Teil wird diese neue Architektur praktisch umgesetzt.
Da ein komplettes Refactoring des Systems nicht dem Rahmen dieser Arbeit angemessen wäre, wird lediglich ein Prototyp mit der neuen Architektur umgesetzt.
Bestandteil des zweiten Teils ist deswegen ebenfalls, einen geeigneten Umfang für diesen Prototyp zu finden, sowie Messmethoden festzulegen, die einen Vergleich des Produkts vor und nach dem Refactoring ermöglichen.

Um die gesetzten Ziele zu erreichen und damit die Forschungsfragen zu beantworten, ist diese Thesis in die folgenden Kapitel aufgeteilt:
Anfangs wird in \textbf{\cref{chap:theoretischer-hintergrund}} in den theoretischen Hintergrund eingeführt, der für diese Arbeit relevant ist. 
Es werden auch das \gls{mmf} und \emph{jadice flow} kurz beschrieben. 
Ebenfalls werden ähnliche Arbeiten vorgestellt.
In \textbf{\cref{chap:methodik}} wird die Methodik beschrieben, mit der diese Thesis durchgeführt werden soll und die Forschungsfragen beantwortet werden sollen.
Mit Verwendung des \gls{mmf} wird in \textbf{\cref{chap:anwendung}} am Produkt \emph{jadice flow} die Planung eines Refactorings durchgeführt. 
Es werden die drei Phasen des Frameworks ausgeführt, wodurch schlussendlich ein neuer Architekturplan entsteht, der in \textbf{\cref{chap:implementierung}} prototypisch implementiert wird.
Der entstandene Prototyp dient dann in \textbf{\cref{chap:auswertung}} zum quantitativen Vergleich zwischen dem Produkt vor und nach Refactoring.
Abschließend werden in \textbf{\cref{chap:gueltigkeit}} Einschränkungen und Gültigkeit dieser Arbeit diskutiert und in \textbf{\cref{chap:fazit}} ein Fazit gezogen und der Ausblick diskutiert.  
\chapter{Theoretischer Hintergrund}
\label{chap:theoretischer-hintergrund}

Es liegt bereits eine Vielzahl an akademischen Publikationen zum Architekturprinzip der Microservices vor.
In diesem Kapitel wird in den theoretischen Hintergrund der für diese Thesis relevanten Literatur eingeführt.
Zuerst wird ein kleiner Vergleich zwischen dem häufig genutzten Legacy-Architekturprinzip \emph{Monolith} und dem behandelten neuen Architekturprinzip \emph{Microservices} gezogen.
Danach beschreiben wir aktuelle Arbeiten zur Migration zu Microservices-Architekturen.
Außerdem führen wir in das Werkzeug \gls{arh} ein, das großer Bestandteil und Untersuchungsobjekt dieser Arbeit ist.
Abschließend wird ein Überblick über das Produkt gegeben, das mithilfe des \gls{arh} überarbeitet werden soll.

\section{Vergleich von monolithischen und Microservices-Systemen}

Die monolithische Architektur von Systemen hat lange Zeit einen Großteil der Softwareprodukte ausgemacht und ist der einfachste Weg, Software zu entwickeln.
Die gesamte Funktionalität eines Monolithen befindet sich in einer Applikation, die in sich selbst geschlossen ist. 
Diese Architektur hat durchaus Stärken.
Da es eine sehr einfache Architektur ist, können kleine Anwendungen sehr schnell erstellt, getestet und lauffähig gemacht werden.
Es wird kein kompliziertes Kommunikationsmodell benötigt, da Monolithen oft in einer Programmiersprache und sogar mit nur einem Framework geschrieben sind.

Je größer Monolithen werden, desto stärken zeigen sich allerdings die Schwächen dieser Architektur.
Es entstehen riesige Codebasen und kleine Änderungen erzwingen erneutes Bauen und Testen der gesamten Applikation.
Das steht auch dem modernen agilen Arbeitsprinzip im Weg.
Falls nur einzelne Komponenten eines Monolithen hohe Last erfahren, muss trotzdem das ganze System skaliert werden, was zu verschwendeter Rechenleistung führen kann und die Applikationen ineffizient macht.

Die meisten dieser Probleme werden durch die Microservices-Architektur gelöst.
Dabei besteht ein System aus vielen kleinen (Micro) Services.
Diese zeichnen sich dadurch aus, dass sie möglichst klein gehalten werden und genau eine Aufgabe haben.
Durch die Entkopplung der Services wird im Gegensatz zu Monolithen ein Kommunikationsmodell zwischen den Services nötig.
Dafür existieren verschiedene Lösungen, die im Rahmen dieser Arbeit noch näher untersucht werden, doch ein sehr häufig benutztes ist beispielsweise \gls{rest}.

Die Vorteile dieser Architektur gegenüber Monolithen sind vielseitig.
Da die Services unabhängig voneinander entworfen sind, können sie einzeln implementiert, getestet und gebaut werden.
Dadurch ist die Automatisierung des Bauens und Testens durch \gls{ci} / \gls{cd} einfacher möglich.
Außerdem können Services je nach Last einzeln skaliert werden, was die Effizienz in manchen Szenarien stark erhöht.
Des weiteren sind Entwickler flexibel in der Wahl der Programmiersprache und des Frameworks für einzelne Services, da die einzige Anforderung die Berücksichtigung des gewählten Kommunikationsmodells ist.

Wie bereits erwähnt gibt es allerdings auch Punkte, in denen Monolithen vorteilhafter sind, wodurch eine Microservices-Architektur nicht universell den Monolithen vorzuziehen ist.
Dazu gehört vor allem die Komplexität des Systems, die mit Microservices-Architektur wesentlich erhöht ist.
Das liegt an der Aufteilung in einzelne Services, die ein ausgefeiltes Kommunikationsmodell benötigen.
Die Kommunikation zwischen den Komponenten ist in jedem Fall langsamen und ineffizienter als das Pendant dazu in Monolithen, wo lediglich Funktionen im selben ausgeführten Programm aufgerufen werden.

\section{Migration zu Microservices}

Aufgrund der genannte Vorteile der Microservices-Architektur gegenüber Monolithen, besteht großes Interesse daran, Monolithen zu Microservices zu migrieren.
Der Eingriff in tiefe Architekturprinzipien eines Systems ist natürlich keine einfache Operation.
Deswegen ist die Migration zu Microservices seit einiger Zeit ein stark untersuchtes Thema.

In dieser Thesis wurde sehr vordergründig das Microservices Migration Framework des ESE  untersucht.
Dieses ist aufgrund vorheriger Arbeiten wie \Citet{10.1007/978-3-030-06019-0_10} entstanden.

In 2019 kategorisieren \Citet{10.1007/978-3-030-06019-0_10} zehn verschiedene Migrations-Methoden.
Dabei wird ein Mängel an praktisch anwendbaren Methoden hervorgehoben, die gute Werkzeugunterstützung und Metriken zur Verifikation der Ergebnisse bieten.

Als Folge darauf stellen \Citet{fritzsch2022architecturecentric} 2022 dann die Planung ein solchen Frameworks vor. Der Migrationsprozess mit Framework und dessen Arbeitsweise in drei Phasen wird bereits beschrieben. Auch wird die Entwicklung eines Web-basierten Werkzeugs erwähnt, das das Framework benutzt und Entwickler bei der Migration unterstützt.

\Citet{10.1145/3242163.3242164} beschreiben geläufige Gefahren bei der Migration zu Microservices-Architekturen.
Dabei  werden fünf architektonische Fehlermuster und vier Fehlermuster bei der Migration aufgeführt.
Außerdem werden Lösungsmöglichkeiten für jedes Muster beschrieben.

%\section{Refactoring}
%
%Obwohl in dieser Thesis ein Framework zur Migration zu Microservices im Mittelpunkt steht, sieht man schnell, dass die Migration eng verwandt mit dem Refactoring ist.
%Deswegen werden folgend ein paar Quellen vorgestellt, die das Refactoring von Microservices-Architekturen behandeln.
%
%\Citet{https://doi.org/10.1002/spe.2974} beschreiben in einem Artikel zur $\mu$TOSCA toolchain, wie sie mit dem  $\mu$TOSCA Modell Architekturen von Microservices-basierten Systemen mit dem OASIS Standard TOSCA darstellen können.
%Außerdem beschreiben sie die Möglichkeit, das $\mu$TOSCA Modell eines Systems durch das zugehörige Kubernetes Deployment automatisch zu ermitteln.
%Das $\mu$TOSCA Modell einer Applikation soll dann dazu genutzt werden, potentielle architektonische Probleme zu finden.
%Für das Ermitteln des $\mu$TOSCA Modells und das Identifizieren von Architektur Smells stellen sie zwei Werkzeug Prototypen ($\mu$Miner und $\mu$Freshener) vor.

\section{\acrfull{arh}}

Aufgrund der beschriebenen Vorteile von Microservices Architekturen ist es im Interesse vieler Unternehmen und Entwickler, existierende Applikationen und Systeme mit monolithischer Architektur zu Microservices-basierten Systemen zu migrieren.
Der Planungsprozess dieser Migration ist allerdings sehr zeit- und kostenintensiv und durch große Risiken geprägt, da dabei Systeme sehr tiefliegend verändert werden.
Erschwert wird er durch das Fehlen klarer und strukturierter Anleitungen.
Es wurde bereits durch mehrere Metastudien versucht, eine Übersicht über verschiedene Refactoring- und Migrationsverfahren zu geben, die für Entwickler hilfreich sein kann.
Jedoch bleibt dabei das Problem bestehen, dass sich der aktuelle Stand der Literatur schnell ändert und Metastudien dadurch teilweise obsolet werden.
Dieses Problem könnte durch die Umsetzung der Anleitung in Form eines Werkzeugs, das Entwickler bei der Migration unterstützt, gelöst werden.
Im Gegensatz zu Metastudien könnte dieses Werkzeug dynamisch an den neuesten Forschungsstand angepasst werden.

Aus diesem Grund hat das \gls{ese} das Werkzeug \gls{arh} \cite{arh-github} entwickelt, das in diesem Kapitel vorgestellt wird.
Während \Citet{fritzsch2022architecturecentric} den Entwurf des Frameworks für das Werkzeug beschreiben, wurde das Werkzeug im Rahmen von zwei Masterarbeiten entwickelt \cite{master-daniel-koch}\cite{master-daniel-koch2}\cite{master-tobias-haller}.
Der \gls{arh} ist eine web-basierte Anleitung, die Entwickler durch drei Phasen der Migrationsplanung führt.
Eine Übersicht über diese Phasen und die Funktionsweise des Werkzeugs ist in \cref{fig:arh-overview} zu sehen.
Im Folgenden werden diese drei Phasen genauer beschrieben: 

\begin{itemize}
	\item \textbf{Phase 1: System Comprehension:}
	In Phase 1 wird das Verständnis des Systems, das migriert werden soll, angestrebt. 
	Mit den Stakeholdern sollen dabei strategische Ziele für Produkt und Unternehmen definiert, sowie Qualitätsmerkmale und gewünschte Szenarien identifiziert werden.
	Es sollen sich dadurch mögliche Treiber für die neue Microservices-Architektur herauskristallisieren.
	Durch das resultierende Verständnis des Systems wird ein Vergleich der alten monolithischen Architektur und einer potentiellen neuen Microservices-Architektur möglich.
	Architekten sollen dann mithilfe dieser Informationen eine Entscheidung für oder gegen die Migration zu Microservices fällen.
	\item \textbf{Phase 2: Strategy Definition:}
	Falls sich in Phase 1 für die Migration zu Microservices entschieden wurde, wird in Phase 2 die Planung der Migration begonnen.
	Eine der zwei hauptsächlichen Aufgaben in dieser Phase ist die konkrete Definition, welche Migrationsstrategie benutzt werden soll, um das System zu modernisieren.
	Dafür kann zwischen (momentan) 22 verschiedenen Methoden gewählt werden, die aus akademischen Publikationen stammen.
	Das Werkzeug kann Entwickler dabei unterstützen, indem es aufgrund der Eingaben aus Phase 1 empfohlene Methoden vorschlägt.
	Der zweite Teil dieser Phase besteht darin, einen Service-Identifikationsansatz zu wählen.
	Durch diesen kann in der nächsten Phase eine sinnvolle Aufteilung in einzelne Services bestimmt werden.
	\item \textbf{Phase 3a: Architecture Definition:}
	Phase 3 ist in zwei Teile aufgeteilt. 
	Im ersten Abschnitt wird die neue Architektur definiert. 
	Hier wird, basierend auf der Methode zur Migration, die in der Planungsphase gewählt wurde, die Identifikation von Microservice-Kandidaten realisiert.
	Außerdem wird allgemeiner die Architektur des gesamten Systems geplant, wobei das Werkzeug Entwickler durch eine Liste von vorgeschlagenen Patterns und Best Practices unterstützen kann.
	 Phase 3a und 3b sind eng verbunden, denn durch Erkenntnisse in der Implementierung kann die Planung häufig noch mehrmals überarbeitet werden und dadurch eine neue Implementierung begonnen werden. 
	\item \textbf{Phase 3b: Service Implementation:} In Phase 3b startet dann ein Zyklus von Implementierungen der in Phase 3a definierten Services.
	Bei Implementierung selbst kann das Werkzeug Entwickler nicht unterstützen.
	Doch Bestandteil der Entwicklungszyklen ist auch, dass das entstehende System anhand der Qualitätsmerkmale zu bewerten.
	Dadurch kann eine unpassende Architekturdefinition oder auch eine unpassende Aufteilung in Services frühzeitig erkannt und korrigiert werden.
	Ist diese Phase abgeschlossen, ist eine erste Version des migrierten Systems fertig.
\end{itemize}

\begin{figure}
	\centering
	\includegraphics[width=\textwidth]{figures/ese-mmf-overview.png}
	\caption[Architecture Refactoring Helper Übersicht]{
		Übersicht über den \acrlong{arh} des \gls{ese} \cite{fritzsch2022architecturecentric}. Es sind die drei Phasen des Prozesses zu erkennen, sowie einige Schritte der einzelnen Phasen.
	}
	\label{fig:arh-overview}
\end{figure}

\section{jadice flow}

Das vorgestellte Werkzeug soll in dieser Thesis nun erstmals in der Industrie benutzt werden.
Deswegen wird in diesem Abschnitt näher auf das Produkt \emph{jadice flow} eingegangen, das damit überarbeitet werden soll.

\emph{jadice flow} ist ein Produkt der Firma \emph{levigo solutions}\footnote{\url{https://solutions.levigo.de/}}.
\emph{levigo solutions} beschäftigt über 30 Mitarbeiter, von denen ungefähr sechs seit vier Jahren an \emph{jadice flow} arbeiten.
\emph{jadice flow} basiert bereits auf einer Microservices-Architektur, ist jedoch erst die erste Generation nach der Überführung des vorherigen Monoliths \emph{jadice server}\footnote{\url{https://jadice.com/produkte/server/}}.
Beide Produkte bieten Workflow-basierte Dokumentenverarbeitung an, bei der mit komplexen Datenströmen umgegangen werden kann.
Durch den Umstieg auf eine Microservices-Architektur können in der neuen Generation einzelne Verarbeitungsschritte skaliert werden.
Der Großteil der Services ist in Java geschrieben, doch durch den Betrieb in Containern ist jadice flow prinzipiell unabhängig von der Programmiersprache.

Einen beispielhaften Workflow stellt der \emph{E-Mail converter} dar.
Dabei werden E-Mails in Einzelteile aufgeteilt (Körper, Anhänge) und diese dann einzeln in das Zielformat konvertiert.
Die einzelnen Arbeitsschritte dabei können parallelisiert werden und separat skaliert werden.
Am Ende werden die Ergebnisse kombiniert.
Im Gegensatz zu einfacher PDF-Darstellung von E-Mails in geläufigen Programmen bietet \emph{jadice flow} zusätzliche Features, wie die tiefe Analyse und Darstellung von Archiven in Anhängen.

Da \emph{jadice flow} die erste Generation in Microservices-Architektur ist, wird noch einiges Verbesserungspotential vermutet.
Durch die Analyse von \emph{jadice flow 1.0} mithilfe des Werkzeugs und weiterer Forschung im Rahmen dieser Bachelorarbeit soll ein Refactoring-Prozess für \emph{jadice flow 2.0} angestoßen werden.

Da das Framework für die Anwendung auf Monolithen gedacht ist, soll in dieser Arbeit die Anwendung des Frameworks auf ein bereits migriertes System angewendet werden, um dessen Eigenschaften weiter zu verbessern.
Dadurch wird erforscht, inwiefern sich Framework und Werkzeug auf eine bereits existierende Microservice-Architektur anwenden lassen.
Falls es deshalb für diesen Anwendungsfall zielführend ist, wird zusätzlich die ursprüngliche monolithische Anwendung von \emph{jadice flow} (\emph{jadice server}) hinzugezogen.
Die Ergebnisse dessen könnten dann mit \emph{jadice flow} verglichen werden.




\chapter{Forschungsfragen}

[1] Wie kann eine bereits bestehende Microservices-Architektur mit Hilfe des ESE Microservices Migration Frameworks hinsichtlich konkreter Qualitätsaspekte weiter optimiert werden?
\begin{enumerate}[i.]
	\item Welche Refactoring-Verfahren eignen sich zur Bestimmung der optimalen Service Granularität einer bestehenden Microservices-Architektur?
	\item Welche Ansätze, Patterns oder Best Practices eignen sich zur Optimierung des Kommunikationsmodells und der Verringerung des IO-Flaschenhalses zwischen den einzelnen Services?
\end{enumerate}

\chapter{Methodik}
\label{chap:methodik}

\section{Aufbau}
\section{Vergleichskriterien \& Messmethoden}

\input{chapters/5_anwendung}
\input{chapters/6_erstellung_prototyp}
\input{chapters/7_leistungsanalyse_prototyp}
\input{chapters/8_auswertung}
\input{chapters/9_fazit}


\printbibliography

Alle URLs wurden zuletzt am %17.\,03.\,2018
TODO geprüft.

%\renewcommand{\appendixtocname}{Anhang}
%\renewcommand{\appendixname}{Anhang}
%\renewcommand{\appendixpagename}{Anhang}
\appendix
%\input{latexhints-german}

\pagestyle{empty}
\renewcommand*{\chapterpagestyle}{empty}
\Versicherung
\end{document}
