% DE: wird fuer Tabellen benötigt (z.B. >{centering\RBS}p{2.5cm} erzeugt einen zentrierten 2,5cm breiten Absatz in einer Tabelle
\newcommand{\RBS}{\let\\=\tabularnewline}

% EN: To avoid issues with Springer's \mathplus
%     See also http://tex.stackexchange.com/q/212644/9075
\providecommand\mathplus{+}

% DE: typoraphisch richtige Abkürzungen
\newcommand{\zB}{z.\,B.\xspace}
\newcommand{\bzw}{bzw.\xspace}
\newcommand{\usw}{usw.\xspace}
\renewcommand{\dh}{d.\,h.\xspace}

% EN: from hmks makros.tex - \indexify
\newcommand{\toindex}[1]{\index{#1}#1}

% DE: Tipp aus "The Comprehensive LaTeX Symbol List"
\newcommand{\dotcup}{\ensuremath{\,\mathaccent\cdot\cup\,}}

% DE: Anstatt $|x|$ $\abs{x}$ verwenden.
%     Die Betragsstriche skalieren automatisch, falls "x" etwas größer sein sollte...
\newcommand{\abs}[1]{\left\lvert#1\right\rvert}

% DE: für Zitate
\newcommand{\citeS}[2]{\cite[S.~#1]{#2}}
\newcommand{\citeSf}[2]{\cite[S.~#1\,f.]{#2}}
\newcommand{\citeSff}[2]{\cite[S.~#1\,ff.]{#2}}
\newcommand{\vgl}{vgl.\ }
\newcommand{\Vgl}{Vgl.\ }

% EN: For the algorithmic package
\newcommand{\commentchar}{\ensuremath{/\mkern-4mu/}}
\algrenewcommand{\algorithmiccomment}[1]{\hfill $\commentchar$ #1}

% DE: Seitengrößen - Gegen Schusterjungen und Hurenkinder...
\newcommand{\largepage}{\enlargethispage{\baselineskip}}
\newcommand{\shortpage}{\enlargethispage{-\baselineskip}}

\newcommand{\initialism}[1]{%
  \ifdeutsch%
    \textsc{#1}\xspace%
  \else%
    \textlcc{#1}\xspace%
  \fi%
}
\newcommand{\OMG}{\initialism{OMG}}
\newcommand{\BPEL}{\initialism{BPEL}}
\newcommand{\BPMN}{\initialism{BPMN}}
\newcommand{\UML}{\initialism{UML}}

% green arrow up for improvement
\newcommand{\advantage}{\raisebox{-0.35\height}{\includegraphics[width=.04\textwidth]{up\_icon.png}}}
%\newcommand{\advantage}{\textcolor{green}{{\Large $\mathbf{\uparrow}$}}}
% red arrow down for
\newcommand{\disadvantage}{\raisebox{-0.35\height}{\includegraphics[width=.04\textwidth]{down\_icon.png}}}
%\newcommand{\disadvantage}{\textcolor{red}{$\mathbf{\downarrow}$}}

% Trick to use more than 9 args: https://tex.stackexchange.com/a/509766
\newcommand{\neworrenewcommand}[1]{\providecommand{#1}{}\renewcommand{#1}}

\newcommand{\fieldnote}[9]{
	\neworrenewcommand{\fieldnoteextended}[2]{
		\begin{longtable}{|p{2.5cm}|p{11cm}|}
			%\caption{#1} \label{tab:field-notes-long-#2} \\
			\caption{#1} \label[feldnotiz]{feldnotiz:#2} \\
			
			\hline \textbf{Eintragstyp} & \textbf{Inhalt} \\ \hline
			\endfirsthead
			
			\multicolumn{2}{c}%
			{{\bfseries \tablename\ \thetable{} -- von der vorherigen Seite weitergeführt}} \\
			\hline \textbf{Eintragstyp} & \textbf{Inhalt}
			\endhead
			
			\hline \multicolumn{2}{|r|}{{Wird auf der n\"achsten Seite fortgef\"uhrt}} \\ \hline
			\endfoot
			
			\hline
			\endlastfoot
			
			Feldnotitz Nr. & #2 \\ \hline
			Datum, Uhrzeit & #3 \\ \hline
			Ort &  #4 \\ \hline
			Beteiligte Personen & #5 \\ \hline
			Phase \& Schritt des MMF & #6 \\ \hline
			Aktionen und Entscheidungen &  #7 \\ \hline
			Kommentare &  #8 \\ \hline
			Was lief gut &  #9 \\ \hline
			Probleme &  ##1 \\ \hline
			Empfindungen & ##2 \\ \hline
			
		\end{longtable}
	}
	\fieldnoteextended
}

%\newcommand{\fieldnote}[9]{
%	\neworrenewcommand{\fieldnoteextended}[2]{
%		\begin{table}
%			\centering
%			\begin{tabular}{|l|p{9cm}|}
%				\hline \textbf{Eintragstyp} & \textbf{Inhalt} \\ \hline
%				Feldnotitz Nr. & #2 \\ \hline
%				Datum, Uhrzeit & #3 \\ \hline
%				Ort &  #4 \\ \hline
%				Beteiligte Personen & #5 \\ \hline
%				Phase \& Schritt des MMF & #6 \\ \hline
%				Aktionen und Entscheidungen &  #7 \\ \hline
%				Kommentare &  #8 \\ \hline
%				Was lief gut &  #9 \\ \hline
%				Probleme &  ##1 \\ \hline
%				Empfindungen & ##2 \\ \hline
%			\end{tabular}
%			\caption{#1} \label{tab:field-notes-#2}			
%		\end{table}
%	}
%	\fieldnoteextended
%}

% https://tex.stackexchange.com/a/136050
\newenvironment{shortitemize}{
	\begin{itemize}
		\setlength{\itemsep}{0pt}
		\setlength{\parskip}{0pt}
		\setlength{\parsep}{0pt}     }
	{ \end{itemize}                  } 

% Mark content of the filter table with prio 1 and prio 2
\newcommand{\prioOne}[1]{\underline{\smash{\textbf{#1}}}}
\newcommand{\prioTwo}[1]{\underline{\smash{#1}}}
